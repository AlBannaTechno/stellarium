%% Part of Stellarium User Guide
%% Status:
%% 2015-12-30 taken from wiki
%% 2016-04-06 GZ reformatted a bit. 
%% TODO: document the more obscure entries. Add links to the respective chapters.
%% TODO: remove unused entries, add new ones.  

%\chapterimage{chapter-t1-bg} % Chapter heading image

\chapter{Configuration File}
\label{sec:config.ini}
See \ref{sec:ConfigurationFile}{The
Main Configuration File} for information about this file, including its
default installed location, and command line options that can
affect how it is processed.

%Deprecated parameters are marked by gray background. 
%Possible new parameters are marked by yellow background.
%% GZ This is no longer true. Reactivate somehow? 

The file \file{config.ini} (or a file which you can load instead with
the \command{-\/-config <file>} option) is structured into the following
parts. In addition, plugins can add a section named like the plugin
(Exception: The Text User Interface plugin's section is named [tui]
for historical reasons).

\section{\big[astro\big]}
\label{sec:config.ini:astro}

This section includes settings for the commonly displayed objects.

\begin{longtabu} to \textwidth {l|l|l|X}
\toprule
\emph{ID} & \emph{Type} & \emph{Default} & \emph{Description}\\\midrule
apparent\_magnitude\_algorithm & string & Harris & Set algorithm for computation of apparent magnitude of the planets. 
                                                   Possible values: \emph{Planesas}, \emph{Mueller}, \emph{Harris} 
                                                   and \emph{Generic}\footnote{Available since version 0.13.3}.\\\midrule
nebula\_magnitude\_limit         & float & 8.5  & Value of limiting magnitude for the deep-sky objects.\\
star\_magnitude\_limit           & float &      & Value of limiting magnitude for the stars. 
                                                  Sometimes you don't want to display more stars when zooming in. \\\midrule
flag\_nebula\_magnitude\_limit   & bool & false & Set to \emph{true} to activate limit for showing deep-sky objects.\\\midrule
flag\_star\_magnitude\_limit     & bool &       & Set to \emph{true} to activate limit for showing stars\\\midrule
%% STILL USED?
flag\_extinction\_below\_horizon & bool &       & Set to \emph{true} to apply extinction effects to sky below horizon\\\midrule
extinction\_mode\_below\_horizon & string & zero & Set extinction mode for atmosphere below horizon. 
                                                   Possible values: \emph{zero}, \emph{mirror} and \emph{max}.\\\midrule
flag\_stars               & bool & true  & Set to \emph{false} to hide the stars on start-up\\\midrule
flag\_star\_name          & bool & true  & Set to \emph{false} to hide the star labels on start-up\\\midrule
flag\_planets             & bool & true  & Set to \emph{false} to hide the planet labels on start-up\\\midrule
flag\_planets\_hints      & bool & true  & Set to \emph{false} to hide the planet hints on startup (names and circular highlights)\\\midrule
flag\_planets\_orbits     & bool & false & Set to \emph{true} to show the planet orbits on startup\\\midrule
flag\_permanent\_orbits   & bool & false & Set to \emph{true} to show the orbit of planet, when planet is out of the viewport also.\\\midrule
flag\_planets\_pointers   & bool & false & Set to \emph{true} to show the planet pointer markers on startup\\\midrule
flag\_light\_travel\_time & bool & true  & Should be \emph{true} to improve accuracy in the movement of the planets by compensating 
                                           for the time it takes for light to travel. This has a slight impact on performance, 
                                           but is essential e.g.\ for Jupiter's moons.\\\midrule
flag\_object\_trails      & bool & false & Turns on and off drawing of object trails (which show the movement of the planets over time)\\\midrule
flag\_nebula              & bool & true  & Set to \emph{false} to hide the nebulae on start-up. \\\midrule
flag\_nebula\_name        & bool & false & Set to \emph{true} to show the nebula labels on start-up. \\\midrule
flag\_nebula\_long\_name  & bool &       & Set to \emph{true} to show the nebula long labels on start-up.\\\midrule
%% TODO: Still Used?
flag\_nebula\_display\_no\_texture & bool  & false & Set to \emph{true} to suppress displaying of nebula textures. \\\midrule
nebula\_hints\_amount              & float & 3.0   & Sets the amount of hints [0\ldots10]. \\\midrule
nebula\_labels\_amount             & float & 3.0   & Sets the amount of labels [0\ldots10].\\\midrule
flag\_milky\_way                   & bool  & true  & Set to \emph{false} to hide the Milky Way.\\\midrule
milky\_way\_intensity              & float & 1.0   & Sets the relative brightness with which the milky way is drawn. Typical [1\ldots3]. \\\midrule
flag\_zodiacal\_light              & bool  & true  & Set to \emph{false} to hide the zodiacal light\\\midrule
zodiacal\_light\_intensity         & float & 1.0   & Sets the relative brightness with which the zodiacal light is drawn. \\\midrule
max\_mag\_nebula\_name             & float & 8.0   & Sets the magnitude of the nebulae whose name is shown. \\\midrule
%% GZ The following indeed exists, but as of 0.15pre does nothing useful inside the program. TODO take out?
%nebula\_scale & float & Sets how much to scale nebulae. a setting of \emph{1} will display nebulae at normal size. Default value: \emph{1.0}.\\\midrule
%% GZ NOT USED
%flag\_bright\_nebulae & bool & Set to \emph{true} to increase nebulae brightness to enhance viewing (less realistic)\\\midrule
%flag\_nebula\_ngc & bool & Enables/disables display of all NGC objects\\\midrule
flag\_nebula\_hints\_proportional & bool & false &Enables/disables proportional markers for deep-sky objects. \\\midrule
flag\_surface\_brightness\_usage  & bool & false &Enables/disables usage surface brightness for markers for deep-sky objects.\\\midrule
flag\_use\_type\_filter           & bool & false &Enables/disables usage of the type filters for deep-sky objects. \\\bottomrule
\end{longtabu}

\section{\big[color\big]}
\label{sec:config.ini:color}

This section defines the RGB colors for the various objects, lines,
grids, labels etc. Values are given in float from 0 to
1. e.g. \emph{1.0,1.0,1.0} for white, or \emph{1,0,0} for red. Leave
no whitespace between the numbers!

\begin{longtabu} to \textwidth {l|l|X}
\toprule
\emph{ID} & \emph{Default} & \emph{Colour of the\ldots}\\
\midrule
default\_color           & 0.5,0.5,0.7 & default colour.\\\midrule
azimuthal\_color         & 0.3,0.2,0.1 &  azimuthal grid. \\\midrule
equatorial\_color        & 0.2,0.3,0.8 &  equatorial grid (of date). \\\midrule
equatorial\_J2000\_color & 0.1,0.1,0.5 &  equatorial grid (J2000). \\\midrule
galactic\_color          & 0.0,0.3,0.2 &  galactic grid. \\\midrule
galactic\_equator\_color & 0.5,0.3,0.1 &  galactic equator line. \\\midrule
equator\_color           & 0.3,0.5,1.0 &  equatorial line. \\\midrule
ecliptic\_color          & 0.9,0.6,0.2 &  ecliptic line (J2000). \\\midrule
ecliptic\_J2000\_color   & 0.7,0.2,0.2 &  ecliptic grid (J2000). \\\midrule
meridian\_color          & 0.2,0.6,0.2 &  meridian line. \\\midrule
horizon\_color           & 0.2,0.6,0.2 &  horizon line. \\\midrule
const\_lines\_color      & 0.2,0.2,0.6 &  constellation lines. \\\midrule
const\_names\_color      & 0.4,0.6,0.9 &  constellation names. \\\midrule
const\_boundary\_color   & 0.3,0.1,0.1 &  constellation boundaries. \\\midrule
star\_label\_color       & 0.4,0.3,0.5 &  star labels. \\\midrule
cardinal\_color          & 0.8,0.2,0.1 &  cardinal points. \\\midrule
planet\_names\_color     & 0.5,0.5,0.7 &  planet names. \\\midrule
planet\_orbits\_color    & 0.7,0.2,0.2 &  orbits. \\\midrule
planet\_pointers\_color  & 1.0,0.3,0.3 &  planet pointers. \\\midrule
object\_trails\_color    & 1.0,0.7,0.0 &  planet trails. \\\midrule
telescope\_circle\_color & 0.6,0.4,0.0 &  telescope location indicator. \\\midrule
telescope\_label\_color  & 0.6,0.4,0.0 &  telescope label (next to location circle). \\\midrule
script\_console\_keyword\_color  & 1.0,0.0,1.0 & syntax highlight for keywords in the script console. \\\midrule
script\_console\_module\_color   & 0.0,1.0,1.0 & syntax highlight for modules in the script console. \\\midrule
script\_console\_comment\_color  & 1.0,1.0,0.0 & syntax highlight for comments in the script console. \\\midrule
script\_console\_function\_color & 0.0,1.0,0.0 & syntax highlight for functions in the script console. \\\midrule
script\_console\_constant\_color & 1.0,0.5,0.5 & syntax highlight for constants in the script console. \\\midrule
daylight\_text\_color            & 0.0,0.0,0.0 & info text at daylight. \\\midrule
dso\_label\_color                       & 0.2,0.6,0.7 & deep-sky objects labels. \\\midrule
dso\_circle\_color                      & 1.0,0.7,0.2 & deep-sky objects symbols, if not of the types below. \\\midrule
dso\_galaxy\_color                      & 1.0,0.2,0.2 & galaxies symbols. \\\midrule
dso\_radio\_galaxy\_color               & 0.3,0.3,0.3 & radio galaxies symbols. \\\midrule
dso\_active\_galaxy\_color              & 1.0,0.5,0.2 & active galaxies symbols. \\\midrule
dso\_interacting\_galaxy\_color         & 0.2,0.5,1.0 & interacting galaxies symbols. \\\midrule
dso\_quasar\_color                      & 1.0,0.2,0.2 & quasars symbols. \\\midrule
dso\_possible\_quasar\_color            & 1.0,0.2,0.2 & possible quasars symbols. \\\midrule
dso\_bl\_lac\_color                     & 1.0,0.2,0.2 & BL Lac objects symbols. \\\midrule
dso\_blazar\_color                      & 1.0,0.2,0.2 & blazars symbols. \\\midrule
dso\_nebula\_color                      & 0.1,1.0,0.1 & nebulae symbols. \\\midrule
dso\_planetary\_nebula\_color           & 0.1,1.0,0.1 & planetary nebulae symbols. \\\midrule
dso\_reflection\_nebula\_color          & 0.1,1.0,0.1 & reflection nebulae symbols. \\\midrule
dso\_bipolar\_nebula\_color             & 0.1,1.0,0.1 & bipolar nebulae symbols. \\\midrule
dso\_emission\_nebula\_color            & 0.1,1.0,0.1 & emission nebulae symbols. \\\midrule
dso\_dark\_nebula\_color                & 0.3,0.3,0.3 & dark nebulae symbols. \\\midrule
dso\_hydrogen\_region\_color            & 0.1,1.0,0.1 & hydrogen regions symbols. \\\midrule
dso\_supernova\_remnant\_color          & 0.1,1.0,0.1 & supernovae remnants symbols. \\\midrule
dso\_interstellar\_matter\_color        & 0.1,1.0,0.1 & interstellar matter symbols. \\\midrule
dso\_cluster\_with\_nebulosity\_color   & 0.1,1.0,0.1 & clusters associated with nebulosity symbols. \\\midrule
dso\_molecular\_cloud\_color            & 0.1,1.0,0.1 & molecular clouds symbols. \\\midrule
dso\_possible\_planetary\_nebula\_color & 0.1,1.0,0.1 & possible planetary nebulae symbols. \\\midrule
dso\_protoplanetary\_nebula\_color      & 0.1,1.0,0.1 & protoplanetary nebulae symbols. \\\midrule
dso\_cluster\_color                     & 1.0,1.0,0.1 & star clusters symbols. \\\midrule
dso\_open\_cluster\_color               & 1.0,1.0,0.1 & open star clusters symbols. \\\midrule
dso\_globular\_cluster\_color           & 1.0,1.0,0.1 & globular star clusters symbols. \\\midrule
dso\_stellar\_association\_color        & 1.0,1.0,0.1 & stellar associations symbols. \\\midrule
dso\_star\_cloud\_color                 & 1.0,1.0,0.1 & star clouds symbols. \\\midrule
dso\_star\_color                        & 1.0,0.7,0.2 & star symbols. \\\midrule
dso\_emission\_object\_color            & 1.0,0.7,0.2 & emission objects symbols. \\\midrule
dso\_young\_stellar\_object\_color      & 1.0,0.7,0.2 & young stellar objects symbols. \\
\bottomrule
\end{longtabu}

\section{\big[custom\_selected\_info\big]}
\label{sec:config.ini:custom_selected_info}

You can fine-tune the bits of information to display for the selected object in this section. Set the entry to \emph{true} to display it.

\begin{longtabu} to \textwidth {l|l|X}\toprule
\emph{ID} & \emph{Type} & \emph{Description}\\\midrule
flag\_show\_absolutemagnitude & bool & absolute magnitude for objects.\\\midrule
flag\_show\_altaz             & bool & horizontal coordinates for objects.\\\midrule
flag\_show\_catalognumber     & bool & catalog designations for objects.\\\midrule
flag\_show\_distance          & bool & distance to object.\\\midrule
flag\_show\_extra             & bool & extra info for object.\\\midrule
flag\_show\_hourangle         & bool & hour angle for object.\\\midrule
flag\_show\_magnitude         & bool & magnitude for object.\\\midrule
flag\_show\_name              & bool & common name for object.\\\midrule
flag\_show\_radecj2000        & bool & equatorial coordinates (J2000) of object.\\\midrule
flag\_show\_radecofdate       & bool & equatorial coordinates (of date) of object.\\\midrule
flag\_show\_size              & bool & size of object.\\\midrule
flag\_show\_galcoord          & bool & galactic coordinates (System~II) of object.\\\midrule
flag\_show\_eclcoord          & bool & ecliptic coordinates (J2000 and of date) of object.\\\midrule
flag\_show\_type              & bool & type of object\\\bottomrule
\end{longtabu}

%% TODO: Maybe separate flag_show_eclcoord and flag_show_eclcoordj2000?

\section{\big[custom\_time\_correction\big]}
\label{sec:config.ini:custom_time_correction}

Stellarium allows experiments with $\Delta T$. See \ref{sec:Concepts:DeltaT} for details.

\begin{longtabu} to \textwidth {l|l|X}\toprule
\emph{ID}    & \emph{Type} & \emph{Description}\\\midrule
coefficients & [float,float,float] & Coefficients for custom equation of DeltaT\\\midrule
ndot & float & n-dot value for custom equation of DeltaT\\\midrule
year & int   & Year for custom equation of DeltaT\\\bottomrule
\end{longtabu}

\section{\big[devel\big]}
\label{sec:config.ini:devel}

This section is for developers only. 
%% TODO: Still it has to be documented!

\begin{longtabu} to \textwidth {l|l|X}\toprule
\emph{ID}              & \emph{Type} & \emph{Description}\\\midrule
convert\_dso\_catalog        & bool & Set to \emph{true} to convert file \file{catalog.txt} 
                                      into file \file{catalog.dat}. Default value: \emph{false}.\\\midrule
convert\_dso\_decimal\_coord & bool & Set to \emph{true} to use decimal values for coordinates 
                                      in source catalog. Default value: \emph{true}.\\\bottomrule
\end{longtabu}

\section{\big[dso\_catalog\_filters\big]}
\label{sec:config.ini:dso_catalog_filters}
In this section you can fine-tune which of the deep-sky catalogs should be selected on startup.

\begin{longtabu} to \textwidth {l|l|l|X}\toprule
\emph{ID} & \emph{Type} & \emph{Default} & \emph{Description}\\\midrule
flag\_show\_ngc & bool & true  & New General Catalogue (NGC). \\\midrule
flag\_show\_ic  & bool & true  & Index Catalogue (IC). \\\midrule
flag\_show\_m   & bool & true  & Messier Catalog (M). \\\midrule
flag\_show\_c   & bool & false & Caldwell Catalogue (C). \\\midrule
flag\_show\_b   & bool & false & Barnard Catalogue (B). \\\midrule
flag\_show\_sh2 & bool & false & Sharpless Catalogue (Sh-II). \\\midrule
flag\_show\_vdb & bool & false & Van den Bergh Catalogue of reflection nebulae (VdB). \\\midrule
flag\_show\_rcw & bool & false & The RCW catalogue of H$\alpha$-emission regions in the southern Milky Way. \\\midrule
flag\_show\_lbn & bool & false & Lynds' Catalogue of Bright Nebulae (LBN). \\\midrule
flag\_show\_ldn & bool & false & Lynds' Catalogue of Dark Nebulae (LDN). \\\midrule
flag\_show\_cr  & bool & false & Collinder Catalogue (Cr). \\\midrule
flag\_show\_mel & bool & false & Melotte Catalogue of Deep Sky Objects (Mel).  \\\midrule
flag\_show\_pgc & bool & false & HYPERLEDA. I. Catalog of galaxies (PGC). \\\midrule
flag\_show\_ced & bool & false & Cederblad Catalog of bright diffuse Galactic nebulae (Ced). \\\midrule
flag\_show\_ugc & bool & false & The Uppsala General Catalogue of Galaxies (UGC). \\\bottomrule
\end{longtabu}

\section{\big[dso\_type\_filters\big]}
\label{sec:config.ini:dso_type_filters}

\begin{longtabu} to \textwidth {l|l|l|X}\toprule
\emph{ID} & \emph{Type} & \emph{Default} & \emph{Description}\\\midrule
flag\_show\_galaxies              & bool & true & display galaxies.\\\midrule
flag\_show\_active\_galaxies      & bool & true & display active galaxies. \\\midrule
flag\_show\_interacting\_galaxies & bool & true & display interacting galaxies.  \\\midrule
flag\_show\_clusters              & bool & true & display star clusters.  \\\midrule
flag\_show\_bright\_nebulae       & bool & true & display bright nebulae.  \\\midrule
flag\_show\_dark\_nebulae         & bool & true & display dark nebulae.  \\\midrule
flag\_show\_planetary\_nebulae    & bool & true & display planetary nebulae.  \\\midrule
flag\_show\_hydrogen\_regions     & bool & true & display hydrogen regions. \\\midrule
flag\_show\_supernova\_remnants   & bool & true & display supernovae remnants. \\\midrule
flag\_show\_other                 & bool & true & display other deep-sky objects.  \\\bottomrule
\end{longtabu}

\section{\big[gui\big]}\label{sec:config.ini:gui}

\begin{longtabu} to \textwidth {l|l|l|X}
\toprule
\emph{ID} & \emph{Type} & \emph{Default} & \emph{Description}\\\midrule
base\_font\_size & int    & 13 & Sets the font size. Typical value: \emph{15}\\\midrule
base\_font\_name & string & Verdana (Windows)    & Selects the name for base font\\
                 &        & DejaVu Sans (others) & \\\midrule
%safe\_font\_name & string & Selects the name for safe font, e.g. \emph{Verdana}\\\midrule
base\_font\_file & string & & Selects the name for font file, e.g. \emph{DejaVuSans.ttf}\\\midrule
flag\_show\_fps                 & bool   & true & see at how many frames per second Stellarium is rendering\\\midrule
flag\_show\_fov                 & bool   & true & see how many degrees your vertical field of view is\\\midrule
%% TODO: Is this required? timeout=0 also keeps cursor visible.
flag\_mouse\_cursor\_timeout    & bool  & true  & Set to \emph{false} if you want to have cursor visible at all times.\\\midrule
mouse\_cursor\_timeout          & float & 10    & Set to \emph{0} if you want to keep the mouse cursor visible at all times. 
                                                  non-0 values mean the cursor will be hidden after that many seconds of inactivity\\\midrule
flag\_show\_flip\_buttons       & bool  & false & Enables/disables display of the image flipping buttons in the main 
                                                  toolbar (see section \ref{sec:gui:configuration:tools})\\\midrule
flag\_show\_nebulae\_background\_button & bool & false & Set to \emph{true} if you want to have access to the 
                                                         button for enabling/disabling backgrounds for deep-sky objects\\\midrule
flag\_use\_degrees              & bool   &&\\\midrule
selected\_object\_info          & string & all  & Values: \emph{all}, \emph{short}, \emph{none}, 
                                                  and \emph{custom} (since V0.12.0; see~\ref{sec:gui:configuration:info}).\\\midrule
auto\_hide\_horizontal\_toolbar & bool   & true & Set to \emph{true} if you want auto hide the horizontal toolbar.\\\midrule
auto\_hide\_vertical\_toolbar   & bool   & true & Set to \emph{true} if you want auto hide the vertical toolbar.\\\midrule
%% NO LONGER HERE!
%day\_key\_mode                  & string & Specifies the amount of time which is added and subtracted when the 
%                                                    {[} {]} - and = keys are pressed - calendar days, or sidereal days. 
%                                                    This option only makes sense for Digitalis planetariums. %% WHY?
%                                                    Values: \emph{calendar} or \emph{sidereal}\\\midrule
flag\_use\_window\_transparency & bool   & false & If \emph{false}, show menu bars opaque\\\midrule
flag\_show\_datetime            & bool   & true  & If \emph{true}, display date and time in the bottom bar\\\midrule
flag\_time\_jd                  & bool   & false & If \emph{true}, use JD format for time in the bottom bar\\\midrule
flag\_show\_location            & bool   & true  & If \emph{true}, display location in the bottom bar\\\midrule
flag\_fov\_dms                  & bool   & false & If \emph{true}, using DMS format for FOV in the bottom bar\\\midrule
flag\_show\_decimal\_degrees    & bool   & false & If \emph{true}, use decimal degrees for coordinates\\\midrule
flag\_use\_azimuth\_from\_south & bool   & false & If \emph{true}, calculate azimuth from south towards west 
                                                   (as in some astronomical literature)\\\midrule
flag\_show\_gui                 & bool   & false & If \emph{true}, display GUI\\\bottomrule
\end{longtabu}

\section{\big[init\_location\big]}\label{sec:config.ini:init_location}

\begin{longtabu} to \textwidth {l|l|X}\toprule
\emph{ID} & \emph{Type} & \emph{Description}\\\midrule
landscape\_name   & string & Sets the landscape you see. Built-in options are \emph{garching, geneva, grossmugl, guereins, 
                             hurricane, jupiter, mars, moon, neptune, ocean, saturn, trees, uranus, zero}.\\\midrule
location          & string & Name of location on which to start stellarium.\\\midrule
last\_location    & string & Coordinates of last used location in stellarium.\\\bottomrule
\end{longtabu}

\section{\big[landscape\big]}\label{sec:config.ini:landscape}

\begin{longtabu} to \textwidth {l|l|X}\toprule
\emph{ID} & \emph{Type} & \emph{Description}\\\midrule
atmosphere\_fade\_duration      & float & Sets the time (seconds) it takes for the atmosphere to fade when de-selected\\\midrule
flag\_landscape                 & bool & Set to false if you don't want to see the landscape at all\\\midrule
flag\_fog                       & bool & Set to \emph{false} if you don't want to see fog on start-up\\\midrule
flag\_atmosphere                & bool & Set to \emph{false} if you don't want to see atmosphere on start-up\\\midrule
flag\_landscape\_sets\_location & bool & Set to \emph{true} if you want Stellarium to modify the observer location 
                                         when a new landscape is selected (changes planet and longitude/latitude/altitude 
                                         if location data is available in the landscape.ini file)\\\midrule
minimal\_brightness                  & float & Set minimal brightness for landscapes. [0\ldots1] Typical value: \emph{0.01}\\\midrule
atmospheric\_extinction\_coefficient & float & Set atmospheric extinction coefficient $k$ [mag/airmass]\\\midrule
temperature\_C                       & float & Set atmospheric temperature [Celsius]\\\midrule
pressure\_mbar                       & float & Set atmospheric pressure [mbar]\\\bottomrule
\end{longtabu}

\section{\big[localization\big]}\label{sec:config.ini:localization}

\begin{longtabu} to \textwidth {l|l|X}\toprule
\emph{ID} & \emph{Type} & \emph{Description}\\\midrule
sky\_culture & string & Sets the sky culture to use. E.g. \emph{western, polynesian, egyptian, chinese, 
                        lakota, navajo, inuit, korean, norse, tupi, maori, aztec, sami}.\\\midrule
sky\_locale & string & Sets language used for names of objects in the sky (e.g. planets). 
                       The value is a short locale code, e.g. \emph{en, de, en\_GB}\\\midrule
app\_locale & string & Sets language used for Stellarium's user interface. 
                       The value is a short locale code, e.g. \emph{en, de, en\_GB}.\\\midrule
time\_zone & string  & Sets the time zone. Valid values: \emph{system\_default}, 
                       or some region/location combination, e.g. \emph{Pacific/Marquesas}\\\midrule
time\_display\_format & string & time display format: can be \emph{system\_default}, \emph{24h} or \emph{12h}.\\\midrule
date\_display\_format & string & date display format: can be \emph{system\_default}, \emph{mmddyyyy}, \emph{ddmmyyyy} or \emph{yyyymmdd} (ISO8601).\\\bottomrule
\end{longtabu}

\section{\big[main\big]}\label{sec:config.ini:main}

\begin{longtabu} to \textwidth {l|l|X}\toprule
\emph{ID}                 & \emph{Type} & \emph{Description}\\\midrule
invert\_screenshots\_colors & bool & If \emph{true}, Stellarium will saving the screenshorts with inverted colors.\\\midrule
restore\_defaults           & bool & If \emph{true}, Stellarium will restore default settings at startup. 
                                     This is equivalent to calling Stellarium with the \command{--restore-defaults} option.\\\midrule
screenshot\_dir             & string & Path for saving screenshots\\\midrule
version                     & string & Version of Stellarium. This parameter may be used to detect necessary changes in config.ini file, do not edit.\\\midrule
use\_separate\_output\_file & bool   & Set to \emph{true} if you want to create a new file for script output for each start of Stellarium\\\midrule
check\_requirements         & bool   & Set to \emph{false} if you want to disable and permanently ignore checking hardware requirements at startup. 
                                       Expect problems if hardware is below requirements!\\
\bottomrule
\end{longtabu}

\section{\big[navigation\big]}\label{sec:config.ini:navigation}

This section controls much of the look\&feel of Stellarium. Be careful if you change something here.

%% TODO: improve description of the various zoom settings.

\begin{longtabu} to \textwidth {l|l|X}\toprule
\emph{ID}              & \emph{Type} & \emph{Description}\\\midrule
preset\_sky\_time        & float & Preset sky time used by the dome version. Unit is Julian Day. Typical value: \emph{2451514.250011573}\\\midrule
startup\_time\_mode      & string & Set the start-up time mode, can be \emph{actual} (start with current real world time), 
                                    or \emph{Preset} (start at time defined by preset\_sky\_time)\\\midrule
flag\_enable\_zoom\_keys & bool & Set to \emph{false} if you want to disable the zoom\\\midrule
flag\_manual\_zoom       & bool & Set to \emph{false} for normal zoom behaviour as described in this guide. 
                                  When set to true, the auto zoom feature only moves in a small amount and must be pressed many times\\\midrule
flag\_enable\_move\_keys & bool & Set to \emph{false} if you want to disable the arrow keys\\\midrule
flag\_enable\_mouse\_navigation & bool & Set to \emph{false} if you want to disable the mouse navigation.\\\midrule
init\_fov                       & float & Initial field of view, in degrees, typical value:'' 60''\\\midrule
init\_view\_pos                 & floats & Initial viewing direction. This is a vector with x,y,z-coordinates. x being N-S (S +ve), 
                                  y being E-W (E +ve), z being up-down (up +ve). Thus to look South at the horizon use \emph{1,0,0}. 
                                  To look Northwest and up at $45\degree$, use \emph{-1,-1,1} and so on.\\\midrule
auto\_move\_duration            & float & Duration for the program to move to point at an object when the space bar is pressed. Typical value: \emph{1.4}\\\midrule
mouse\_zoom                     & float & Sets the mouse zoom amount (mouse-wheel)\\\midrule
move\_speed                     & float & Sets the speed of movement\\\midrule
zoom\_speed                     & float & Sets the zoom speed\\\midrule
viewing\_mode                   & string & If set to \emph{horizon}, the viewing mode simulate an alt/azi mount, 
                                           if set to \emph{equator}, the viewing mode simulates an equatorial mount\\\midrule
flag\_manual\_zoom              & bool & Set to \emph{true} if you want to auto-zoom in incrementally.\\\midrule
auto\_zoom\_out\_resets\_direction & bool & Set to \emph{true} if you want to auto-zoom restoring direction.\\\midrule
time\_correction\_algorithm     & string  & Algorithm of DeltaT correction.\\\bottomrule %% TODO: list values! And write ref. chapter!
\end{longtabu}

\section{\big[plugins\_load\_at\_startup\big]}
\label{sec:config.ini:plugins_load_at_startup}

This section lists which plugins are loaded at startup (those with
\emph{true} values). Each plugin can add another section into this
file with its own content, which is described in the respective plugin
documentation, see \ref{ch:Plugins}. You activate loading of plugins
in the \key{F2} settings dialog, tab ``Plugins''. After selection of
which plugins to load, you must restart Stellarium.

\begin{longtabu} to \textwidth {l|l|X}
\toprule
\emph{ID} & \emph{Type} & \emph{Description}\\\midrule
AngleMeasure          & bool & Angle Measure plugin\\\midrule
ArchaeoLines          & bool & ArchaeoLines plugin\\\midrule
CompassMarks          & bool & Compass Marks plugin\\\midrule
MeteorShowers         & bool & Meteor Showers plugin \\\midrule
Exoplanets            & bool & Exoplanets plugin \\\midrule
Observability         & bool & Observability Analysis\\\midrule
Oculars               & bool & Oculars plugin \\\midrule
Pulsars               & bool & Pulsars plugin \\\midrule
Quasars               & bool & Quasars plugin \\\midrule
Satellites            & bool & Satellites plugin \\\midrule
SolarSystemEditor     & bool & Solar System Editor plugin\\\midrule
Supernovae            & bool & Historical Supernovae plugin \\\midrule
TelescopeControl      & bool & Telescope Control plugin \\\midrule
TextUserInterface     & bool & Text User Interface plugin \\\midrule
TimeZoneConfiguration & bool & Time Zone plugin \\\midrule
Novae                 & bool & Bright Novae plugin \\\midrule
Scenery3dMgr          & bool & Scenery 3D plugin \\\bottomrule
\end{longtabu}

\section{\big[projection\big]}\label{sec:config.ini:projection}

This section contains the projection of your choice and several
advanced settings important if you run Stellarium on a single screen,
multi-projection, dome projection, or other setups.

\begin{longtabu} to \textwidth {l|l|X}\toprule
\emph{ID} & \emph{Type} & \emph{Description}\\\midrule
type & string & Sets projection mode. Values: \emph{ProjectionPerspective,
                ProjectionEqualArea, ProjectionStereographic, ProjectionFisheye,
                ProjectionHammer, ProjectionCylinder, ProjectionMercator,
                ProjectionOrthographic}, or \emph{StelProjectorSinusoidal}.\\\midrule
viewport & string & How the view-port looks. Values: \emph{none} (regular rectangular screen), 
                    \emph{disk} (circular screen, useful for planetarium setup).\\\midrule
viewportMask & string & How the view-port looks. Values: \emph{none}.\\\midrule %% TODO: DESCRIBE HOW TO USE!
% UNUSED in 0.15
%flag\_use\_gl\_point\_sprite & bool & (deprecated.)\  \\\midrule
flip\_horz                   & bool & \\\midrule
flip\_vert                   & bool & \\\midrule
viewport\_center\_offset\_x & float & [-0.5\ldots+0.5] Usually 0. \\\midrule
viewport\_center\_offset\_y & float & [-0.5\ldots+0.5] Use negative values to move the horizon lower. \\
\bottomrule
\end{longtabu}

\section{\big[proxy\big]}\label{sec:config.ini:proxy}
%% TODO: What does this do?

\begin{longtabu} to \textwidth {l|l|X}\toprule
\emph{ID}  & \emph{Type} & \emph{Description}\\\midrule
host\_name & string & Name of host for proxy. E.g. \emph{proxy.org}\\\midrule
port       & int    & Port of proxy. E.g. \emph{8080}\\\midrule
user       & string & Username for proxy. E.g. \emph{michael\_knight}\\\midrule
password   & string & Password for proxy. E.g. \emph{xxxxx}\\\bottomrule
\end{longtabu}

\section{\big[scripts\big]}\label{sec:config.ini:scripts}

\begin{longtabu} to \textwidth {l|l|l|X}\toprule
\emph{ID}                  & \emph{Type} & \emph{Default} & \emph{Description}\\\midrule
startup\_script                & string & startup.ssc & name of script executed on program start\\\bottomrule 
%% These are unused as on 0.15pre.
%flag\_script\_allow\_ui        & bool &\\\midrule
%scripting\_allow\_write\_files & bool &\\\bottomrule
\end{longtabu}

\section{\big[search\big]}\label{sec:config.ini:search}

\begin{longtabu} to \textwidth {l|l|X}\toprule
\emph{ID} & \emph{Type} & \emph{Description}\\\midrule
flag\_search\_online & bool   & If \emph{true}, Stellarium will be use SIMBAD for search.\\\midrule
simbad\_server\_url  & string & URL for SIMBAD mirror\\\midrule
flag\_start\_words   & bool   & If \emph{false}, Stellarium will be search phrase only from start of words\\\midrule
coordinate\_system   & string & Specifies the coordinate system. 
                                \emph{Possible values:} equatorialJ2000, equatorial, horizontal, galactic. \emph{Default value:} equatorialJ2000.\\
\bottomrule
\end{longtabu}

\section{\big[spheric\_mirror\big]}\label{sec:config.ini:spheric_mirror}

Stellarium can be used in planetarium domes. You can use a projector with a hemispheric mirror with geometric properties given in this section. 
Note: These functions are only rarely used, some may not work as expected.
%% TODO: Make sure we can remove this note! Maybe add a chapter in the ``Advanced setup'' part?
 
\begin{longtabu} to \textwidth {l|l|l|X}\toprule
\emph{ID} & \emph{Type} & \emph{Default}&\emph{Description}\\\midrule
flip\_horz             & bool  &true & Flip the projection horizontally\\\midrule
flip\_vert             & bool  &false& Flip the projection vertically\\\midrule
projector\_alpha       & float &0& This parameter controls the properties of the spheric mirror projection mode.\\\midrule
projector\_gamma       & float && This parameter controls the properties of the spheric mirror projection mode.\\\midrule
projector\_delta       & float &-1e100& This parameter controls the properties of the spheric mirror projection mode.\\\midrule
projector\_phi         & float &0& This parameter controls the properties of the spheric mirror projection mode.\\\midrule
projector\_position\_x & float &0& \\\midrule
projector\_position\_y & float &1& \\\midrule
projector\_position\_z & float &-0.2& \\\midrule
mirror\_position\_x    & float &0& \\\midrule
mirror\_position\_y    & float &2& \\\midrule
mirror\_position\_z    & float &0& \\\midrule
image\_distance\_div\_height&float&-1e100\\\midrule
mirror\_radius         & float &0.25& \\\midrule
dome\_radius           & float &2.5& \\\midrule
custom\_distortion\_file& string& \\\midrule
texture\_triangle\_base\_length&float& \\\midrule
zenith\_y              & float &0.125& deprecated\\\midrule
scaling\_factor        & float &0.8  & deprecated \\\midrule
%% GZ I found some very strange names in StelViewportEffect.cpp
distorter\_max\_fov & float & 175.0&Set the maximum field of view for the spheric mirror distorter in degrees. Typical value: \emph{180}\\\midrule
%% TODO: MAKE SURE THIS IS STILL USED?
%flag\_use\_ext\_framebuffer\_object & bool & Some video hardware incorrectly claims to support some GL extension, GL\_FRAMEBUFFER\_EXTEXT. If, when using the spheric mirror distorter the
%frame rate drops to a very low value (e.g. 0.1 FPS), set this parameter to \emph{false} to tell Stellarium to ignore the claim of the video driver that it can use this extension\\\midrule
viewportCenterWidth   & float && projected image width, pixels\\\midrule %% TODO Value has buggy name. Test functionality/Discuss!
viewportCenterHeight  & float && projected image height, pixels\\\midrule %% TODO Value has buggy name. Test functionality/Discuss!
viewportCenterX       & float && projected image center X, pixels\\\midrule
viewportCenterY       & float && projected image center Y, pixels\\\midrule
\bottomrule
\end{longtabu}

\section{\big[stars\big]}\label{sec:config.ini:stars}

This section controls how stars are rendered.
%% TODO: Explain what we mean with "magnitude conversion routine"!

\begin{longtabu} to \textwidth {l|l|X}\toprule
\emph{ID}                & \emph{Type} & \emph{Description}\\\midrule
relative\_scale          & float       & relative size of bright and faint stars. Higher values mean that bright stars are comparitively larger when rendered. Typical value: \emph{1.0}\\\midrule
absolute\_scale          & float       & Changes how large stars are rendered. larger value lead to larger depiction. Typical value: \emph{1.0}\\\midrule
star\_twinkle\_amount    & float       & amount of twinkling. Typical value: \emph{0.3}\\\midrule
flag\_star\_twinkle      & bool        & \emph{true} to allow twinkling (but only when atmosphere is active!).\\\midrule
mag\_converter\_max\_fov & float       & maximum field of view for which the magnitude conversion routine is used. Typical value: \emph{90.0}.\\\midrule
mag\_converter\_min\_fov & float       & minimum field of view for which the magnitude conversion routine is used. Typical value: \emph{0.001}.\\\midrule
labels\_amount           & float       & amount of labels. Typical value: \emph{3.0}\\\midrule
init\_bortle\_scale      & int         & initial value of light pollution on the Bortle scale. Typical value: \emph{3}.\\\bottomrule
\end{longtabu}

\section{\big[tui\big]}\label{sec:config.ini:tui}

The built-in text user interface (TUI) plugin (see chapter~\ref{sec:plugins:TextUserInterface}) is most useful for planetariums. You can even configure a system shutdown command. 
For historical reasons, the section is not called \texttt{[TextUserInterface]} but simply \texttt{[tui]}.
%% TODO: Move to TUI plugin chapter?

\begin{longtabu} to \textwidth {l|l|l|X}\toprule
\emph{ID} & \emph{Type} &\emph{Default}& \emph{Description}\\\midrule
%flag\_enable\_tui\_menu & bool & Enables or disables the TUI menu\\\midrule
tui\_font\_size           &float    &15       &Font size for TUI.\\\midrule
tui\_font\_color          &floatRGB &0.3,1,0.3&Font color for TUI.\\\midrule
flag\_show\_gravity\_ui   & bool    &false    & Enables or disables gravity mode for UI\\\midrule
flag\_show\_tui\_datetime & bool    &false    &Set to \emph{true} if you want to see a date and time label suited for dome projections\\\midrule
flag\_show\_tui\_short\_obj\_info   & bool&    &set to \emph{true} if you want to see object info suited for dome projections\\\midrule
% TODO: DESCRIBE!
tui\_admin\_shutdown\_command       & string  && e.g.\ for Linux: ``shutdown --poweroff +2'' \\\bottomrule
\end{longtabu}

\section{\big[video\big]}\label{sec:config.ini:video}

\begin{longtabu} to \textwidth {l|l|X}\toprule
\emph{ID}  & \emph{Type} & \emph{Description}\\\midrule
fullscreen       & bool   & If \emph{true}, Stellarium will start up in full-screen mode, else windowed mode\\\midrule
screen\_w        & int    & Display width when in windowed mode. Value in pixels, e.g. \emph{1024}\\\midrule
screen\_h        & int    & Display height when in windowed mode. Value in pixels, e.g. \emph{768}\\\midrule
screen\_x        & int    & Horizontal position of the top-left corner in windowed mode. Value in pixels, e.g. \emph{0}\\\midrule
screen\_y        & int    & Vertical   position of the top-left corner in windowed mode. Value in pixels, e.g. \emph{0}\\\midrule
viewport\_effect & string & This is used when the spheric mirror display mode is activated. Values include \emph{none} and \emph{sphericMirrorDistorter}.\\\midrule
minimum\_fps     & int    & Sets the minimum number of frames per second to display at (hardware performance permitting)\\\midrule
maximum\_fps     & int    & Sets the maximum number of frames per second to display at. This is useful to reduce power consumption in laptops.\\\bottomrule
\end{longtabu}

\section{\big[viewing\big]}\label{sec:config.ini:viewing}

This section defines which objects, labels, lines, grids etc.\ you
want to see on startup. Set those to \emph{true}. Most items can be
toggled with hotkeys or switched in the GUI.

\begin{longtabu} to \textwidth {l|l|X}
\toprule
\emph{ID} & \emph{Type} & \emph{Description}\\\midrule
flag\_constellation\_drawing    & bool  & Display constellation line drawing\\\midrule
flag\_constellation\_name       & bool  & Display constellation names\\\midrule
flag\_constellation\_art        & bool  & Display constellation art\\\midrule
flag\_constellation\_boundaries & bool  & Display constellation boundaries \\\midrule
flag\_constellation\_isolate\_selected  & bool & If \emph{true}, constellation lines, boundaries and art will be limited to the constellation of the selected star, 
                                                 if that star is ``on'' one of the constellation lines.\\\midrule
flag\_constellation\_pick     & bool & Set to \emph{true} if you only want to see the line drawing, art and name of the selected constellation star\\\midrule
flag\_isolated\_trails        & bool & Set to \emph{true} if you only want to see the trail line drawn for the selected planet (asteroid, comet, moon)\\\midrule
flag\_isolated\_orbits        & bool & Set to \emph{true} if you want to see orbits only for selected planet and their moons.\\\midrule
flag\_azimutal\_grid          & bool & Display azimuthal grid \\\midrule
flag\_equatorial\_grid        & bool & Display equatorial grid (of date) \\\midrule
flag\_equatorial\_J2000\_grid & bool & Display equatorial grid (J2000) \\\midrule
flag\_ecliptic\_grid          & bool & Display ecliptic grid (of date) \\\midrule
flag\_ecliptic\_J2000\_grid   & bool & Display ecliptic grid (J2000) \\\midrule
flag\_galactic\_grid          & bool & Display galactic grid (System~II)\\\midrule
flag\_galactic\_equator\_line & bool & Display galactic equator line \\\midrule
flag\_equator\_line           & bool & Display celestial equator line (of date) \\\midrule
flag\_equator\_J2000\_line    & bool & Display celestial equator line (J2000) \\\midrule
flag\_ecliptic\_line          & bool & Display ecliptic line (of date) \\\midrule
flag\_ecliptic\_J2000\_line   & bool & Display ecliptic line (J2000) \\\midrule
flag\_meridian\_line          & bool & Display meridian line \\\midrule
flag\_prime\_vertical\_line   & bool & Display Prime Vertical line (East-Zenith-West) \\\midrule
flag\_colure\_lines           & bool & Display colure lines (Celestial Pole-\Aries/\Cancer/\Libra/\Capricorn) \\\midrule
flag\_cardinal\_points        & bool & Display cardinal points\\\midrule
% TODO: This should go into the projection group:
flag\_gravity\_labels         & bool & Set to \emph{true} if you want labels to undergo gravity (top side of text points toward zenith). Useful with dome projection.\\\midrule
flag\_moon\_scaled            & bool & Set to \emph{false} if you want to see the real moon size \\\midrule
moon\_scale                   & float & Sets the moon scale factor, sometimes useful to correlate to our perception of the moon's size. Typical value: \emph{4}\\\midrule
constellation\_art\_intensity      & float & brightness [0\ldots1] of the constellation art images. Typical value: \emph{0.5}\\\midrule
constellation\_art\_fade\_duration & float & time the constellation art takes to fade in or out, in seconds. Typical value: \emph{1.5}\\\midrule
constellation\_font\_size          & int   & font size for constellation labels\\\midrule
constellation\_line\_thickness     & float & thickness of lines of the constellations [1\ldots5]. Typical value: \emph{1}\\\midrule
flag\_night                        & bool  & Enable night mode (red-only mode) on startup\\\midrule
light\_pollution\_luminance        & float & Sets the level of the light pollution simulation\\\midrule %% TODO: WHAT IS THIS?
use\_luminance\_adaptation         & bool  & Enable dynamic eye adaptation.\\ %% TODO: DESCRIBE!
\bottomrule
\end{longtabu}





%%% Local Variables: 
%%% mode: latex
%%% TeX-PDF-mode: t
%%% TeX-master: "guide"
%%% End: 

