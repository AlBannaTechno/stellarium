%% Stellarium user guide.
%% State: 2015-12 wiki->Guide. Was called Precision, but this is wrong.
%% 2016-04 GZ Typofixes, made proper chapter from this. 
%% TODO: (GZ) I am not sure about the 1-arcsecond accuracy of VSOP! More should be written with due diligence.
%% 2016-07-28 Extended chapter with more than just VSOP notes.

%\chapterimage{chapter-t1-bg} % Chapter heading image


\chapter{Accuracy}
\label{ch:Accuracy}

Stellarium originally was developed to present a beautiful simulation
of the night sky, mostly to understand what is visible in the sky when
you leave your house, i.e., for present times. To save computation
time, some concessions were made in astronomical accuracy by using
simplified models which seemed acceptable at that time. However, many
users started to overstress Stellarium's capabilities to simulate the
historical sky of many centuries in the past, and found
inconsistencies.  Unfortunately, celestial motions are indeed more
complicated than a simple clockwork, and the process of retrofitting
detailed and accurate models which started around v0.11.0 is not
completed yet. Therefore, when using Stellarium for scientific work
like eclipse simulation to illustrate records found in Cuneiform
tablets, always also use some other reference to compare. You can of
course contact us if you are willing and able to help improving
Stellarium's accuracy!


\section{Planetary Positions}
\label{sec:Accuracy:Planets}

Stellarium uses the VSOP87 \cite{1988A&A...202..309B}
theory\footnote{\url{http://vizier.cfa.harvard.edu/viz-bin/ftp-index?/ftp/cats/VI/81}}
to calculate the positions of the planets over time.

%As with other methods, the accuracy of the calculations vary according
%to the planet and the time for which one makes the calculation. Reasons
%for these inaccuracies include the fact that the motion of the planet
%isn't as predictable as Newtonian mechanics would have us believe.
VSOP87 is an analytical ephemeris modeled to match the numerical
integration run DE200 from NASA JPL. Its use is recommended for the
years -4000\ldots+8000. You can observe the sun leaving the ``ecliptic
of date'' and running on the ``ecliptic J2000'' outside this date
range. This is obviously a mathematical trick to keep
continuity. Still, positions may be somewhat useful outside this
range, but don't expect anything reliable 50,000 years in the past!

The optionally usable JPL DE431 delivers planet positions strictly for
-13000\ldots+17000 only, and nothing outside. Outside of this range,
positions from VSOP87 will be shown again.

As far as Stellarium is concerned, the user should bear in mind the
following properties of the VSOP87 method. Accuracy values here are
positional as observed from Earth.

\newpage
\begin{longtabu} to \textwidth {X|l|X}
\toprule
\emph{Object(s)} & \emph{Method} & \emph{Notes}\tabularnewline
\midrule
Mercury, Venus, Earth-Moon barycenter, Mars & VSOP87 & Accuracy is 1 arc-second from 2000 B.C. -- 6000 A.D. \\\midrule
Jupiter, Saturn                             & VSOP87 & Accuracy is 1 arc-second from 0 A.D. -- 4000 A.D.    \\\midrule
Uranus, Neptune                             & VSOP87 & Accuracy is 1 arc-second from 4000 B.C. -- 8000 A.D. \\\midrule
Pluto                                       & ?      & Pluto's position is valid only from 1885 A.D. -- 2099 A.D.\\\midrule
Earth's Moon                                & ELP2000-82B & Unsure about interval of validity or accuracy at time of writing. Possibly valid from 1828 A.D. to 2047 A.D.\\\midrule
Galilean satellites                         & L2     & Valid from 500 A.D. -- 3500 A.D.\\
\bottomrule
\end{longtabu}


\section{Minor Bodies}
\label{sec:Accuracy:MinorBodies}

Positions for the Minor Bodies (Dwarf Planets, Asteroids, Comets) are
computed with standard algorithms found in astronomical text
books. The generally used method of orbital elements allows to compute
the positions of the respective object on an undisturbed Kepler orbit
around the sun. However, gravitational, and in the case of comets,
non-gravitational (outgassing) disturbances slowly change these
orbital elements. Therefore an \emph{epoch} is given for such
elements, and computation of positions for times far from this epoch
will lead to positional errors. Therefore, when searching for
asteroids or comets, always update your orbital elements, and use
elements with an epoch as close to your time of observation as
possible! Stellarium does not simulate gravitational perturbances and
orbital changes of minor bodies passing major planets.

\section{Precession and Nutation}
\label{sec:Accuracy:Precession}

Since v0.14.0, Stellarium computes the orientation of earth's axis
according to the IAU2006 Precession in a long-time variant developed
by Vondr\'ak et al.~\cite{2011AA:Vondrak} and IAU2000B Nutation. This
also now allows proper depiction of the changes in ecliptic obliquity
and display of ``instantaneous precession circles'' around the
ecliptic poles. These circles are indeed varying according to
ecliptical obliquity. Nutation is only computed for about 500 years
around J2000.0. Nobody could have observed it before 1609, and it is
unclear for how long the model is applicable.

\section{Planet Axes}
\label{sec:Accuracy:PlanetAxes}

Orientation for the other planets is still simplified. Future versions
should implement modern IAU guidelines.


%%% Local Variables: 
%%% mode: latex
%%% TeX-PDF-mode: t
%%% TeX-master: "guide"
%%% End: 

