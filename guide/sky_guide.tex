%% Stellarium User Guide
% Status:
% 2016-01-03 Wiki-->LaTeX
% 2016-04-04 GZ adapted wording to new layout, updated some information, fixed spelling. Ready for book, but can be enhanced with pictures, or more objects. (Gems of the Southern Sky?)

\chapter{A Little Sky Guide}
\label{ch:SkyGuide}

This chapter lists some astronomical objects that can be located using
Stellarium. All of them can be seen with the naked eye or binoculars.
Since many astronomical objects have more than one name (often having a
'proper name', a 'common name' and various catalogue numbers), the chapter
lists the name as it appears in Stellarium --- use this name when using
Stellarium's search function --- and any other commonly used names.

The Location Guide entry gives brief instructions for finding each
object using nearby bright stars or groups of stars when looking at the
real sky --- a little time spent learning the major constellations
visible from your latitude will pay dividends when it comes to locating
fainter (and more interesting!) objects. When trying to locate these
objects in the night sky, keep in mind that Stellarium displays many
stars that are too faint to be visible without optical aid, and even
bright stars can be dimmed by poor atmospheric conditions and light
pollution.

\section{Dubhe and Merak, The Pointers}
\textbf{Type:} Stars \\
\textbf{Magnitude:} 1.83, 2.36 \\
\textbf{Location Guide:} \textit{The two 'rightmost' of the seven stars that form the main shape of 'The Plough' (or 'Big Dipper', part of Ursa Major).} 

Northern hemisphere observers are very fortunate to have two stars
that point towards Polaris which lie very close to the northern
celestial pole. Whatever the time of night or season of the year they
are always an immediate clue to the location of Polaris, the Pole
Star.

\section{M31, Messier 31, The Andromeda Galaxy}
\textbf{Type:} Spiral Galaxy \\
\textbf{Magnitude:} 3.4 \\ 
\textbf{Location Guide:} \textit{Find the three bright stars that constitute the main part of the constellation of Andromeda. From the middle of these look toward the constellation of Cassiopeia.}

M31 is the most distant object visible to the naked eye, and among the
few nebulae that can be seen without a telescope or powerful
binoculars. Under good conditions it appears as a large fuzzy patch of
light. It is a galaxy containing billions of stars whose distance is
roughly 2.5 million light years from Earth.

\section{The Garnet Star, $\mu$ Cephei}
\textbf{Type:} Variable Star \\
\textbf{Magnitude:} 4.25 (Avg.) \\
\textbf{Location Guide:} \textit{Cephius lies 'above' the W-shape of Cassiopeia. The Garnet Star lies slightly to one side of a point halfway between 5 Cephei and 21 Cephei.}

A 'supergiant' of spectral class M with a strong red colour. Given its
name by Sir William Herschel in the 18th century, the colour is
striking in comparison to its blue-white neighbours.

\section{4 and 5 Lyrae, $\epsilon$ Lyrae}
\textbf{Type:} Double Star \\
\textbf{Magnitude:} 4.7 \\
\textbf{Location Guide:} \textit{Close to Vega ($\alpha$ Lyrae), one of the brightest stars in the sky.}

In binoculars $\epsilon$ Lyrae is resolved into two separate
stars. Remarkably, each of these is also a double star (although this
will only be seen with a telescope) and all four stars form a physical
system.

\section{M13, Hercules Cluster} 
\textbf{Type:} Globular Cluster \\ 
\textbf{Magnitude:} 5.8 \\
\textbf{Location Guide:} \textit{Located approximately 1/3 of the way along a line from 44 ($\eta$) to 40 ($\zeta$) Herculis.}

This cluster of hundreds of thousands of mature stars appears as
a circular 'cloud' using the naked eye or binoculars (a large
telescope is required to resolve individual stars). Oddly the cluster
appears to contain one young star and several areas that are almost
devoid of stars.

\section{M45, The Pleiades, The Seven Sisters}
\textbf{Type:} Open Cluster \\
\textbf{Magnitude:} 1.2 (Avg.) \\
\textbf{Location Guide:} \textit{Lies on the Bull's back, about 1/3 between Aldebaran in Taurus and Almaak in Andromeda.} 

Depending upon conditions, six to 9 of the blueish stars in this
famous cluster will be visible to someone with average eyesight, and in
binoculars it is a glorious sight. The cluster has more than 500
members in total, many of which are shown to be surrounded by nebulous
material in long exposure photographs.

\section{Algol, The Demon Star, $\beta$ Persei}
\textbf{Type:} Variable Star \\
\textbf{Magnitude:} 3.0 (Avg.) \\
\textbf{Location Guide:} \textit{Halfway between Aldebaran in Taurus and the middle star of the 'W' of Cassiopeia.}

Once every three days or so, Algol's brightness changes from 2.1 to 3.4
and back within a matter of hours. The reason for this change is that
Algol has a dimmer giant companion star, with an orbital period of
about 2.8 days, that causes a regular partial eclipse. Although
Algol's fluctuations in magnitude have been known since at least the
17th century, it was the first to be proved to be due to an eclipsing
companion --- it is therefore the prototype Eclipsing Variable.

\section{Sirius, $\alpha$ Canis Majoris}
\textbf{Type:} Star \\
\textbf{Magnitude:} -1.47 \\
\textbf{Location Guide:} \textit{Sirius is easily found by following the line of three stars in Orion's belt southwards.} 

Sirius is a white dwarf star at a comparatively close 8.6 light
years. This proximity and its high innate luminance makes it the
brightest star in our sky. Sirius is a double star; its companion is a White Dwarf, 
much dimmer but very hot, and is believed to be smaller than the earth.

\section{M44, The Beehive, Praesepe}
\textbf{Type:} Open Cluster \\
\textbf{Magnitude:} 3.7 \\ 
\textbf{Location Guide:} \textit{Cancer lies about halfway between the twins (Castor \& Pollux) in Gemini and Regulus, the brightest star in Leo. The Beehive can be found between Asellus Borealis and Asellus Australis.} 

There are probably 350 or so stars in this cluster, although it appears
to the naked eye simply as a misty patch. It contains a mixture of
stars from red giants to white dwarf and is estimated to be some 700
million years old.

\section{27 Cephei, $\delta$ Cephei} 
\textbf{Type:} Variable Star \\
\textbf{Magnitude:} 4.0 (Avg.) \\ 
\textbf{Location Guide:} \textit{Locate the four stars that form the square of Cepheus. One corner of the square has two other bright stars nearby forming a distinctive triangle --- $\delta$ is at the head of this triangle in the direction of Cassiopeia.} 

$\delta$ Cephei gives its name to a whole class of variables, all of
which are pulsating high-mass stars in the later stages of their
evolution. $\delta$ Cephei is also a double star with a companion of
magnitude 6.3 visible in binoculars.

\section{M42, The Great Orion Nebula} 
\textbf{Type:} Nebula \\
\textbf{Magnitude:} 4 \\
\textbf{Location Guide:} \textit{Almost in the middle of the area bounded by Orion's belt and lower stars, Saiph and Rigel.} 

The Great Orion Nebula is the brightest nebula visible in the night
sky and lies at about 1.500 light years from earth. It is a truly
gigantic gas and dust cloud that extends for several hundred light
years, reaching almost halfway across the constellation of Orion. The
nebula contains a cluster of hot young stars known as the Trapezium,
and more stars are believed to be forming within the cloud.

\section{La Superba, Y Canum Venaticorum, HIP 62223}
\textbf{Type:} Star \\
\textbf{Magnitude:} 5.4 (Avg.) \\
\textbf{Location Guide:} \textit{Almost the center of the arch of stars of Ursa Major's tail. Forms a neat triangle with Phekda ($\gamma$) and Alkaid ($\eta$, tail tip) in Ursa Major towards Canes Venatici.} 

La Superba is a 'Carbon Star' --- a group of relatively cool gigantic
(usually variable) stars that have an outer shell containing high
levels of carbon. This shell is very efficient at absorbing short
wavelength blue light, giving carbon stars a distinctive red or orange
tint.

\section{52 and 53 Bootis, $\nu^1$ and $\nu^2$ Bootis} 
\textbf{Type:} Double Star \\
\textbf{Magnitude:} 5.02, 5.02 \\
\textbf{Location Guide:} \textit{Follow a line from Seginus ($\gamma$ Boo, left shoulder) to Nekkar ($\beta$ Boo, the head) and then continue for the same distance again to arrive at this double star.} 

This optical double star consists of a pair of different spectral type, and 52 Bootis, at
approximately 800 light years, is twice as far away as 53.


%%% Local Variables: 
%%% mode: latex
%%% TeX-master: "guide"
%%% End: 
