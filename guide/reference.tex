\chapterimage{chapter-t1-bg} % Chapter heading image

\chapter{Stellarium Reference}\label{stellarium-reference}

\section{Configuration File}\label{configuration-file}
See \href{Advanced_Use\#The_Main_Configuration_File}{Advanced Use, The
Main Configuration File} for information about this file, including its
default installed location, name, and command line options that can
effect how it is processed.

Deprecate parameters marked by gray background. Possible new parameters
marked by yellow background.

\subsection{Section \big[astro\big]}\label{section-astro}

\begin{longtabu} to \textwidth {l|l|X}
\toprule
\emph{ID} & \emph{Type} & \emph{Description}\tabularnewline
\midrule
apparent\_magnitude\_algorithm & string & Set
algorithm for computation of apparent magnitude of the planets. Possible
values: \emph{Planesas}, \emph{Mueller}, \emph{Harris} and
\emph{Generic}\footnote{Available since version 0.13.3}. Default value: \emph{Harris}.\tabularnewline
\midrule
nebula\_magnitude\_limit & float & Value of limit
magnitude of the deep-sky objects. Default value: \emph{8.5}.\tabularnewline
star\_magnitude\_limit & float & Value of limit magnitude of
the stars\tabularnewline
\midrule
flag\_nebula\_magnitude\_limit & bool & Set to
\emph{true} to set limits for showing deep-sky objects. Default value: \emph{false}.\tabularnewline
\midrule
flag\_star\_magnitude\_limit & bool & Set to \emph{true} to
set limits for showing stars\tabularnewline
\midrule
flag\_extinction\_below\_horizon & bool & Set to \emph{true}
to apply extinction effects to sky below horizon\tabularnewline
\midrule
extinction\_mode\_below\_horizon & string & Set extinction
mode for atmosphere below horizon. Possible values: \emph{zero},
\emph{mirror} and \emph{max}. Default value is
\emph{zero}.\tabularnewline
\midrule
flag\_stars & bool & Set to \emph{false} to hide the
stars on start-up\tabularnewline
\midrule
flag\_star\_name & bool & Set to \emph{false} to hide the
star labels on start-up\tabularnewline
\midrule
flag\_planets & bool & Set to \emph{false} to hide the
planet labels on start-up\tabularnewline
\midrule
flag\_planets\_hints & bool & Set to \emph{false} to hide
the planet hints on start-up (names and circular
highlights)\tabularnewline
\midrule
flag\_planets\_orbits & bool & Set to \emph{true} to show
the planet orbits on start-up\tabularnewline
\midrule
flag\_permanent\_orbits & bool & Set to \emph{true} to show
the orbit of planet, when planet is out of the viewport
also.\tabularnewline
\midrule
flag\_planets\_markers & bool & Set to \emph{true} to show
the planet pointer markers on start-up\tabularnewline
\midrule
flag\_light\_travel\_time & bool & Set to \emph{true} to
improve accuracy in the movement of the planets by compensating for the
time it takes for light to travel. This has an impact on
performance.\tabularnewline
\midrule
flag\_object\_trails & bool & Turns on and off drawing of
object trails (which show the movement of the planets over
time)\tabularnewline
\midrule
flag\_nebula & bool & Set to \emph{false} to
hide the nebulae on start-up. Default value: \emph{true}.\tabularnewline
\midrule
flag\_nebula\_name & bool & Set to
\emph{true} to show the nebula labels on start-up. Default value: \emph{false}.\tabularnewline
\midrule
flag\_nebula\_long\_name & bool & Set to \emph{true} to show
the nebula long labels on start-up.\tabularnewline
\midrule
flag\_nebula\_display\_no\_texture & bool & Set
to \emph{true} to suppress displaying of nebula textures. Default value: \emph{false}.\tabularnewline
\midrule
nebula\_hints\_amount & float & Sets the amount of
hints (between 0 and 10). Default value: \emph{3.0}.\tabularnewline
\midrule
nebula\_labels\_amount & float & Sets the amount of
labels (between 0 and 10). Default value: \emph{3.0}.\tabularnewline
\midrule
flag\_milky\_way & bool & Set to \emph{false}
to hide the Milky Way. Default value: \emph{true}.\tabularnewline
\midrule
milky\_way\_intensity & float & Sets the relative
brightness with which the milky way is drawn. Typical value: \emph{1} to
\emph{10}. Default value: \emph{1.0}.\tabularnewline
\midrule
flag\_zodiacal\_light & bool & Sets to \emph{false} to hide
the zodiacal light\tabularnewline
\midrule
zodiacal\_light\_intensity & float & Sets the relative
brightness with which the zodiacal light is drawn. Typical value:
\emph{1} to \emph{10}.\tabularnewline
\midrule
max\_mag\_nebula\_name & float & Sets the magnitude of
the nebulae whose name is shown. Typical value: \emph{8}\tabularnewline
\midrule
nebula\_scale & float & Sets how much to scale nebulae. a
setting of \emph{1} will display nebulae at normal size. Default value: \emph{1.0}.\tabularnewline
\midrule
flag\_bright\_nebulae & bool & Set to \emph{true} to increase
nebulae brightness to enhance viewing (less realistic)\tabularnewline
\midrule
flag\_nebula\_ngc & bool & Enables/disables display of all
NGC objects\tabularnewline
\midrule
flag\_nebula\_hints\_proportional & bool &
Enables/disables proportional markers for deep-sky
objects. Default value: \emph{false}.\tabularnewline
\midrule
flag\_surface\_brightness\_usage & bool &
Enables/disables usage surface brightness for markers for deep-sky
objects. Default value: \emph{false}.\tabularnewline
\midrule
flag\_use\_type\_filter & bool &
Enables/disables usage of the type filters for deep-sky
objects. Default value: \emph{false}.\tabularnewline
\bottomrule
\end{longtabu}

\subsection{Section \big[color\big]}\label{section-color}

\begin{longtabu} to \textwidth {l|l|X}
\toprule
\emph{ID} & \emph{Type} & \emph{Description}\tabularnewline
\midrule
default\_color & float R,G,B & Sets the default
colour in RGB values, where 1 is the maximum, e.g. \emph{1.0,1.0,1.0}
for white. Default value: \emph{0.5,0.5,0.7}. \tabularnewline
\midrule
azimuthal\_color & float R,G,B & Sets the
colour of the azimuthal grid in RGB values, where 1 is the maximum, e.g.
\emph{1.0,1.0,1.0} for white. Default value: \emph{0.3,0.2,0.1}.\tabularnewline
\midrule
equatorial\_color & float R,G,B & Sets the
colour of the equatorial grid (on date) in RGB values, where 1 is the
maximum, e.g. \emph{1.0,1.0,1.0} for white. Default value: \emph{0.2,0.3,0.8}.\tabularnewline
\midrule
equatorial\_J2000\_color & float R,G,B & Sets
the colour of the equatorial grid (J2000) in RGB values, where 1 is the
maximum, e.g. \emph{1.0,1.0,1.0} for white. Default value: \emph{0.1,0.1,0.5 }.\tabularnewline
\midrule
galactic\_color & float R,G,B & Sets the colour
of the galactic grid in RGB values, where 1 is the maximum, e.g.
\emph{1.0,1.0,1.0} for white. Default value: \emph{0.0,0.3,0.2}.\tabularnewline
\midrule
galactic\_equator\_color & float R,G,B & Sets the
colour of the galactic equator line in RGB values, where 1 is the
maximum, e.g. \emph{1.0,1.0,1.0} for white. Default value: \emph{0.5,0.3,0.1}. \tabularnewline
\midrule
equator\_color & float R,G,B & Sets the
colour of the equatorial line in RGB values, where 1 is the maximum,
e.g. \emph{1.0,1.0,1.0} for white. Default value: \emph{0.3,0.5,1.0}. \tabularnewline
\midrule
ecliptic\_color & float R,G,B & Sets the
colour of the ecliptic line (J2000) in RGB values, where 1 is the
maximum, e.g. \emph{1.0,1.0,1.0} for white. Default value: \emph{0.9,0.6,0.2}. \tabularnewline
\midrule
ecliptic\_J2000\_color & float R,G,B & Sets the
colour of the ecliptic grid (J2000) in RGB values, where 1 is the
maximum, e.g. \emph{1.0,1.0,1.0} for white. Default value: \emph{0.7,0.2,0.2}. \tabularnewline
\midrule
meridian\_color & float R,G,B & Sets the
colour of the meridian line in RGB values, where 1 is the maximum, e.g.
\emph{1.0,1.0,1.0} for white. Default value: \emph{0.2,0.6,0.2}. \tabularnewline
\midrule
horizon\_color & float R,G,B & Sets the colour
of the horizon line in RGB values, where 1 is the maximum, e.g.
\emph{1.0,1.0,1.0} for white. Default value: \emph{0.2,0.6,0.2}. \tabularnewline
\midrule
const\_lines\_color & float R,G,B & Sets the
colour of the constellation lines in RGB values, where 1 is the maximum,
e.g. \emph{1.0,1.0,1.0} for white. Default value: \emph{0.2,0.2,0.6}. \tabularnewline
\midrule
const\_names\_color & float R,G,B & Sets the
colour of the constellation names in RGB values, where 1 is the maximum,
e.g. \emph{1.0,1.0,1.0} for white. Default value: \emph{0.4,0.6,0.9}. \tabularnewline
\midrule
const\_boundary\_color & float R,G,B & Sets
the colour of the constellation boundaries in RGB values, where 1 is the
maximum, e.g. \emph{1.0,1.0,1.0} for white. Default value: \emph{0.3,0.1,0.1}. \tabularnewline
\midrule
star\_label\_color & float R,G,B & Sets the
colour of the star labels in RGB values, where 1 is the maximum, e.g.
\emph{1.0,1.0,1.0} for white. Default value: \emph{0.4,0.3,0.5}. \tabularnewline
\midrule
cardinal\_color & float R,G,B & Sets the
colour of the cardinal points in RGB values, where 1 is the maximum,
e.g. \emph{1.0,1.0,1.0} for white. Default value: \emph{0.8,0.2,0.1}. \tabularnewline
\midrule
planet\_names\_color & float R,G,B & Sets the
colour of the planet names in RGB values, where 1 is the maximum, e.g.
\emph{1.0,1.0,1.0} for white. Default value: \emph{0.5,0.5,0.7}. \tabularnewline
\midrule
planet\_orbits\_color & float R,G,B & Sets
the colour of the orbits in RGB values, where 1 is the maximum, e.g.
\emph{1.0,1.0,1.0} for white. Default value: \emph{0.7,0.2,0.2}. \tabularnewline
\midrule
planet\_pointers\_color & float R,G,B & Sets the
colour of the planet pointers in RGB values, where 1 is the maximum,
e.g. 1.0,1.0,1.0 for white. Default value: \emph{1.0,0.3,0.3}. \tabularnewline
\midrule
object\_trails\_color & float R,G,B & Sets
the colour of the planet trails in RGB values, where 1 is the maximum,
e.g. \emph{1.0,1.0,1.0} for white. Default value: \emph{1.0,0.7,0.0}. \tabularnewline
\midrule
telescope\_circle\_color & float R,G,B & Sets
the colour of the telescope location indicator. RGB values, where 1 is
the maximum, e.g. \emph{1.0,1.0,1.0} for white. Default value: \emph{0.6,0.4,0.0}. \tabularnewline
\midrule
telescope\_label\_color & float R,G,B & Sets
the colour of the telescope location label. RGB values, where 1 is the
maximum, e.g. \emph{1.0,1.0,1.0} for white. Default value: \emph{0.6,0.4,0.0}. \tabularnewline
\midrule
script\_console\_keyword\_color & float R,G,B & Sets the colour of the keywords in the script console. RGB values, where 1 is the maximum, e.g. \emph{1.0,1.0,1.0} for white. Default value: \emph{1.0,0.0,1.0}. \tabularnewline
\midrule
script\_console\_module\_color & float R,G,B & Sets the colour of the modules in the script console. RGB values, where 1 is the maximum, e.g. \emph{1.0,1.0,1.0} for white. Default value: \emph{0.0,1.0,1.0}. \tabularnewline
\midrule
script\_console\_comment\_color & float R,G,B & Sets the colour of the comments in the script console. RGB values, where 1 is the maximum, e.g. \emph{1.0,1.0,1.0} for white. Default value: \emph{1.0,1.0,0.0}. \tabularnewline
\midrule
script\_console\_function\_color & float R,G,B & Sets the colour of the functions in the script console. RGB values, where 1 is the maximum, e.g. \emph{1.0,1.0,1.0} for white. Default value: \emph{0.0,1.0,0.0}. \tabularnewline
\midrule
script\_console\_constant\_color & float R,G,B & Sets the colour of the constants in the script console. RGB values, where 1 is the maximum, e.g. \emph{1.0,1.0,1.0} for white. Default value: \emph{1.0,0.5,0.5}. \tabularnewline
\midrule
daylight\_text\_color & float R,G,B & Sets the colour of the info text at daylight. RGB values, where 1 is the maximum, e.g. \emph{1.0,1.0,1.0} for white. Default value: \emph{0.0,0.0,0.0}. \tabularnewline
\midrule
dso\_label\_color & float R,G,B & Sets the colour of the deep-sky objects labels in RGB values, where 1 is the maximum, e.g. \emph{1.0,1.0,1.0} for white. Default value: \emph{0.2,0.6,0.7}. \tabularnewline
\midrule
dso\_circle\_color & float R,G,B & Sets the colour of the deep-sky objects symbols in RGB values, where 1 is the maximum, e.g. \emph{1.0,1.0,1.0} for white. Default value: \emph{1.0,0.7,0.2}. \tabularnewline
\midrule
dso\_galaxy\_color & float R,G,B & Sets the colour of the galaxies symbols in RGB values, where 1 is the maximum, e.g. \emph{1.0,1.0,1.0} for white. Default value: \emph{1.0,0.2,0.2}. \tabularnewline
\midrule
dso\_radio\_galaxy\_color & float R,G,B & Sets the colour of the radio galaxies symbols in RGB values, where 1 is the maximum, e.g. \emph{1.0,1.0,1.0} for white. Default value: \emph{0.3,0.3,0.3}. \tabularnewline
\midrule
dso\_active\_galaxy\_color & float R,G,B & Sets the colour of the active galaxies symbols in RGB values, where 1 is the maximum, e.g. \emph{1.0,1.0,1.0} for white. Default value: \emph{1.0,0.5,0.2}. \tabularnewline
\midrule
dso\_interacting\_galaxy\_color & float R,G,B & Sets the colour of the interacting galaxies symbols in RGB values, where 1 is the maximum, e.g. \emph{1.0,1.0,1.0} for white. Default value: \emph{0.2,0.5,1.0}. \tabularnewline
\midrule
dso\_quasar\_color & float R,G,B & Sets the colour of the quasars symbols in RGB values, where 1 is the maximum, e.g. \emph{1.0,1.0,1.0} for white. Default value: \emph{1.0,0.2,0.2}. \tabularnewline
\midrule
dso\_possible\_quasar\_color & float R,G,B & Sets the colour of the possible quasars symbols in RGB values, where 1 is the maximum, e.g. \emph{1.0,1.0,1.0} for white. Default value: \emph{1.0,0.2,0.2}. \tabularnewline
\midrule
dso\_bl\_lac\_color & float R,G,B & Sets the colour of the BL Lac objects symbols in RGB values, where 1 is the maximum, e.g. \emph{1.0,1.0,1.0} for white. Default value: \emph{1.0,0.2,0.2}. \tabularnewline
\midrule
dso\_blazar\_color & float R,G,B & Sets the colour of the blazars symbols in RGB values, where 1 is the maximum, e.g. \emph{1.0,1.0,1.0} for white. Default value: \emph{1.0,0.2,0.2}. \tabularnewline
\midrule
dso\_nebula\_color & float R,G,B & Sets the colour of the nebulae symbols in RGB values, where 1 is the maximum, e.g. \emph{1.0,1.0,1.0} for white. Default value: \emph{0.1,1.0,0.1}. \tabularnewline
\midrule
dso\_planetary\_nebula\_color & float R,G,B & Sets the colour of the planetary nebulae symbols in RGB values, where 1 is the maximum, e.g. \emph{1.0,1.0,1.0} for white. Default value: \emph{0.1,1.0,0.1}. \tabularnewline
\midrule
dso\_reflection\_nebula\_color & float R,G,B & Sets the colour of the reflection nebulae symbols in RGB values, where 1 is the maximum, e.g. \emph{1.0,1.0,1.0} for white. Default value: \emph{0.1,1.0,0.1}. \tabularnewline
\midrule
dso\_bipolar\_nebula\_color & float R,G,B & Sets the colour of the bipolar nebulae symbols in RGB values, where 1 is the maximum, e.g. \emph{1.0,1.0,1.0} for white. Default value: \emph{0.1,1.0,0.1}. \tabularnewline
\midrule
dso\_emission\_nebula\_color & float R,G,B & Sets the colour of the emission nebulae symbols in RGB values, where 1 is the maximum, e.g. \emph{1.0,1.0,1.0} for white. Default value: \emph{0.1,1.0,0.1}. \tabularnewline
\midrule
dso\_dark\_nebula\_color & float R,G,B & Sets the colour of the dark nebulae symbols in RGB values, where 1 is the maximum, e.g. \emph{1.0,1.0,1.0} for white. Default value: \emph{0.3,0.3,0.3}. \tabularnewline
\midrule
dso\_hydrogen\_region\_color & float R,G,B & Sets the colour of the hydrogen regions symbols in RGB values, where 1 is the maximum, e.g. \emph{1.0,1.0,1.0} for white. Default value: \emph{0.1,1.0,0.1}. \tabularnewline
\midrule
dso\_supernova\_remnant\_color & float R,G,B & Sets the colour of the supernovae remnants symbols in RGB values, where 1 is the maximum, e.g. \emph{1.0,1.0,1.0} for white. Default value: \emph{0.1,1.0,0.1}. \tabularnewline
\midrule
dso\_interstellar\_matter\_color & float R,G,B & Sets the colour of the interstellar matter symbols in RGB values, where 1 is the maximum, e.g. \emph{1.0,1.0,1.0} for white. Default value: \emph{0.1,1.0,0.1}. \tabularnewline
\midrule
dso\_cluster\_with\_nebulosity\_color & float R,G,B & Sets the colour of the clusters associated with nebulosity
symbols in RGB values, where 1 is the maximum, e.g. \emph{1.0,1.0,1.0}
for white. Default value: \emph{0.1,1.0,0.1}. \tabularnewline
\midrule
dso\_molecular\_cloud\_color & float R,G,B & Sets the colour of the molecular clouds symbols in RGB values, where 1 is the maximum, e.g. \emph{1.0,1.0,1.0} for white. Default value: \emph{0.1,1.0,0.1}. \tabularnewline
\midrule
dso\_possible\_planetary\_nebula\_color & float R,G,B & Sets the colour of the possible planetary nebulae symbols in RGB values, where 1 is the maximum, e.g. \emph{1.0,1.0,1.0} for white. Default value: \emph{0.1,1.0,0.1 }. \tabularnewline
\midrule
dso\_protoplanetary\_nebula\_color & float R,G,B & Sets the colour of the protoplanetary nebulae symbols in RGB values, where 1 is the maximum, e.g. \emph{1.0,1.0,1.0} for white. Default value: \emph{0.1,1.0,0.1}. \tabularnewline
\midrule
dso\_cluster\_color & float R,G,B & Sets the colour of the star clusters symbols in RGB values, where 1 is the maximum, e.g. \emph{1.0,1.0,1.0} for white. Default value: \emph{1.0,1.0,0.1}. \tabularnewline
\midrule
dso\_open\_cluster\_color & float R,G,B & Sets the colour of the open star clusters symbols in RGB values, where 1 is the maximum, e.g. \emph{1.0,1.0,1.0} for white. Default value: \emph{1.0,1.0,0.1}. \tabularnewline
\midrule
dso\_globular\_cluster\_color & float R,G,B & Sets the colour of the globular star clusters symbols in RGB values, where 1 is the maximum, e.g. \emph{1.0,1.0,1.0} for white. Default value: \emph{1.0,1.0,0.1}. \tabularnewline
\midrule
dso\_stellar\_association\_color & float R,G,B & Sets the colour of the stellar associations symbols in RGB values, where 1 is the maximum, e.g. \emph{1.0,1.0,1.0} for white. Default value: \emph{1.0,1.0,0.1}. \tabularnewline
\midrule
dso\_star\_cloud\_color & float R,G,B & Sets the colour of the star clouds symbols in RGB values, where 1 is the maximum, e.g. \emph{1.0,1.0,1.0} for white. Default value: \emph{1.0,1.0,0.1}. \tabularnewline
\midrule
dso\_star\_color & float R,G,B & Sets the colour of the star symbols in RGB values, where 1 is the maximum, e.g. \emph{1.0,1.0,1.0} for white. Default value: \emph{1.0,0.7,0.2}. \tabularnewline
\midrule
dso\_emission\_object\_color & float R,G,B & Sets the colour of the emission objects symbols in RGB values, where 1 is the maximum, e.g. \emph{1.0,1.0,1.0} for white. Default value: \emph{1.0,0.7,0.2}. \tabularnewline
\midrule
dso\_young\_stellar\_object\_color & float R,G,B & Sets the colour of the young stellarium objects symbols in RGB values, where 1 is the maximum, e.g. \emph{1.0,1.0,1.0} for white. Default value: \emph{1.0,0.7,0.2}. \tabularnewline
\bottomrule
\end{longtabu}

\subsection{Section \big[custom\_selected\_info\big]}\label{section-custom-selected-info}

\begin{longtabu} to \textwidth {l|l|X}
\toprule
\emph{ID} & \emph{Type} & \emph{Description}\tabularnewline
\midrule
flag\_show\_absolutemagnitude & bool & If \emph{true},
Stellarium will be show absolute magnitude for objects.\tabularnewline
\midrule
flag\_show\_altaz & bool & If \emph{true}, Stellarium will be
show horizontal coordinates for objects.\tabularnewline
\midrule
flag\_show\_catalognumber & bool & If \emph{true}, Stellarium
will be show catalog designations for objects.\tabularnewline
\midrule
flag\_show\_distance & bool & If \emph{true}, Stellarium will
be show distance to object.\tabularnewline
\midrule
flag\_show\_extra & bool & If \emph{true}, Stellarium will be
show extra info for object.\tabularnewline
\midrule
flag\_show\_hourangle & bool & If \emph{true}, Stellarium will
be show hour angles for object.\tabularnewline
\midrule
flag\_show\_magnitude & bool & If \emph{true}, Stellarium will
be show magnitude for object.\tabularnewline
\midrule
flag\_show\_name & bool & If \emph{true}, Stellarium will be
show common name for object.\tabularnewline
\midrule
flag\_show\_radecj2000 & bool & If \emph{true}, Stellarium
will be show geocentrical equatorial coordinates (J2000) of
object.\tabularnewline
\midrule
flag\_show\_radecofdate & bool & If \emph{true}, Stellarium
will be show geocentrical equatorial coordinates (of date) of
object.\tabularnewline
\midrule
flag\_show\_size & bool & If \emph{true}, Stellarium will be
show size of object.\tabularnewline
\midrule
flag\_show\_galcoord & bool & If \emph{true}, Stellarium will
be show galactic coordinates of object.\tabularnewline
\midrule
flag\_show\_eclcoord & bool & If \emph{true}, Stellarium will
be show ecliptic coordinates (J2000) of object.\tabularnewline
\midrule
flag\_show\_type & bool & If \emph{true}, Stellarium will be
show type of object.\tabularnewline
\bottomrule
\end{longtabu}

\subsection{Section \big[custom\_time\_correction\big]}\label{section-custom-time-correction}

\begin{longtabu} to \textwidth {l|l|X}
\toprule
\emph{ID} & \emph{Type} & \emph{Description}\tabularnewline
\midrule
coefficients & {[}float,float,float{]} & Coefficients for
custom equation of DeltaT\tabularnewline
\midrule
ndot & float & n-dot value for custom equation of
DeltaT\tabularnewline
\midrule
year & int & Year for custom equation of DeltaT\tabularnewline
\bottomrule
\end{longtabu}

\subsection{Section \big[devel\big]}\label{section-devel}

\begin{longtabu} to \textwidth {l|l|X}
\toprule
\emph{ID} & \emph{Type} & \emph{Description}\tabularnewline
\midrule
convert\_dso\_catalog & bool & Set to
\emph{true} to convert \texttt{catalog.txt} file into
\texttt{catalog.dat} file. Default value: \emph{false}.\tabularnewline
\midrule
convert\_dso\_decimal\_coord & bool & Set to
\emph{true} to use decimal values for coordinates in source
catalog. Default value: \emph{true}.\tabularnewline
\bottomrule
\end{longtabu}

\subsection{Section
\big[dso\_catalog\_filters\big]}\label{section-dso-catalog-filters}

\begin{longtabu} to \textwidth {l|l|X}
\toprule
\emph{ID} & \emph{Type} & \emph{Description}\tabularnewline
\midrule
flag\_show\_ngc & bool & Set to \emph{true} to displaying New General Catalogue (NGC). Default value: \emph{true}. \tabularnewline
\midrule
flag\_show\_ic & bool & Set to \emph{true} to
displaying Index Catalogue (IC). Default value: \emph{true}. \tabularnewline
\midrule
flag\_show\_m & bool & Set to \emph{true} to displaying Messier Catalog (M). Default value: \emph{true}. \tabularnewline
\midrule
flag\_show\_c & bool & Set to \emph{true} to displaying Caldwell Catalogue (C). Default value: \emph{false}. \tabularnewline
\midrule
flag\_show\_b & bool & Set to \emph{true} to displaying Barnard Catalogue (B). Default value: \emph{false}. \tabularnewline
\midrule
flag\_show\_sh2 & bool & Set to \emph{true} to displaying Sharpless Catalogue (Sh2). Default value: \emph{false}. \tabularnewline
\midrule
flag\_show\_vdb & bool & Set to \emph{true} to displaying Van den Bergh Catalogue of reflection nebulae (VdB). Default value: \emph{false}. \tabularnewline
\midrule
flag\_show\_rcw & bool & Set to \emph{true} to displaying a catalogue of H$\alpha$-emission regions in the southern Milky Way (RCW). Default value: \emph{false}. \tabularnewline
\midrule
flag\_show\_lbn & bool & Set to \emph{true} to displaying Lynds' Catalogue of Bright Nebulae (LBN). Default value: \emph{false}. \tabularnewline
\midrule
flag\_show\_ldn & bool & Set to \emph{true} to displaying Lynds' Catalogue of Dark Nebulae (LDN). Default value: \emph{false}. \tabularnewline
\midrule
flag\_show\_cr & bool & Set to \emph{true} to displaying Collinder Catalogue (Cr). Default value: \emph{false}. \tabularnewline
\midrule
flag\_show\_mel & bool & Set to \emph{true} to displaying Melotte Catalogue of Deep Sky Objects (Mel). Default value: \emph{false}. \tabularnewline
\midrule
flag\_show\_pgc & bool & Set to \emph{true} to displaying HYPERLEDA. I. Catalog of galaxies (PGC). Default value: \emph{false}. \tabularnewline
\midrule
flag\_show\_ced & bool & Set to \emph{true} to displaying Cederblad Catalog of bright diffuse Galactic nebulae (Ced). Default value: \emph{false}. \tabularnewline
\midrule
flag\_show\_ugc & bool & Set to \emph{true} to displaying The Uppsala General Catalogue of Galaxies (UGC). Default value: \emph{false}. \tabularnewline
\bottomrule
\end{longtabu}

\subsection{Section
\big[dso\_type\_filters\big]}\label{section-dsoux5ftypeux5ffilters}

\begin{longtabu} to \textwidth {l|l|X}
\toprule
\emph{ID} & \emph{Type} & \emph{Description}\tabularnewline
\midrule
flag\_show\_galaxies & bool & Set to \emph{true}
to displaying galaxies. Default value: \emph{true}. \tabularnewline
\midrule
flag\_show\_active\_galaxies & bool & Set to
\emph{true} to displaying active galaxies. Default value: \emph{true}. \tabularnewline
\midrule
flag\_show\_interacting\_galaxies & bool & Set
to \emph{true} to displaying interacting galaxies. Default value: \emph{true}. \tabularnewline
\midrule
flag\_show\_clusters & bool & Set to \emph{true}
to displaying star clusters. Default value: \emph{true}. \tabularnewline
\midrule
flag\_show\_bright\_nebulae & bool & Set to
\emph{true} to displaying bright nebulae. Default value: \emph{true}. \tabularnewline
\midrule
flag\_show\_dark\_nebulae & bool & Set to
\emph{true} to displaying dark nebulae. Default value: \emph{true}. \tabularnewline
\midrule
flag\_show\_planetary\_nebulae & bool & Set to
\emph{true} to displaying planetary nebulae. Default value: \emph{true}. \tabularnewline
\midrule
flag\_show\_hydrogen\_regions & bool & Set to
\emph{true} to displaying hydrogen regions. Default value: \emph{true}. \tabularnewline
\midrule
flag\_show\_supernova\_remnants & bool & Set to
\emph{true} to displaying supernovae remnants. Default value: \emph{true}. \tabularnewline
\midrule
flag\_show\_other & bool & Set to \emph{true} to
displaying other deep-sky objects. Default value: \emph{true}. \tabularnewline
\bottomrule
\end{longtabu}

\subsection{Section \big[gui\big]}\label{section-gui}

\begin{longtabu} to \textwidth {l|l|X}
\toprule
\emph{ID} & \emph{Type} & \emph{Description}\tabularnewline
\midrule
base\_font\_size & int & Sets the font size. Typical value:
\emph{15}\tabularnewline
\midrule
base\_font\_name & string & Selects the name for base font, e.g. \emph{DejaVu
Sans}\tabularnewline
\midrule
safe\_font\_name & string & Selects the name for safe font,
e.g. \emph{Verdana}\tabularnewline
\midrule
base\_font\_file & string & Selects the name for font file,
e.g. \emph{DejaVuSans.ttf}\tabularnewline
\midrule
flag\_show\_fps & bool & Set to \emph{false} if you don't
want to see at how many frames per second Stellarium is
rendering\tabularnewline
\midrule
flag\_show\_fov & bool & Set to \emph{false} if you don't
want to see how many degrees your field of view is\tabularnewline
\midrule
flag\_mouse\_cursor\_timeout & float & Set to \emph{0} if you
want to keep the mouse cursor visible at all times. non-0 values mean
the cursor will be hidden after that many seconds of
inactivity\tabularnewline
\midrule
flag\_show\_flip\_buttons & bool & Enables/disables display of
the image flipping buttons in the main toolbar (see section
\ref{imageflipping})\tabularnewline
\midrule
flag\_show\_nebulae\_background\_button & bool & Set to
\emph{true} if you want to have access to the button for
enabling/disabling backgrounds for deep-sky objects\tabularnewline
\midrule
flag\_use\_degrees & bool &\tabularnewline
\midrule
selected\_object\_info & string & Values: \emph{all},
\emph{short} and \emph{none}. Value \emph{custom} added since version
0.12.0.\tabularnewline
\midrule
auto\_hide\_horizontal\_toolbar & bool & Set to \emph{true} if
you want auto hide the horizontal toolbar.\tabularnewline
\midrule
auto\_hide\_vertical\_toolbar & bool & Set to \emph{true} if
you want auto hide the vertical toolbar.\tabularnewline
\midrule
day\_key\_mode & string & Specifies the amount of time which is
added and subtracted when the {[} {]} - and = keys are pressed -
calendar days, or sidereal days. This option only makes sense for
Digitalis planetariums. Values: \emph{calendar} or
\emph{sidereal}\tabularnewline
\midrule
flag\_use\_window\_transparency & bool & Set to \emph{false}
if you want see the opacity for bars\tabularnewline
\midrule
flag\_show\_datetime & bool & Set to \emph{true} if you want
display date and time in the bottom bar\tabularnewline
\midrule
flag\_time\_jd & bool & Set to \emph{true} if you want using
JD format for time in the bottom bar\tabularnewline
\midrule
flag\_show\_location & bool & Set to \emph{true} if you want
display location in the bottom bar\tabularnewline
\midrule
flag\_fov\_dms & bool & Set to \emph{true} if you want using
DMS format for FOV in the bottom bar\tabularnewline
\midrule
flag\_show\_decimal\_degrees & bool & Set to \emph{true} if
you want use decimal degrees for coordinates\tabularnewline
\midrule
flag\_use\_azimuth\_from\_south & bool & Set to \emph{true} if
you want calculate azimuth from south towards west (as in old
astronomical literature)\tabularnewline
\midrule
flag\_show\_gui & bool & Set to \emph{true} if you want
display GUI\tabularnewline
\bottomrule
\end{longtabu}

\subsection{Section
\big[init\_location\big]}\label{section-init-location}

\begin{longtabu} to \textwidth {l|l|X}
\toprule
\emph{ID} & \emph{Type} & \emph{Description}\tabularnewline
\midrule
landscape\_name & string & Sets the landscape you see.
Other options are \emph{garching, guereins, trees, moon, ocean,
hurricane, hogerielen}\tabularnewline
\midrule
location & string & Name of location on which to start
stellarium.\tabularnewline
\midrule
last\_location & string & Coordinates of last used location in
stellarium.\tabularnewline
\bottomrule
\end{longtabu}

\subsection{Section \big[landscape\big]}\label{section-landscape}

\begin{longtabu} to \textwidth {l|l|X}
\toprule
\emph{ID} & \emph{Type} & \emph{Description}\tabularnewline
flag\_landscape & bool & Set to false if you don't want to
see the landscape at all\tabularnewline
\midrule
flag\_fog & bool & Set to \emph{false} if you don't want to
see fog on start-up\tabularnewline
\midrule
flag\_atmosphere & bool & Set to \emph{false} if you don't
want to see atmosphere on start-up\tabularnewline
\midrule
flag\_landscape\_sets\_location & bool & Set to \emph{true}
if you want Stellarium to modify the observer location when a new
landscape is selected (changes planet and longitude/latitude/altitude if
that data is available in the landscape.ini file)\tabularnewline
\midrule
minimal\_brightness & float & Set minimal brightness for
landscapes. Typical value: \emph{0.01}\tabularnewline
\midrule
atmospheric\_extinction\_coefficient & float & Set atmospheric
extinction coefficient\tabularnewline
\midrule
temperature\_C & float & Set atmospheric temperature
(Celsius)\tabularnewline
\midrule
pressure\_mbar & float & Set atmospheric pressure
(mbar)\tabularnewline
\bottomrule
\end{longtabu}

\subsection{Section \big[localization\big]}\label{section-localization}

\begin{longtabu} to \textwidth {l|l|X}
\toprule
\emph{ID} & \emph{Type} & \emph{Description}\tabularnewline
\midrule
sky\_culture & string & Sets the sky culture to use. E.g.
\emph{western, polynesian, egyptian, chinese, lakota, navajo, inuit,
korean, norse, tupi, maori, aztec, sami}.\tabularnewline
\midrule
sky\_locale & string & Sets language used for names of
objects in the sky (e.g. planets). The value is a short locale code,
e.g. \emph{en, de, en\_GB}\tabularnewline
\midrule
app\_locale & string & Sets language used for Stellarium's
user interface. The value is a short locale code, e.g. \emph{en, de,
en\_GB}.\tabularnewline
\midrule
time\_zone & string & Sets the time zone. Valid values:
\emph{system\_default}, or some region/location combination, e.g.
\emph{Pacific/Marquesas}\tabularnewline
\midrule
time\_display\_format & string & Set the time display format
mode: can be \emph{system\_default}, \emph{24h} or
\emph{12h}.\tabularnewline
\midrule
date\_display\_format & string & Set the date display format
mode: can be \emph{system\_default}, \emph{mmddyyyy}, \emph{ddmmyyyy} or
\emph{yyyymmdd} (ISO8601).\tabularnewline
\bottomrule
\end{longtabu}

\subsection{Section \big[main\big]}\label{section-main}

\begin{longtabu} to \textwidth {l|l|X}
\toprule
\emph{ID} & \emph{Type} & \emph{Description}\tabularnewline
\midrule
invert\_screenshots\_colors & bool & If \emph{true},
Stellarium will saving the screenshorts with inverted
colors.\tabularnewline
\midrule
restore\_defaults & bool & If \emph{true}, Stellarium will be
restore default settings at startup.\tabularnewline
\midrule
screenshot\_dir & string & Path for saving
screenshots\tabularnewline
\midrule
version & string & Version of Stellarium. This parameter
using for updating config.ini file\tabularnewline
\midrule
use\_separate\_output\_file & bool & Set to \emph{true} if you
want to create a new file for output for each start of
Stellarium\tabularnewline
\midrule
check\_requirements & bool & Set to \emph{false} if you want
to disable checking hardware requirements at startup. This options for
developers only!\tabularnewline
\bottomrule
\end{longtabu}

\subsection{Section \big[navigation\big]}\label{section-navigation}

\begin{longtabu} to \textwidth {l|l|X}
\toprule
\emph{ID} & \emph{Type} & \emph{Description}\tabularnewline
\midrule
preset\_sky\_time & float & Preset sky time used by the
dome version. Unit is Julian Day. Typical value:
\emph{2451514.250011573}\tabularnewline
\midrule
startup\_time\_mode & string & Set the start-up time mode,
can be \emph{actual} (start with current real world time), or
\emph{Preset} (start at time defined by
preset\_sky\_time)\tabularnewline
\midrule
flag\_enable\_zoom\_keys & bool & Set to \emph{false} if
you want to disable the zoom\tabularnewline
\midrule
flag\_manual\_zoom & bool & Set to'' false'' for normal
zoom behaviour as described in this guide. When set to true, the auto
zoom feature only moves in a small amount and must be pressed many
times\tabularnewline
\midrule
flag\_enable\_move\_keys & bool & Set to \emph{false} if
you want to disable the arrow keys\tabularnewline
\midrule
flag\_enable\_mouse\_navigation & bool & Set to
\emph{false} if you want to disable the mouse navigation.\tabularnewline
\midrule
init\_fov & float & Initial field of view, in degrees,
typical value:'' 60''\tabularnewline
\midrule
init\_view\_pos & floats & Initial viewing direction. This
is a vector with x,y,z-coordinates. x being N-S (S +ve), y being E-W (E
+ve), z being up-down (up +ve). Thus to look South at the horizon use
\emph{1,0,0}. To look Northwest and up at $45\degree$, use \emph{-1,-1,1} and so
on.\tabularnewline
\midrule
auto\_move\_duration & float & Duration for the program to
move to point at an object when the space bar is pressed. Typical value:
\emph{1.4}\tabularnewline
\midrule
mouse\_zoom & float & Sets the mouse zoom amount
(mouse-wheel)\tabularnewline
\midrule
move\_speed & float & Sets the speed of
movement\tabularnewline
\midrule
zoom\_speed & float & Sets the zoom speed\tabularnewline
\midrule
viewing\_mode & string & If set to \emph{horizon}, the
viewing mode simulate an alt/azi mount, if set to \emph{equator}, the
viewing mode simulates an equatorial mount\tabularnewline
\midrule
flag\_manual\_zoom & bool & Set to \emph{true} if you want
to auto-zoom in incrementally.\tabularnewline
\midrule
auto\_zoom\_out\_resets\_direction & bool & Set to
\emph{true} if you want to auto-zoom restoring direction.\tabularnewline
\midrule
time\_correction\_algorithm & string & Algorithm of DeltaT
correction.\tabularnewline
\bottomrule
\end{longtabu}

\subsection{Section
\big[plugins\_load\_at\_startup\big]}\label{section-plugins-load-at-startup}

\begin{longtabu} to \textwidth {l|l|X}
\toprule
\emph{ID} & \emph{Type} & \emph{Description}\tabularnewline
\midrule
AngleMeasure & bool & If \emph{true}, Stellarium will be load
Angle Measure plugin at startup.\tabularnewline
\midrule
CompassMarks & bool & If \emph{true}, Stellarium will be load
Compass Marks plugin at startup.\tabularnewline
\midrule
Exoplanets & bool & If \emph{true}, Stellarium will be load
Exoplanets plugin at startup.\tabularnewline
\midrule
Observability & bool & If \emph{true}, Stellarium will be load
Observability Analysis plugin at startup.\tabularnewline
\midrule
Oculars & bool & If \emph{true}, Stellarium will be load
Oculars plugin at startup.\tabularnewline
\midrule
Pulsars & bool & If \emph{true}, Stellarium will be load
Pulsars plugin at startup.\tabularnewline
\midrule
Quasars & bool & If \emph{true}, Stellarium will be load
Quasars plugin at startup.\tabularnewline
\midrule
Satellites & bool & If \emph{true}, Stellarium will be load
Satellites plugin at startup.\tabularnewline
\midrule
SolarSystemEditor & bool & If \emph{true}, Stellarium will be
load Solar System Editor plugin at startup.\tabularnewline
\midrule
Supernovae & bool & If \emph{true}, Stellarium will be load
Historical Supernovae plugin at startup.\tabularnewline
\midrule
TelescopeControl & bool & If \emph{true}, Stellarium will be
load Telescope Control plugin at startup.\tabularnewline
\midrule
TextUserInterface & bool & If \emph{true}, Stellarium will be
load Text User Interface plugin at startup.\tabularnewline
\midrule
TimeZoneConfiguration & bool & If \emph{true}, Stellarium will
be load Time Zone plugin at startup.\tabularnewline
\midrule
Novae & bool & If \emph{true}, Stellarium will be load Bright
Novae plugin at startup.\tabularnewline
\midrule
Scenery3dMgr & bool & If \emph{true}, Stellarium will be load
Scenery 3D plugin at startup.\tabularnewline
\bottomrule
\end{longtabu}

\subsection{Section \big[projection\big]}\label{section-projection}

\begin{longtabu} to \textwidth {l|l|X}
\toprule
\emph{ID} & \emph{Type} & \emph{Description}\tabularnewline
\midrule
type & string & Sets projection mode. Values: \emph{ProjectionPerspective,
ProjectionEqualArea, ProjectionStereographic, ProjectionFisheye,
ProjectionHammer, ProjectionCylinder, ProjectionMercator} or
\emph{ProjectionOrthographic}.\tabularnewline
\midrule
viewport & string & How the view-port looks. Values:
\emph{none, disk}.\tabularnewline
\midrule
viewportMask & string & How the view-port looks. Values:
\emph{none}.\tabularnewline
\midrule
flag\_use\_gl\_point\_sprite & bool & \tabularnewline
\midrule
flip\_horz & bool & \tabularnewline
\midrule
flip\_vert & bool & \tabularnewline
\bottomrule
\end{longtabu}

\subsection{Section \big[proxy\big]}\label{section-proxy}

\begin{longtabu} to \textwidth {l|l|X}
\toprule
\emph{ID} & \emph{Type} & \emph{Description}\tabularnewline
\midrule
host\_name & string & Name of host for proxy. E.g. \emph{proxy.org}\tabularnewline
\midrule
port & int & Port of proxy. E.g. \emph{8080}\tabularnewline
\midrule
user & string & Username for proxy. E.g. \emph{michael\_knight}\tabularnewline
\midrule
password & string & Password for proxy. E.g. \emph{xxxxx}\tabularnewline
\bottomrule
\end{longtabu}

\subsection{Section \big[scripts\big]}\label{section-scripts}

\begin{longtabu} to \textwidth {l|l|X}
\toprule
\emph{ID} & \emph{Type} & \emph{Description}\tabularnewline
\midrule
flag\_script\_allow\_ui & bool &\tabularnewline
\midrule
scripting\_allow\_write\_files & bool &\tabularnewline
\bottomrule
\end{longtabu}

\subsection{Section \big[search\big]}\label{section-search}

\begin{longtabu} to \textwidth {l|l|X}
\toprule
\emph{ID} & \emph{Type} & \emph{Description}\tabularnewline
\midrule
flag\_search\_online & bool & If \emph{true}, Stellarium will be use SIMBAD for search.\tabularnewline
\midrule
simbad\_server\_url & string & URL for SIMBAD mirror\tabularnewline
\midrule
flag\_start\_words & bool & If \emph{false}, Stellarium will be search phrase only from start of words\tabularnewline
\midrule
coordinate\_system & string & Specifies the coordinate system. \emph{Possible values:} equatorialJ2000, equatorial, horizontal,
galactic. \emph{Default value:} equatorialJ2000.\tabularnewline
\bottomrule
\end{longtabu}

\subsection{Section
\big[spheric\_mirror\big]}\label{section-sphericux5fmirror}

\begin{longtabu} to \textwidth {l|l|X}
\toprule
\emph{ID} & \emph{Type} & \emph{Description}\tabularnewline
\midrule
distorter\_max\_fov & float & Set the maximum field of view
for the spheric mirror distorter in degrees. Typical value,
\emph{180}\tabularnewline
\midrule
flag\_use\_ext\_framebuffer\_object & bool & Some video
hardware incorrectly claims to support some GL extension,
GL\_FRAMEBUFFER\_EXTEXT. If, when using the spheric mirror distorter the
frame rate drops to a very low value (e.g. 0.1 FPS), set this parameter
to false to tell Stellarium ignore the claim of the video driver that it
can use this extension\tabularnewline
\midrule
flip\_horz & bool & Flip the projection horizontally\tabularnewline
\midrule
flip\_vert & bool & Flip the projection vertically\tabularnewline
\midrule
projector\_gamma & float & This parameter controls the
properties of the spheric mirror projection mode.\tabularnewline
\midrule
projector\_position\_x & float & \tabularnewline
\midrule
projector\_position\_y & float & \tabularnewline
\midrule
projector\_position\_z & float & \tabularnewline
\midrule
mirror\_position\_x & float & \tabularnewline
\midrule
mirror\_position\_y & float & \tabularnewline
\midrule
mirror\_position\_z & float & \tabularnewline
\midrule
mirror\_radius & float & \tabularnewline
\midrule
dome\_radius & float & \tabularnewline
\midrule
zenith\_y & float & \tabularnewline
\midrule
scaling\_factor & float & \tabularnewline
\bottomrule
\end{longtabu}

\subsection{Section \big[stars\big]}\label{section-stars}

\begin{longtabu} to \textwidth {l|l|X}
\toprule
\emph{ID} & \emph{Type} & \emph{Description}\tabularnewline
\midrule
relative\_scale & float & Changes the relative size of
bright and faint stars. Higher values mean that bright stars are
comparitively larger when rendered. Typical value:
\emph{1.0}\tabularnewline
\midrule
absolute\_scale & float & Changes how large stars are
rendered. larger value lead to larger depiction. Typical value:
\emph{1.0}\tabularnewline
\midrule
star\_twinkle\_amount & float & Sets the amount of
twinkling. Typical value: \emph{0.3}\tabularnewline
\midrule
flag\_star\_twinkle & bool & Set to \emph{false} to turn
star twinkling off, \emph{true} to allow twinkling.\tabularnewline
\midrule
mag\_converter\_max\_fov & float & Sets the maximum field
of view for which the magnitude conversion routine is used. Typical
value: \emph{90.0}.\tabularnewline
\midrule
mag\_converter\_min\_fov & float & Sets the minimum field
of view for which the magnitude conversion routine is used. Typical
value: \emph{0.001}.\tabularnewline
\midrule
labels\_amount & float & Sets the amount of labels. Typical
value: \emph{3.0}\tabularnewline
\midrule
init\_bortle\_scale & int & Sets the initial value of the
bortle scale. Typical value: \emph{3}.\tabularnewline
\bottomrule
\end{longtabu}

\subsection{Section \big[tui\big]}\label{section-tui}

\begin{longtabu} to \textwidth {l|l|X}
\toprule
\emph{ID} & \emph{Type} & \emph{Description}\tabularnewline
\midrule
flag\_enable\_tui\_menu & bool & Enables or disables the TUI menu\tabularnewline
\midrule
flag\_show\_gravity\_ui & bool & Enables or disables gravity mode for UI\tabularnewline
\midrule
flag\_show\_tui\_datetime & bool & Set to \emph{true} if you want to see a date and time label suited for dome projections\tabularnewline
\midrule
flag\_show\_tui\_short\_obj\_info & bool & set to \emph{true} if you want to see object info suited for dome projections\tabularnewline
\bottomrule
\end{longtabu}

\subsection{Section \big[video\big]}\label{section-video}

\begin{longtabu} to \textwidth {l|l|X}
\toprule
\emph{ID} & \emph{Type} & \emph{Description}\tabularnewline
\midrule
fullscreen & bool & If \emph{true}, Stellarium will start
up in full-screen mode. If \emph{false}, Stellarium will start in
windowed mode\tabularnewline
\midrule
screen\_w & int & Sets the display width when in windowed mode. Value in pixels, e.g. \emph{1024}\tabularnewline
\midrule
screen\_h & int & Sets the display height when in windowed mode. Value in pixels, e.g. \emph{768}\tabularnewline
\midrule
screen\_x & int & Value in pixels, e.g. \emph{0}\tabularnewline
\midrule
screen\_y & int & Value in pixels, e.g. \emph{0}\tabularnewline
\midrule
viewport\_effect & string & This is used when the spheric mirror display mode is activated. Values include \emph{none} and \emph{sphericMirrorDistorter}.\tabularnewline
\midrule
minimum\_fps & int & Sets the minimum number of frames per second to display at (hardware performance permitting)\tabularnewline
\midrule
maximum\_fps & int & Sets the maximum number of frames per second to display at. This is useful to reduce power consumption in laptops.\tabularnewline
\bottomrule
\end{longtabu}

\subsection{Section \big[viewing\big]}\label{section-viewing}

\begin{longtabu} to \textwidth {l|l|X}
\toprule
\emph{ID} & \emph{Type} & \emph{Description}\tabularnewline
\midrule
atmosphere\_fade\_duration & float & Sets the time it takes for
the atmosphere to fade when de-selected\tabularnewline
\midrule
flag\_constellation\_drawing & bool & Set to \emph{true} if you want to see the constellation line drawing on start-up\tabularnewline
\midrule
flag\_constellation\_name & bool & Set to \emph{true} if you want to see the constellation names on start-up\tabularnewline
\midrule
flag\_constellation\_art & bool & Set to \emph{true} if you want to see the constellation art on start-up\tabularnewline
\midrule
flag\_constellation\_boundaries & bool & Set to \emph{true} if you want to see the constellation boundaries on start-up\tabularnewline
\midrule
flag\_constellation\_isolate\_selected & bool & When set to \emph{true}, constellation lines, boundaries and art will be limited to the constellation of the selected star, if that star is ''on'' one of the constellation lines.\tabularnewline
\midrule
flag\_constellation\_pick & bool & Set to \emph{true} if you only want to see the line drawing, art and name of the selected constellation star\tabularnewline
\midrule
flag\_isolated\_trails & bool & Set to \emph{true} if you only want to see the tail line drawing of the selected planets (asteroids, comets, moons)\tabularnewline
\midrule
flag\_azimutal\_grid & bool & Set to \emph{true} if you want to see the azimuthal grid on start-up\tabularnewline
\midrule
flag\_equatorial\_grid & bool & Set to \emph{true} if you want to see the equatorial grid (on date) on start-up\tabularnewline
\midrule
flag\_equatorial\_J2000\_grid & bool & Set to \emph{true} if you want to see the equatorial grid (J2000) on start-up\tabularnewline
\midrule
flag\_ecliptic\_J2000\_grid & bool & Set to \emph{true} if you want to see the ecliptic grid (J2000) on start-up\tabularnewline
\midrule
flag\_galactic\_grid & bool & Set to \emph{true} if you want to see the galactic grid on start-up\tabularnewline
\midrule
flag\_galactic\_plane\_line & bool & Set to \emph{true} if you want to see the galactic plane line on start-up\tabularnewline
\midrule
flag\_equator\_line & bool & Set to \emph{true} if you want to see the equator line on start-up\tabularnewline
\midrule
flag\_ecliptic\_line & bool & Set to \emph{true} if you want to see the ecliptic line (J2000) on start-up\tabularnewline
\midrule
flag\_meridian\_line & bool & Set to \emph{true} if you want to see the meridian line on start-up\tabularnewline
\midrule
flag\_cardinal\_points & bool & Set to \emph{false} if you don't want to see the cardinal points\tabularnewline
\midrule
flag\_gravity\_labels & bool & Set to \emph{true} if you want labels to undergo gravity (top side of text points toward zenithzenith). Useful with dome projection.\tabularnewline
\midrule
flag\_moon\_scaled & bool & Change to \emph{false} if you want to see the real moon size on start-up\tabularnewline
\midrule
moon\_scale & float & Sets the moon scale factor, to correlate to our perception of the moon's size. Typical value: \emph{4}\tabularnewline
\midrule
constellation\_art\_intensity & float & This number multiplies the brightness of the constellation art images. Typical value: \emph{0.5}\tabularnewline
\midrule
constellation\_art\_fade\_duration & float & Sets the amount of time the constellation art takes to fade in or out, in seconds. Typical value: \emph{1.5}\tabularnewline
\midrule
constellation\_font\_size & int & Set font size for constellation labels\tabularnewline
\midrule
constellation\_line\_thickness & float & Set the thickness of lines of the constellations\tabularnewline
\midrule
flag\_night & bool & Enable night mode on startup\tabularnewline
\midrule
light\_pollution\_luminance & float & Sets the level of the light pollution simulation\tabularnewline
\midrule
use\_luminance\_adaptation & bool & Enable dynamic eye adaptation.\tabularnewline
\midrule
flag\_isolated\_trails & float & Change to \emph{false} if you want to see the trails for all celestial bodies from Solar system.\tabularnewline
\midrule
flag\_isolated\_orbits & float & Change to \emph{false} if you want to see orbits for selected planet and their moons.\tabularnewline
\bottomrule
\end{longtabu}

\section{Precision}
Stellarium uses the
\href{http://vizier.cfa.harvard.edu/viz-bin/ftp-index?/ftp/cats/VI/81}{VSOP87}
method to calculate the variation in position of the planets over time.

As with other methods, the precision of the calculations vary according
to the planet and the time for which one makes the calculation. Reasons
for these inaccuracies include the fact that the motion of the planet
isn't as predictable as Newtonian mechanics would have us believe.

As far as Stellarium is concerned, the user should bear in mind the
following properties of the VSOP87 method. Precision values here are
positional as observed from Earth.

\begin{longtabu} to \textwidth {X|l|X}
\toprule
\emph{Object(s)} & \emph{Method} & \emph{Notes}\tabularnewline
\midrule
Mercury, Venus, Earth-Moon barycenter, Mars & VSOP87 & Precision is 1
arc-second from 2000 B.C. -- 6000 A.D.\tabularnewline
\midrule
Jupiter, Saturn & VSOP87 & Precision is 1 arc-second from 0 A.D. -- 4000
A.D.\tabularnewline
\midrule
Uranus, Neptune & VSOP87 & Precision is 1 arc-second from 4000 B.C --
8000 A.D.\tabularnewline
\midrule
Pluto & ? & Pluto's position is valid from 1885 A.D. -- 2099
A.D.\tabularnewline
\midrule
Earth's Moon & ELP2000-82B & Unsure about interval of validity or
precision at time of writing. Possibly valid from 1828 A.D. to 2047
A.D.\tabularnewline
\midrule
Galilean satellites & L2 & Valid from 500 A.D -- 3500
A.D.\tabularnewline
\bottomrule
\end{longtabu}

\section{TUI Commands}
\begin{longtabu} to \textwidth {l|l|X}
\toprule
1 & Set Location & (menu group)\tabularnewline
\midrule
1.1 & Latitude & Set the latitude of the observer in
degrees\tabularnewline
\midrule
1.2 & Longitude & Set the longitude of the observer in
degrees\tabularnewline
\midrule
1.3 & Altitude (m) & Set the altitude of the observer in
meters\tabularnewline
\midrule
1.4 & Solar System Body & Select the solar system body on which the
observer is\tabularnewline
\midrule
2 & Set Time & (menu group)\tabularnewline
\midrule
2.1 & Sky Time & Set the time and date for which Stellarium will
generate the view\tabularnewline
\midrule
2.2 & Set Time Zone & Set the time zone. Zones are split into continent
or region, and then by city or province\tabularnewline
\midrule
2.3 & Days keys & The setting ``Calendar'' makes the - = {[} {]} and
keys change the date value by calendar days (multiples of 24 hours). The
setting ``Sidereal day'' changes these keys to change the date by
sidereal days\tabularnewline
\midrule
2.4 & Preset Sky Time & Select the time which Stellarium starts with (if
the ``Sky Time At Start-up'' setting is ``Preset Time''\tabularnewline
\midrule
2.5 & Sky Time At Start-up & The setting ``Actual Time'' sets
Stellarium's time to the computer clock when Stellarium runs. The
setting ``Preset Time'' selects a time set in menu item ``Preset Sky
Time''\tabularnewline
\midrule
2.6 & Time Display Format & Change how Stellarium formats time values.
``system default'' takes the format from the computer settings, or it is
possible to select 24 hour or 12 hour clock modes\tabularnewline
\midrule
2.7 & Date Display Format & Change how Stellarium formats date values.
``system default'' takes the format from the computer settings, or it is
possible to select ``yyyymmdd'', ``ddmmyyyy'' or ``mmddyyyy''
modes\tabularnewline
\midrule
3 & General & (menu group)\tabularnewline
\midrule
3.1 & Sky Culture & Select the sky culture to use (changes constellation
lines, names, artwork)\tabularnewline
\midrule
3.2 & Sky Language & Change the language used to describe objects in the
sky\tabularnewline
\midrule
4 & Stars & (menu group)\tabularnewline
\midrule
4.1 & Show & Turn on/off star rendering\tabularnewline
\midrule
4.2 & Star Magnitude Multiplier & Can be used to change the brightness
of the stars which are visible at a given zoom level. This may be used
to simulate local seeing conditions - the lower the value, the less
stars will be visible\tabularnewline
\midrule
4.3 & Maximum Magnitude to Label & Changes how many stars get labelled
according to their apparent magnitude (if star labels are turned
on)\tabularnewline
\midrule
4.4 & Twinkling & Sets how strong the star twinkling effect is - zero is
off, the higher the value the more the stars will
twinkle.\tabularnewline
\midrule
5 & Colors & (menu group)\tabularnewline
\midrule
5.1 & Constellation Lines & Changes the colour of the constellation
lines\tabularnewline
\midrule
5.2 & Constellation Names & Changes the colour of the labels used to
name stars\tabularnewline
\midrule
5.3 & Constellation Art Intensity & Changes the brightness of the
constellation artconstellation art\tabularnewline
\midrule
5.4 & Constellation Boundaries & Changes the colour of the constellation
boundary lines\tabularnewline
\midrule
5.5 & Cardinal Points & Changes the colour of the cardinal points
markers\tabularnewline
\midrule
5.6 & Planet Names & Changes the colour of the labels for
planets\tabularnewline
\midrule
5.7 & Planet Orbits & Changes the colour of the orbital guide lines for
planets\tabularnewline
\midrule
5.8 & Planet Trails & Changes the colour of the planet trails
lines\tabularnewline
\midrule
5.9 & Meridian Line & Changes the colour of the meridian
line\tabularnewline
\midrule
5.10 & Azimuthal Grid & Changes the colour of the lines and labels for
the azimuthal grid\tabularnewline
\midrule
5.11 & Equatorial Grid & Changes the colour of the lines and labels for
the equatorial grid\tabularnewline
\midrule
5.12 & Equator Line & Changes the colour of the equator
line\tabularnewline
\midrule
5.13 & Ecliptic Line & Changes the colour of the ecliptic
line\tabularnewline
\midrule
5.14 & Nebula Names & Changes the colour of the labels for
nebulae\tabularnewline
\midrule
5.15 & Nebula Circles & Changes the colour of the circles used to denote
the positions of nebulae (only when enabled int he configuration file,
note this feature is off by default)\tabularnewline
\midrule
6 & Effects & (menu group)\tabularnewline
\midrule
6.1 & Light Pollution Luminance & Changes the intensity of the light
pollution simulation\tabularnewline
\midrule
6.2 & Landscape & Used to select the landscape which Stellarium draws
when ground drawing is enabled\tabularnewline
\midrule
6.3 & Manual zoom & Changes the behaviour of the / and \textbackslash{}
keys. When set to ``No'', these keys zoom all the way to a level defined
by object type (auto zoom mode). When set to ``Yes'', these keys zoom in
and out a smaller amount and multiple presses are
required\tabularnewline
\midrule
6.4 & Object Sizing Rule & When set to ``Magnitude'', stars are drawn
with a size based on their apparent magnitude. When set to ``Point'' all
stars are drawn with the same size on the screen\tabularnewline
\midrule
6.5 & Magnitude Sizing Multiplier & Changes the size of the stars when
``Object Sizing Rule'' is set to ``Magnitude''\tabularnewline
\midrule
6.6 & Milky Way intensity & Changes the brightness of the Milky Way
texture\tabularnewline
\midrule
6.7 & Maximum Nebula Magnitude to Label & Changes the magnitude limit
for labelling of nebulae\tabularnewline
\midrule
6.8 & Zoom Duration & Sets the time for zoom operations to take (in
seconds)\tabularnewline
\midrule
6.9 & Cursor Timeout & Sets the number of seconds of mouse inactivity
before the cursor vanishes\tabularnewline
\midrule
6.10 & Setting Landscape Sets Location & If ``Yes'' then changing the
landscape will move the observer location to the location for that
landscape (if one is known). Setting this to ``No'' means the observer
location is not modified when the landscape is changed\tabularnewline
\midrule
7 & Scripts & (menu group)\tabularnewline
\midrule
7.1 & Local Script & Run a script from the scripts sub-directory of the
User Directory or Installation Directory (see section
\href{Advanced_Use\#Files_and_Directories}{Files and
Directories})\tabularnewline
\midrule
7.2 & CD/DVD Script & Run a script from a CD or DVD (only used in
planetarium set-ups)\tabularnewline
\midrule
8 & Administration & (menu group)\tabularnewline
\midrule
8.1 & Load Default Configuration & Reset all settings according to the
main configuration file\tabularnewline
\midrule
8.2 & Save Current Configuration as Default & Save the current settings
to the main configuration file\tabularnewline
\midrule
8.3 & Shutdown & Quit Stellarium\tabularnewline
\midrule
8.4 & Update me via Internet & Only used in planetarium
set-ups\tabularnewline
\midrule
8.5 & Set UI Locale & Change the language used for the user
interface\tabularnewline
\bottomrule
\end{longtabu}


