\chapterimage{chapter-bg.png} % Chapter heading image

\chapter{Installation}

\section{System Requirements}\label{system-requirements}

\subsection{Minimum}
\begin{itemize}
\item Linux/Unix; Windows 7 and above\footnote{Since version 0.14.0 only
  classic application can be work on Windows XP. Version 0.12.4 required
  Windows 2000 and above}; OS X 10.7.4 and above\footnote{Version 0.12.4
  required OS X 10.6.8. Version 0.10.6 required OS X 10.5. Version
  0.10.2 runs on OS X 10.3.9}
\item 3D graphics card which supports OpenGL 3.0 and GLSL 1.3\footnote{Version
  0.12.4 required 3D graphics card which supports OpenGL 1.2}
\item 512 MB RAM
\item 250 MB free on disk
\end{itemize}

\subsection{Recommended}
\begin{itemize}
\item Linux/Unix; Windows 7 and above; OS X 10.8.5 and above
\item 3D graphics card which supports OpenGL 3.3 and above\footnote{Version
  0.12.4 required 3D graphics card which supports OpenGL 2.1 and above}
\item 1 GB RAM or more
\item 1.5 GB free on disk
\end{itemize}
 A dark room for realistic rendering --- details like the Milky Way or
star twinkling can't be seen in a bright room.

\section{Downloading}\label{downloading}

Download the correct package for your operating system directly from the
\href{http://stellarium.org}{main page}.

\section{Installation}\label{installation-stellarium}

\subsection{Windows}\label{windows}

\begin{enumerate}
\item
  Double click on the \texttt{stellarium-0.14.1-win32.exe} (or
  \texttt{stellarium-0.14.1-win64.exe} on Windows x64) file to run the
  installer\footnote{Or \texttt{stellarium-0.14.1-classic-win32.exe} on
    Windows XP.}.
\item
  Follow the on-screen instructions.
\end{enumerate}

\subsection{MacOS X}\label{macos-x}

\begin{enumerate}
\item
  Locate the Stellarium-0.14.1.dmg\footnote{Or
    \texttt{Stellarium-0.10.6-PPC.dmg} for PowerPC MacOS X} file in
  Finder and double click on it or open it using the Disk Utility
  application. Now, a new disk appears on your desktop and Stellarium is
  in it.
\item
  Open the new disk and please take a moment to read the ReadMe file.
  Then drag Stellarium to the Applications folder.
\item
  Note: You should copy Stellarium to the Applications folder before
  running it --- some users have reported problems running it directly
  from the disk image (.dmg).
\end{enumerate}

\subsection{Linux}\label{linux}

Check if your distribution has a package for Stellarium already --- if
so you're probably best off using it. If not, you can download and build
the source.

For ubuntu we provide a package repository with the latest stable
releases. Open a terminal and type:

\texttt{sudo~add-apt-repository~ppa:stellarium/stellarium-releases}\\
\texttt{sudo~apt-get~update}\\
\texttt{sudo~apt-get~install~stellarium}

\section{Running Stellarium}\label{running-stellarium}

\begin{itemize}
\item
  \textbf{Windows} The Stellarium installer creates an item in the
  \textbf{Start Menu} under in \textbf{Programs} section. Select this to
  run \emph{Stellarium}.
\item
  \textbf{MacOS X} Double click on the \emph{Stellarium} application.
  Add it to your \textbf{Dock} for quick access.
\item
  \textbf{Linux} If your distribution had a package you'll probably
  already have an item in the GNOME or KDE application menus. If not,
  just use a open a terminal and type \texttt{stellarium}.
\end{itemize}