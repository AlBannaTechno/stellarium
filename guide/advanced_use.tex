\chapter{Files and Directories}
\label{sec:FilesAndDirectories}

\section{Directories}
\label{sec:Directories}

Stellarium has many data files containing such things as star catalogue
data, nebula images, button icons, font files and configuration files.
When Stellarium looks for a file, it looks in two places. First, it
looks in the \emph{user directory} for the account which is running
Stellarium. If the file is not found there, Stellarium looks in the
\emph{installation directory}\footnote{The installation directory was
  referred to as the config root directory in previous versions of this
  guide}. Thus it is possible for Stellarium to be installed as an
administrative user and yet have a writable configuration file for
non-administrative users. Another benefit of this method is on
multi-user systems: Stellarium can be installed by the administrator,
and different users can maintain their own configuration and other files
in their personal user accounts.

In addition to the main search path, Stellarium saves some files in
other locations, for example screens shots and recorded scripts.

The locations of the user directory, installation directory,
\emph{screenshot save directory} and \emph{script save directory} vary
according to the operating system and installation options used. The
following sections describe the locations for various operating systems.

\subsection{Windows}
\label{sec:FilesAndDirectories:Windows}

\begin{description}
\item[installation directory] By default this is
  \file{C:\textbackslash{}Program\ Files\textbackslash{}Stellarium\textbackslash{}},
  although this can be adjusted during the installation process.
\item[user directory] This is the Stellarium sub-folder in the
  Application Data folder for the user account which is used to run
  Stellarium. Depending on the version of Windows and its configuration,
  this could be any of the following (each of these is tried, if it
  fails, the next in the list if tried).

\begin{commands}
%APPDATA%\Stellarium\
%USERPROFILE%\Stellarium\
%HOMEDRIVE%\%HOMEPATH%\Stellarium\
%HOME%\Stellarium\
Stellarium's installation directory
\end{commands}

Thus, on a typical Windows Vista/7/10 system with user ``Bob
Dobbs'', the user directory will be:

\begin{commands}
C:\Users\Bob Dobbs\AppData\Roaming\Stellarium\
\end{commands}

\item[screenshot save directory] Screenshots will be saved to the
  \file{Pictures/Stellarium} directory, although this can be changed with a command line option (see
  section~\ref{sec:CommandLineOptions}\footnote{Windows Vista users who do not run Stellarium with
    administrator priviliges should adjust the shortcut in the start
    menu to specify a different directory for screenshots as the Desktop
    directory is not writable for normal progams. 
    Stellarium includes a GUI option to specify the screenshot
    directory.}).

\end{description}

\subsection{Mac OS X}
\label{sec:FilesAndDirectories:MacOSX}

\begin{description}
\item[installation directory] This is found inside the application
  bundle, \file{Stellarium.app}. See the \emph{Inside Application
    Bundles}\footnote{\url{http://www.mactipsandtricks.com/articles/Wiley_HT_appBundles.lasso}}
  for more information.
\item[user directory] This is the
  \file{Library/Preferences/Stellarium/} (or
  \file{\textasciitilde{}/Library/Application\ Support/Stellarium} on
  newest versions of Mac OS X) sub-directory of the users home
  directory.
\item[screenshot save directory] Screenshots are saved to the user's
  Desktop.
\end{description}

\subsection{Linux}
\label{sec:FilesAndDirectories:Linux}

\begin{description}
\item[installation directory] This is in the
  \texttt{share/stellarium} sub-directory of the installation prefix,
  i.e. usually \file{/usr/share/stellarium} or
  \file{/usr/local/share/stellarium/}.
\item[user directory] This is the \texttt{.stellarium} sub-directory
  of users home directory, i.e. \texttt{\textasciitilde{}/.stellarium/}.
\item[screenshot save directory] Screenshots are saved to the users
  home directory.
\end{description}

\section{Directory Structure}
\label{sec:FilesAndDirectories:DirectoryStructure}

Within the \emph{installation directory} and \emph{user directory}
defined in section~\ref{sec:Directories}, files are arranged in the
following sub-directories.

\begin{description}
\item[\file{landscapes/}] contains data files and textures used for
  Stellarium's various landscapes. Each landscape has it's own
  sub-directory. The name of this sub-directory is called the
  \emph{landscape ID}, which is used to specify the default landscape in
  the main configuration file.
\item[\file{skycultures/}] contains constellations, common star names and
  constellation artwork for Stellarium's many sky cultures. Each culture
  has it's own sub-directory in the skycultures directory.
\item[\file{nebulae/}] contains data and image files for nebula textures.
  In future Stellarium will be able to support multiple sets of nebula
  images and switch between them at runtime. This feature is not
  implemented for version 0.9.1, although the directory structure is in
  place - each set of nebula textures has it's own sub-directory in the
  nebulae directory.
\item[\file{stars/}] contains Stellarium's star catalogues. In future
  Stellarium will be able to support multiple star catalogues and switch
  between them at runtime. This feature is not implemented for version
  0.10.0, although the directory structure is in place - each star
  catalogue has it's own sub-directory in the stars directory.
\item[\file{data/}] contains miscellaneous data files including fonts,
  solar system data, city locations etc.
\item[\file{textures/}] contains miscellaneous texture files, such as the
  graphics for the toolbar buttons, planet texture maps etc.
\item[\file{ephem/}] (optional) may contain data files for planetary
  ephemerides DE430 and DE431 (see~\ref{sec:ExtraData:ephemerides}).
\end{description}

If any file exists in both the installation directory and user
directory, the version in the user directory will be used. Thus it is
possible to override settings which are part of the main Stellarium
installation by copying the relevant file to the user area and modifying
it there.

It is also possible to add new landscapes by creating the relevant files
and directories within the user directory, leaving the installation
directory unchanged. In this manner different users on a multi-user
system can customise Stellarium without affecting the other users.

\chapter{The Main Configuration File}\label{the-main-configuration-file}
\label{sec:ConfigurationFile}

The main configuration file is read each time Stellarium starts up, and
settings such as the observer's location and display preferences are
taken from it. Ideally this mechanism should be totally transparent to
the user - anything that is configurable should be configured ``in'' the
program GUI. However, at time of writing Stellarium isn't quite complete
in this respect, despite improvements in version 0.10.0. Some settings
can only be changed by directly editing the configuration file. This
section describes some of the settings a user may wish to modify in this
way, and how to do it.

If the configuration file does not exist in the \emph{user directory}
when Stellarium is started (e.g. the first time the user starts the
program), one will be created with default values for all settings
(refer to section \href{Advanced_Use\#Files_and_Directories}{Files and
Directories} for the location of the user directory on your operating
system). The name of the configuration file is
\texttt{config.ini}\footnote{It is possible to specify a different name
  for the main configuration file using the \texttt{-\/-config-file}
  command line option. See section
  \href{Advanced_Use\#Command_Line_Options}{Command Line Options} for
  details.}.

The configuration file is a regular text file, so all you need to edit
it is a text editor like \emph{Notepad} on Windows, \emph{Text Edit} on
the Mac, or \emph{nano/vi/gedit} etc. on Linux.

The following sub-sections contain details on how to make commonly used
modifications to the configuration file. A complete list of
configuration file options and values may be found in the appendix,
\href{Configuration_file}{Configuration file}.

\chapter{Command Line Options}\label{command-line-options}
\label{sec:CommandLineOptions}

Stellarium's behaviour can be modified by providing parameters to the
program when it is run, via the command line. See table for a full list:

\begin{longtabu} to \textwidth {l|l|X}
\toprule
\emph{Option} & \emph{Option Parameter} & \emph{Description}\tabularnewline
\midrule
-\/-help or -h & {[}none{]} & Print a quick command line help message and exit. \tabularnewline
\midrule
-\/-version or -v & {[}none{]} & Print the program name and version information, and exit. \tabularnewline
\midrule
-\/-config-file or -c & config file name & Specify the configuration file name. The default value is \texttt{config.ini}.

The parameter can be a full path (which will be used verbatim) or a partial path.

Partial paths will be searched for inside the regular search paths
unless they start with a ``\texttt{.}'', which may be used to explicitly
specify a file in the current directory or similar.

For example, using the option \texttt{-c\ my\_config.ini} would resolve
to the file
\texttt{\textless{}user\ directory\textgreater{}/my\_config.ini} whereas
\texttt{-c\ ./my\_config.ini} can be used to explicitly say the file
\texttt{my\_config.ini} in the current working directory.
\tabularnewline
\midrule
-\/-restore-defaults & {[}none{]} & If this option is specified
Stellarium will start with the default configuration. Note: The old
configuration file will be overwritten. \tabularnewline
\midrule
-\/-user-dir & path & Specify the user data directory. \tabularnewline
\midrule
-\/-screenshot-dir & path & Specify the directory to which screenshots will be saved. \tabularnewline
\midrule
-\/-full-screen & yes or no & Over-rides the full screen setting in the config file. \tabularnewline
\midrule
-\/-home-planet & planet & Specify observer planet (English name). \tabularnewline
\midrule
-\/-altitude & altitude & Specify observer altitude in meters. \tabularnewline
\midrule
-\/-longitude & longitude & Specify latitude, e.g. +53d58'16.65" \tabularnewline
\midrule
-\/-latitude & latitude & Specify longitude, e.g. -1d4'27.48" \tabularnewline
\midrule
-\/-list-landscapes & {[}none{]} & Print a list of available landscape IDs. \tabularnewline
\midrule
-\/-landscape & landscape ID & Start using landscape whose ID matches
the passed parameter (dir name for landscape). \tabularnewline
\midrule
-\/-sky-date & date & The initial date in \texttt{yyyymmdd} format. \tabularnewline
\midrule
-\/-sky-time & time & The initial time in \texttt{hh:mm:ss} format. \tabularnewline
\midrule
-\/-startup-script & script name & The name of a script to run after the program has started. \tabularnewline
\midrule
-\/-fov & angle & The initial field of view in degrees. \tabularnewline
\midrule
-\/-projection-type & ptype & The initial projection type (e.g. \texttt{perspective}). \tabularnewline
\midrule
-\/-dump-opengl-details or -d & {[}none{]} & Dump information about
OpenGL support to logfile. Use this is you have graphics problems and
want to send a bug report. \tabularnewline
\midrule
-\/-angle-mode or -a & {[}none{]} & Use ANGLE as OpenGL ES2 rendering
engine (autodetect driver).\footnote{On Windows only}\tabularnewline
\midrule
-\/-angle-d3d9 or -9 & {[}none{]} & Force use Direct3D 9 for ANGLE OpenGL ES2 rendering engine.\footnotemark[4]\tabularnewline
\midrule
-\/-angle-d3d11 & {[}none{]} & Force use Direct3D 11 for ANGLE OpenGL ES2 rendering engine.\footnotemark[4]\tabularnewline
\midrule
-\/-angle-warp & {[}none{]} & Force use the Direct3D 11 software rasterizer for ANGLE OpenGL ES2 rendering engine.\footnotemark[4]\tabularnewline
\midrule
-\/-mesa-mode or -m & {[}none{]} & Use MESA as software OpenGL rendering engine.\footnotemark[4]\tabularnewline
\midrule
-\/-safe-mode or -s & {[}none{]} & Synonymous to -\/-mesa-mode.\footnotemark[4]\tabularnewline
\bottomrule
\end{longtabu}

\section{Examples}\label{examples}
\label{sec:CommandLineOptions:Examples}

\begin{itemize}
\item To start Stellarium using the configuration file,
  \file{configuration\_one.ini} situated in the user directory (use either of
  these):

\begin{commands}
stellarium --config-file=configuration_one.ini
stellarium -c configuration_one.ini
\end{commands}

\item To list the available landscapes, and then to start using the
  landscape with the ID, ``ocean''
\begin{commands}
stellarium --list-landscapes 
stellarium --landscape=ocean
\end{commands}
\end{itemize}

%% GZ found in 2015:
\noindent Note that console output (like \texttt{--list-landscapes}) on Windows is not possible. 

\chapter{Getting Extra Data}\label{getting-extra-star-data}

Stellarium is packaged with over 600 thousand stars in the normal
program download, but much larger star catalogues may be downloaded
using the tool which is in the \emph{Tools} tab of the
\emph{Configuration} dialog.


\section{Alternative Planet Ephemerides: DE430, DE431}
\label{sec:ExtraData:ephemerides}

By default, Stellarium uses the \indexterm{VSOP87} planetary theory,
an analytical solution which is able to deliver planetary positions
for any input date. However, its use is recommended only for the year
range $-4000\ldots+8000$. Outside this range, it seems to be usable
for a few more millennia without too great errors, but with degrading accuracy. 

Since V0.15 you can install extra data files which allow access to the
numerical integration runs \indexterm{DE430} and \indexterm{DE431}
from NASA's Jet Propulsion Laboratory (JPL). The data files have to be
downloaded separately, and most users will likely not need them. DE430
provides highly accurate data for the years $+1550\ldots+2650$, while
DE431 covers years $-13000\ldots+17000$, which allows e.g.\ 
archaeoastronomical research on Mesolithic landscapes. Outside these
year ranges, positional computation falls back to VSOP87.

To enable use of these data, download the files from
JPL\footnote{\url{ftp://ssd.jpl.nasa.gov/pub/eph/planets/Linux/}. (Also
  download from this directory if you are not running Linux!)}:

\begin{description}
\item[DE430] \file{linux\_p1550p2650.430} (md5 hash 707c4262533d52d59abaaaa5e69c5738)
\item[DE431] \file{lnxm13000p17000.431} (md5 hash fad0f432ae18c330f9e14915fbf8960a)
\end{description}


The files can be placed in a folder named \file{ephem} inside either
the \emph{installation directory} or the \emph{user directory}
(see \ref{sec:FilesAndDirectories:DirectoryStructure}). Alternatively,
if you have them already stored elsewhere, you may add the path to
\file{config.ini} like:

\begin{configfile}
[astro]
de430_path = C:/Astrodata/JPL_DE43x/linux_p1550p2650.430
de431_path = C:/Astrodata/JPL_DE43x/lnxm13000p17000.431
\end{configfile}

For fast access avoid storing them on a network drive os USB pendrive!

You activate use of either ephemeris in the configuration panel
(\key{F2}). If you activate both, preference will be given for DE430
if the simulation time allows it. Outside of the valid times, VSOP87
will always be used.


\chapter{Visual Effects}\label{visual-effects}

\section{Light Pollution}\label{light-pollution}

Stellarium can simulate light pollution, which is controlled from the
light pollution section of the \emph{Sky} tab of the \emph{View} window.
Light pollution levels are set using an numerical value between 1 and 9
which corresponds to the \emph{Bortle Dark Sky Scale}.

\subsection{Excellent dark sky site}
\textbf{Level:} 1 \\
\textbf{Colour:} black \\
\textbf{Limiting magnitude (eye):} 7.6 -- 8.0

Zodiacal light, gegenschein, zodiacal band visible; M33 direct vision naked-eye object; Scorpius and Sagittarius regions of the Milky Way cast obvious shadows on the ground; Airglow is readily visible; Jupiter and Venus affect dark adaptation; surroundings basically invisible.

\subsection{Typical truly dark site}
\textbf{Level:} 2 \\
\textbf{Colour:} grey \\
\textbf{Limiting magnitude (eye):} 7.1 -- 7.5

Airglow weakly visible near horizon; M33 easily seen with naked eye; highly structured Summer Milky Way; distinctly yellowish zodiacal light bright enough to cast shadows at dusk and dawn; clouds only visible as dark holes; surroundings still only barely visible silhouetted against the sky; many Messier globular clusters still distinct naked-eye objects.

\subsection{Rural sky}
\textbf{Level:} 3 \\
\textbf{Colour:} blue \\
\textbf{Limiting magnitude (eye):} 6.6 -- 7.0

Some light pollution evident at the horizon; clouds illuminated near horizon, dark overhead; Milky Way still appears complex; M15, M4, M5, M22 distinct naked-eye objects; M33 easily visible with averted vision; zodiacal light striking in spring and
autumn, color still visible; nearer surroundings vaguely visible.

\subsection{Rural/suburban transition}
\textbf{Level:} 4 \\
\textbf{Colour:} green/yellow \\
\textbf{Limiting magnitude (eye):} 6.1 -- 6.5

Light pollution domes visible in various directions over the horizon; zodiacal light is still visible, but not even halfway extending to the zenith at dusk or dawn; Milky Way above the horizon still impressive, but lacks most of the finer details; M33 a difficult averted vision object, only visible when higher than 55$\degree$; clouds illuminated in the directions of the light sources, but still dark overhead; surroundings clearly visible, even at a distance.

\subsection{Suburban sky}
\textbf{Level:} 5 \\
\textbf{Colour:} orange \\
\textbf{Limiting magnitude (eye):} 5.6 -- 6.0

Only hints of zodiacal light are seen on the best nights in autumn and spring; Milky Way is very weak or invisible near the horizon and looks washed out overhead; light sources visible in most, if not all, directions; clouds are noticeably brighter than the sky.

\subsection{Bright suburban sky}
\textbf{Level:} 6 \\
\textbf{Colour:} red \\
\textbf{Limiting magnitude (eye):} 5.1 -- 5.5

Zodiacal light is invisible; Milky Way only visible near the zenith; sky within 35$\degree$ from the horizon glows grayish white; clouds anywhere in the sky appear fairly bright; surroundings easily visible; M33 is impossible to see
without at least binoculars, M31 is modestly apparent to the unaided eye.

\subsection{Suburban/urban transition}
\textbf{Level:} 7 \\
\textbf{Colour:} red \\
\textbf{Limiting magnitude (eye):} 5.0 at best

Entire sky has a grayish-white hue; strong light sources evident in all directions; Milky Way invisible; M31 and M44 may be glimpsed with the naked eye, but are very indistinct; clouds are brightly lit; even in moderate-sized telescopes the brightest Messier objects are only ghosts of their true selves.

\subsection{City sky}
\textbf{Level:} 8 \\
\textbf{Colour:} white \\
\textbf{Limiting magnitude (eye):} 4.5 at best

Sky glows white or orange --- you can easily read; M31 and M44 are barely glimpsed by an experienced observer on good nights; even with telescope, only bright Messier objects can be detected; stars forming familiar constellation patterns may be weak or completely invisible.

\subsection{Inner City sky}
\textbf{Level:} 9 \\
\textbf{Colour:} white \\
\textbf{Limiting magnitude (eye):} 4.0 at best

Sky is brilliantly lit with many stars forming constellations invisible and many weaker
constellations invisible; aside from Pleiades, no Messier object is visible to the naked eye; only objects to provide fairly pleasant views are the Moon, the Planets and a few of the brightest star clusters.

\chapter{Taking Screenshots}\label{taking-screenshots}

You can save what is on the screen to a file by pressing
\textbf{Control-S}. Screenshots are taken in \emph{.png} format, and
have filenames something like this: \emph{stellarium-000.png},
\emph{stellariuim-001.png} (the number increments to prevent
over-writing existing files).

Stellarium creates screenshots in different directories depending in
your system type, see section
\href{Advanced_Use\#Files_and_Directories}{Files and Directories}.



%%% Local Variables: 
%%% mode: latex
%%% TeX-master: "guide"
%%% End: 

