\chapterimage{chapter-bg.png} % Chapter heading image

\chapter{Star Catalogue }\label{star-catalogue}

This chapter describes how Stellarium records it's star catalogues, and
the related file formats.

\section{Stellarium's Sky Model}\label{stellariums-sky-model}

\subsection{Zones}\label{zones}

The celestial sphere is split into zones, which correspond to the
triangular faces of a geodesic sphere. The number of zones (faces)
depends on the level of sub-division of this sphere. The lowest level,
0, is an icosahedron (20 faces), subsequent levels, L, of sub-division
give the number of zones, n as:

\begin{equation}
n=20 \cdot 4^L
\end{equation}

Stellarium uses levels 0 to 7 in the existing star catalogues. Star Data
Records contain the position of a star as an offset from the central
position of the zone in which that star is located, thus it is necessary
to determine the vector from the observer to the centre of a zone, and
add the star's offsets to find the absolute position of the star on the
celestial sphere.

This position for a star is expressed as a 3-dimensional vector which
points from the observer (at the centre of the geodesic sphere) to the
position of the star as observed on the celestial sphere.

\section{Star Catalogue File Format}\label{star-catalogue-file-format}

\subsection{General Description}\label{general-description}

Stellarium's star catalogue data is kept in the stars/default
sub-directory of the Installation Directory and/or User Directory (see
section \href{Advanced_Use\#Files_and_Directories}{Files and
Directories}).

The main catalogue data is split into several files:

\begin{itemize}
\item
  stars\_0\_0v0\_3.cat
\item
  stars\_1\_0v0\_3.cat
\item
  stars\_2\_0v0\_3.cat
\item
  stars\_3\_0v0\_2.cat
\item
  stars\_4\_1v0\_0.cat
\item
  stars\_5\_1v0\_0.cat
\item
  stars\_6\_2v0\_0.cat
\item
  stars\_7\_2v0\_0.cat
\item
  stars\_8\_2v0\_0.cat
\end{itemize}

There also exist some control and reference files:

\begin{itemize}
\item
  stars\_hip\_cids\_0v0\_0.cat
\item
  stars\_hip\_sp\_0v0\_0.cat
\item
  gcvs\_hip\_part.dat
\item
  stars.ini
\item
  name.fab
\end{itemize}

When Stellarium starts, it reads the stars.ini file, from which it
determines the names of the other files, which it then loads.

The \textbf{stars\_hip\_cids\_0v0\_0.cat and stars\_hip\_sp\_0v0\_0.cat}
files contain reference data for the main catalogue files. The
\textbf{gcvs\_hip\_part.dat} file contain data about variables stars
(see section \href{Variable_Stars}{Variable Stars}) which reference for
the main catalogue files too.

A given catalogue file models stars for one and only one level (i.e. for
a fixed number of zones), which is recorded in the header of the file.
Individual star records do not contain full positional coordinates,
instead they contain coordinates relative to the central position of the
zone they occupy. Thus, when parsing star catalogues, it is necessary to
know about the zone model to be able to extract positional data.

\begin{longtabu} to \textwidth {l X X X X X}
\toprule
\emph{File} & \emph{Data Type} & \emph{Data Record Size} &
\emph{Geodesic Level} & \emph{\#Records} & \emph{Notes}\tabularnewline
\midrule
stars\_0\_0v0\_3.cat & 0 & 28 bytes & 0 & 5,013 &
Hipparcos\tabularnewline
stars\_1\_0v0\_3.cat & 0 & 28 bytes & 1 & 21,999 &
Hipparcos\tabularnewline
stars\_2\_0v0\_3.cat & 0 & 28 bytes & 2 & 151,516 &
Hipparcos\tabularnewline
stars\_3\_1v0\_2.cat & 1 & 10 bytes & 3 & 434,064 & Tycho 2\tabularnewline
stars\_4\_1v0\_0.cat & 1 & 10 bytes & 4 & 1,725,497 &
Tycho 2\tabularnewline
stars\_5\_2v0\_0.cat & 2 & 8 bytes & 5 & 7,669,011 &
NOMAD\tabularnewline
stars\_6\_2v0\_0.cat & 2 & 8 bytes & 6 & 26,615,233 &
NOMAD\tabularnewline
stars\_7\_2v0\_0.cat & 2 & 8 bytes & 7 & 57,826,266 &
NOMAD\tabularnewline
stars\_8\_2v0\_0.cat & 2 & 8 bytes & 7 & 116,923,084 &
NOMAD\tabularnewline
\bottomrule
\end{longtabu}

For a given catalogue file, there may be one of three formats for the
actual star data. The variation comes from the source of the data - the
larger catalogues of fainter stars providing less data per star than the
brighter star catalogues. See tables \emph{Stellarium's star catalogue}
and for details.

\subsection{File Sections}\label{file-sections}

The catalogue files are split into three main sections as described in
table \emph{File sections}.

\begin{longtabu} to \textwidth {l l X}
\toprule
\emph{Section} & \emph{Offset} & \emph{Description}\tabularnewline
\midrule
File Header Record & 0 & Contains magic number, geodesic subdivision
level, and magnitude range\tabularnewline
\midrule
Zone Records & 32 & A list of how many records there are for each zone.
The length of the zones section depends on the level value from the
header\tabularnewline
\midrule
Star Data Records & 32+4\emph{n} & This section of the file contains
fixed-size star records, as described below. Records do not contain zone
information, which must be inferred by counting how many records have
been read so far and switching zones when enough have been read to fill
the number of stars for the zone, as specified in the zones section
above. The value of \emph{n} used in the offset description is the
number of zones, as described above.\tabularnewline
\bottomrule
\end{longtabu}

\subsection{Record Types}\label{record-types}

\subsubsection{File Header Record}\label{file-header-record}

The File Header Record describes file-wide settings. It also contains a
magic number which servers as a file type identifier. See table
\emph{Header Record}.

\begin{longtabu} to \textwidth {l l l l X}
\toprule
\emph{Name} & \emph{Offset} & \emph{Type} & \emph{Size} &
\emph{Description}\tabularnewline
\midrule
Magic & 0 & int & 4 & The magic number which identifies the file as a
star catalogue. 0xde0955a3\tabularnewline
\midrule
Data Type & 4 & int & 4 & This describes the type of the file, which
defines the size and structure of the Star Data record for the
file.\tabularnewline
\midrule
Major Version & 8 & int & 4 & The file format major version
number\tabularnewline
\midrule
Minor Version & 12 & int & 4 & The file format minor version
number\tabularnewline
\midrule
Level & 16 & int & 4 & Sets the level of sub-division of the geodesic
sphere used to create the zones. 0 means an icosahedron (20 sizes),
subsequent levels of sub-division lead to numbers of zones as described
in section \href{Star_Catalogue\#Zones}{Zones}\tabularnewline
\midrule
Magnitude Minimum & 20 & int & 4 & The low bound of the magnitude scale
for values in this file. Note that this is still an integer in
Stellarium's own internal representation\tabularnewline
\midrule
Magnitude Range & 24 & int & 4 & The range of magnitudes expressed in
this file\tabularnewline
\midrule
Magnitude Steps & 28 & int & 4 & The number of steps used to describes
values in the range\tabularnewline
\bottomrule
\end{longtabu}

\subsubsection{Zone Records}\label{zone-records}

The \emph{Zone Records} section of the file lists the number of star
records there are per zone. The number of zones is determined from the
level value in the File Header Record, as described in section
\href{Star_Catalogue\#Zones}{Zones}. The Zones section is simply a list
of integer values which describe the number of stars for each zone. The
total length of the Zones section depends on the number of zones. See
table \emph{Zones section}.

\begin{longtabu} to \textwidth {l l l l X}
\toprule
\emph{Name} & \emph{Offset} & \emph{Type} & \emph{Size} &
\emph{Description}\tabularnewline
\midrule
num stars in zone 0 & 0 & int & 4 & The number of records in this file
which are in zone 0\tabularnewline
\midrule
num stars in zone 1 & 4 & int & 4 & The number of records is this file
which are in zone 1\tabularnewline
... & & & &\tabularnewline
num stars in zone \emph{n} & 4\emph{n} & int & 4 & The number of records
is this file which are in zone \emph{n}\tabularnewline
\bottomrule
\end{longtabu}

\subsubsection{Star Data Records}\label{star-data-records}

After the Zones section, the actual star data starts. The star data
records themselves do not contain the zone in which the star belongs.
Instead, the zone is inferred from the position of the record in the
file. For example, if the Zone Records section of the file says that the
first 100 records are for zone 0, the next 80 for zone 1 and so on, it
is possible to infer the zone for a given record by counting how many
records have been read so far.

The actual record structure depends on the value of the Data Type, as
found in the File Header Record.

See tables \emph{Star Data Record Type 0}, \emph{Star Data Record Type
1} and \emph{Star Data Record Type 2} for record structure details.

It should be noted that although the positional data loses accuracy as
one progresses though the Star Record Types, this is compensated for by
the face that the number of zones is much higher for the files where the
smaller precision position fields are used, so the actual resolution on
the sky isn't significantly worse for the type 1 and 2 records in
practice.

\begin{longtabu} to \textwidth {l l l l X}
\toprule
\emph{Name} & \emph{Offset} & \emph{Type} & \emph{Size} &
\emph{Description}\tabularnewline
\midrule
hip & 0 & int & 3 & Hipparcos catalogue number\tabularnewline
\midrule
component\_ids & 3 & unsigned char & 1 & This is an index to an array of
catalogue number suffixes. The list is read from the
\textbf{stars\_hip\_component\_ids.cat} file. The value of this field
turns out to be the line number in the file - 1\tabularnewline
\midrule
x0 & 4 & int & 4 & This is the position of the star relative to the
central point in the star's zone, in axis 1\tabularnewline
\midrule
x1 & 8 & int & 4 & This is the position of the star relative to the
central point in the star's zone, in axis 2\tabularnewline
\midrule
b\_v & 9 & unsigned char & 1 & This is the magnitude level in B-V
colour. This value refers to one of 256 discrete steps in the magnitude
range for the file\tabularnewline
\midrule
mag & 10 & unsigned char & 1 & This is the magnitude level in the V-I
colour. This value refers to one of 256 discrete steps in the magnitude
range for the file\tabularnewline
\midrule
sp\_int & 11 & unsigned short int & 2 & This is the index in an array of
spectral type descriptions which is taken from the file
\textbf{stars\_hip\_sp.cat}, the index corresponds to the line number in
the file - 1\tabularnewline
\midrule
dx0 & 13 & int & 4 & This is the proper motion of the star in axis
1\tabularnewline
\midrule
dx1 & 17 & int & 4 & This is the proper motion of the star in axis
2\tabularnewline
\midrule
plx & 21 & int & 4 & This is the parallax of the star. To get the actual
value, divide by 10000.\tabularnewline
\bottomrule
\end{longtabu}

\begin{longtabu} to \textwidth {l l l l X}
\toprule
\emph{Name} & \emph{Offset} & \emph{Type} & \emph{Size} &
\emph{Description}\tabularnewline
\midrule
x0 & 0 & int & 20 bits & This is the position of the star relative to
the central point in the star's zone, in axis 1\tabularnewline
\midrule
x1 & 20 bits & int & 20 bits & This is the position of the star relative
to the central point in the star's zone, in axis 2\tabularnewline
\midrule
dx0 & 40 bits & int & 14 bits & This is the proper motion of the star in
axis 1\tabularnewline
\midrule
dx1 & 54 bits & int & 14 bits & This is the proper motion of the star in
axis 2\tabularnewline
\midrule
b\_v & 68 bits & unsigned int & 7 bits & This is the magnitude level in
B-V colour. This value refers to one of 256 discrete steps in the
magnitude range for the file\tabularnewline
\midrule
mag & 75 bits & unsigned int & 5 bits & This is the magnitude level in
the V-I colour. This value refers to one of 256 discrete steps in the
magnitude range for the file\tabularnewline
\bottomrule
\end{longtabu}

\begin{longtabu} to \textwidth {l l l l X}
\toprule
\emph{Name} & \emph{Offset} & \emph{Type} & \emph{Size} &
\emph{Description}\tabularnewline
\midrule
x0 & 0 & int & 18 bits & This is the position of the star relative to
the central point in the star's zone, in axis 1\tabularnewline
\midrule
x1 & 18 bits & int & 18 bits & This is the position of the star relative
to the central point in the star's zone, in axis 2\tabularnewline
\midrule
b\_v & 36 bits & unsigned int & 7 bits & This is the magnitude level in
B-V colour. This value refers to one of 256 discrete steps in the
magnitude range for the file\tabularnewline
\midrule
mag & 43 bits & unsigned int & 5 bits & This is the magnitude level in
the V-I colour. This value refers to one of 256 discrete steps in the
magnitude range for the file\tabularnewline
\bottomrule
\end{longtabu}


\section{Variable Stars}
Since version 0.12.2 Stellarium support the subset of variable stars
from \href{http://www.sai.msu.su/gcvs/gcvs/}{GCVS}, which have a HIP
identificators (i.e. stars from Hipparcos catalog). Stellarium's Catalog
of Variable Stars based on \textbf{General Catalogue of Variable Stars} --- Samus N.N., Durlevich O.V., Kazarovets E V., Kireeva N.N., Pastukhova E.N., Zharova A.V., et al., General Catalogue of Variable Stars (Samus+ 2007-2012).

\subsection{Variable Star Catalog File
Format}\label{variable-star-catalog-file-format}

\subsubsection{General Description}\label{general-description}

Stellarium's \href{Variable_Stars}{variable star} catalog data is kept
in the stars/default sub-directory of the Installation Directory and/or
User Directory (see section
\href{Advanced_Use\#Files_and_Directories}{Files and Directories}).

\subsubsection{File Format}\label{file-format}

The \textbf{gvs\_hip\_part.dat} file contains data about 6916
\href{Variable_Stars}{variable stars} which have HIP identificators and
stored in the plain text format with tab delimiter.

\begin{longtabu} to \textwidth {l|l|X}
\toprule
\emph{Name} & \emph{Type} & \emph{Description}\tabularnewline
\midrule
HIP & int & This is the HIP identificator for the star. Used for
referenced to the main star catalogue.\tabularnewline
\midrule
GCVS & string & This is the GCVS designation for the variable
star.\tabularnewline
\midrule
Type & string & Type of variability.\tabularnewline
\midrule
Max & float & Magnitude at maximum brightness.\tabularnewline
\midrule
MFlag & int & Magnitude flag code (0=No flag; 1="(" if max. magnitude is
an amplitude; 2="\textless{}" if max. magnitude is a bright limit;
3="\textgreater{}" is max. magnitude is a faint limit).\tabularnewline
\midrule
Min I & float & First minimum magnitude or amplitude.\tabularnewline
\midrule
Min II & float & Second minimum magnitude.\tabularnewline
\midrule
V & char & The photometric system for magnitudes.\tabularnewline
\midrule
Epoch & float & Epoch for maximum or minimum light (in Julian
days).\tabularnewline
\midrule
Period & float & Period for the variable star (in days).\tabularnewline
\midrule
M-m & float & Rising time or duration of the eclipse (in
percent).\tabularnewline
\midrule
Spectrum & string & Spectral class of the variable star.\tabularnewline
\bottomrule
\end{longtabu}

\subsection{GCVS Variability Types}\label{gcvs-variability-types}

An improved system of variability classification is used in the fourth
edition of the GCVS, based on recent developments in classification
principles and taking into account the suggestions of a number of
specialists. Variability types are grouped according to the major
astrophysical reasons for variability.

All of these classes include objects of a dissimilar nature that belong
to different types of light variability. On the other hand, an object
may be variable because of almost all of the possible reasons or because
of any combination of them. If a variable belongs to several types of
variability, the types are joined in the data field by a "+" sign, e.g.,
E+UG, UV+BY.

Despite considerable success in understanding stellar variability
processes, the classification adopted in the Catalogue is far from
perfect. This is especially the case for explosive, symbiotic and
novalike variables; X-ray sources; and peculiar objects.

You can read about latest different types of variability
\href{http://cdsarc.u-strasbg.fr/vizier/ftp/cats/B/gcvs/vartype.txt}{here}.

\subsubsection{Eruptive Variable Stars}\label{eruptive-variable-stars}

Eruptive variables are stars varying in brightness because of violent
processes and flares occurring in their chromospheres and coronae. The
light changes are usually accompanied by shell events or mass outflow in
the form of stellar winds of variable intensity and/or by interaction
with the surrounding interstellar medium. This class includes the
following types:

\begin{longtabu} to \textwidth {l|X}
\toprule
\emph{Type} & \emph{Description}\tabularnewline
\midrule
\textbf{FU} & Orion variables of the FU Orionis type. Characterized by
gradual increases in brightness by about 6 mag in several months,
followed by either almost complete constancy at maximum that is
sustained for long periods of time or slow decline by 1-2 mag. Spectral
types at maximum are in the range Ae(alpha) - Gpe(alpha). After an
outburst, a gradual development of an emission spectrum is observed and
the spectral type becomes later. These variables probably mark one of
the evolutionary stages of T Tauri-type Orion variables (INT), as
evidenced by an outburst of one member, V1057 Cyg, but its decline (2.5
mag in 11 years) commenced immediately after maximum brightness was
attained. All presently known FU Ori variables are coupled with
reflecting cometary nebulae.\tabularnewline
\midrule
\textbf{GCAS} & Eruptive irregular variables of the Gamma Cas type.
These are rapidly rotating B III-IVe stars with mass outflow from their
equatorial zones. The formation of equatorial rings or disks is often
accompanied by temporary fading. Light amplitudes may reach 1.5 mag in
V.\tabularnewline
\midrule
\textbf{I} & Poorly studied irregular variables with unknown features of
light variations and spectral types. This is a very inhomogeneous group
of objects.\tabularnewline
\midrule
\textbf{IA} & Poorly studied irregular variables of early (O-A) spectral
type.\tabularnewline
\midrule
\textbf{IB} & Poorly studied irregular variables of intermediate (F-G)
to late (K-M) spectral type.\tabularnewline
\midrule
\textbf{IN} & Orion variables. Irregular, eruptive variables connected
with bright or dark diffuse nebulae or observed in the regions of these
nebulae. Some of them may show cyclic light variations caused by axial
rotation. In the Spectrum-Luminosity diagram, they are found in the area
of the main sequence and subgiants. They are probably young objects
that, during the course of further evolution, will become light-constant
stars on the zero-age main sequence (ZAMS). The range of brightness
variations may reach several magnitudes. In the case of rapid light
variations having been observed (up to 1 mag in 1-10 days), the letter
``S'' is added to the symbol for the type (INS). This type may be
divided into the following subtypes:\tabularnewline
\midrule
\emph{INA} & Orion variables of early spectral types (B-A or Ae). They
are often characterized by occasional abrupt Algol-like fadings (T
Ori);\tabularnewline
\midrule
\emph{INB} & Orion variables of intermediate and late spectral types,
F-M or Fe-Me (BH Cep, AH Ori). F-type stars may show Algol-like fadings
similar to those of many INA stars; K-M stars may produce flares along
with irregular light variations;\tabularnewline
\midrule
\emph{INT,IT} & Orion variables of the T Tauri type. Stars are assigned
to this type on the basis of the following (purely spectroscopic)
criteria: spectral types are in the range Fe-Me. The spectra of most
typical stars resemble the spectrum of the solar chromosphere. The
feature specific to the type is the presence of the flourescent emission
lines Fe II 4046, 4132 A (anomalously intense in the spectra of these
stars), emission lines {[}Si II{]} and {[}O I{]}, as well as the
absorption line Li I 6707 A. These variables are usually observed only
in diffuse nebulae. If it is not apparent that the star is associated
with a nebula, the letter ``N'' in the symbol for the type may be
omitted, e.g., IT (RW AUR);\tabularnewline
\midrule
\emph{IN(YY)} & Some Orion variables (YY Ori) show the presence of
absorption components on the redward sides of emission lines, indicating
the infall of matter toward the stars' surfaces. In such cases, the
symbol for the type may be accompanied by the symbol
``YY''.\tabularnewline
\midrule
\textbf{IS} & Rapid irregular variables having no apparent connection
with diffuse nebulae and showing light changes of about 0.5 - 1.0 mag
within several hours or days. There is no strict boundary between rapid
irregular and Orion variables. If a rapid irregular star is observed in
the region of a diffuse nebula, it is considered an Orion variable and
designated by the symbol INS. To attribute a variable to the IS type, it
is necessary to take much care to be certain that its light changes are
really not periodic. Quite a number of the stars assigned to this type
in the third edition of the GCVS turned out to be eclipsing binary
systems, RR Lyrae variables, and even extragalactic BL Lac
objects.\tabularnewline
\midrule
\emph{ISA} & Rapid irregular variables of the early spectral types, B-A
or Ae;\tabularnewline
\midrule
\emph{ISB} & Rapid irregular variables of the intermediate and late
spectral types, F-M and Fe-Me.\tabularnewline
\midrule
\textbf{RCB} & Variables of the R Coronae Borealis type. These are
hydrogen-poor, carbon- and helium-rich, high-luminosity stars belonging
to the spectral types Bpe-R, which are simultaneously eruptive and
pulsating variables. They show slow nonperiodic fadings by 1-9 mag in V
lasting from a month or more to several hundred days. These changes are
superposed on cyclic pulsations with amplitudes up to several tenths of
a magnitude and periods in the range 30-100 days.\tabularnewline
\midrule
\textbf{RS} & Eruptive variables of the RS Canum Venaticorum type. This
type is ascribed to close binary systems with spectra showing Ca II H
and K in emission, their components having enhanced chromospheric
activity that causes quasi-periodic light variability. The period of
variation is close to the orbital one, and the variability amplitude is
usually as great as 0.2 mag in V (UX Ari). They are X-ray sources and
rotating variables. RS CVn itself is also an eclipsing system (see
below).\tabularnewline
\midrule
\textbf{SDOR} & Variables of the S Doradus type. These are eruptive,
high-luminosity Bpec-Fpec stars showing irregular (sometimes cyclic)
light changes with amplitudes in the range 1-7 mag in V. They belong to
the brightest blue stars of their parent galaxies. As a rule, these
stars are connected with diffuse nebulae and surrounded by expanding
envelopes (P Cyg, Eta Car).\tabularnewline
\midrule
\textbf{UV} & Eruptive variables of the UV Ceti type, these are K Ve-M
Ve stars sometimes displaying flare activity with amplitudes from
several tenths of a magnitude up to 6 mag in V. The amplitude is
considerably greater in the ultraviolet spectral region. Maximum light
is attained in several seconds or dozens of seconds after the beginning
of a flare; the star returns to its normal brightness in several minutes
or dozens of minutes.\tabularnewline
\midrule
\textbf{UVN} & Flaring Orion variables of spectral types Ke-Me. These
are phenomenologically almost identical to UV Cet variables observed in
the solar neighborhood. In addition to being related to nebulae, they
are normally characterized by being of earlier spectral type and greater
luminosity, with slower development of flares (V389 Ori). They are
possibly a specific subgroup of INB variables with irregular variations
superimposed by flares.\tabularnewline
\midrule
\textbf{WR} & Eruptive Wolf-Rayet variables. Stars with broad emission
features of He I and He II as well as C II-C IV, O II-O IV, and N III-N
V. They display irregular light changes with amplitudes up to 0.1 mag in
V, which are probably caused by physical processes, in particular, by
nonstable mass outflow from their atmospheres.\tabularnewline
\bottomrule
\end{longtabu}

\subsubsection{Pulsating Variable Stars}\label{pulsating-variable-stars}

Pulsating variables are stars showing periodic expansion and contraction
of their surface layers. The pulsations may be radial or nonradial. A
radially pulsating star remains spherical in shape, while in the case of
nonradial pulsations the star's shape periodically deviates from a
sphere, and even neighboring zones of its surface may have opposite
pulsation phases.

Depending on the period value, on the mass and evolutionary status of
the star, and on the scale of pulsational phenomena, the following types
of pulsating variables may be distinguished:

\begin{longtabu} to \textwidth {l|X}
\toprule
\emph{Type} & \emph{Description}\tabularnewline
\midrule
\textbf{ACYG} & Variables of the Alpha Cygni type, which are nonradially
pulsating supergiants of Bep-AepIa spectral types. The light changes
with amplitudes of the order of 0.1 mag often seem irregular, being
caused by the superposition of many oscillations with close periods.
Cycles from several days to several weeks are observed.\tabularnewline
\midrule
\textbf{BCEP} & Variables of the Beta Cephei type (Beta Cep, Beta CMa),
which are pulsating O8-B6 I-V stars with periods of light and
radial-velocity variations in the range of 0.1 - 0.6 days and light
amplitudes from 0.01 to 0.3 mag in V. The light curves are similar in
shape to average radial-velocity curves but lag in phase by a quarter of
the period, so that maximum brightness corresponds to maximum
contraction, i.e., to minimum stellar radius. The majority of these
stars probably show radial pulsations, but some (V469 Per) display
nonradial pulsations; multiperiodicity is characteristic of many of
these stars.\tabularnewline
\midrule
\textbf{BCEPS} & A short-period group of Beta Cep variables. The
spectral types are B2-B3 IV-V; periods and light amplitudes are in the
ranges 0.02 - 0.04 days and 0.015 - 0.025 days, respectively, i.e., an
order of magnitude smaller than the normally observed
ones.\tabularnewline
\midrule
\textbf{CEP} & Cepheids. Radially pulsating, high luminosity (classes
Ib-II) variables with periods in the range of 1-135 days and amplitudes
from several hundredths to 2 mag in V (in the B band, the amplitudes are
greater). Spectral type at maximum light is F; at minimum, the types are
G-K. The longer the period of light variation, the later is the spectral
type. The maximum of the surface-layer expansion velocity almost
coinciding with maximum light.\tabularnewline
\midrule
\textbf{CEP(B)} & Cepheids (TU Cas, V 367 Sct) displaying the presence
of two or more simultaneously operating pulsation modes (usually the
fundamental tone with the period P0 and the first overtone P1). The
periods P0 are in the range from 2 to 7 days, with the ratio P1/P0
approx. 0.71.\tabularnewline
\midrule
\textbf{CW} & Variables of the W Virginis type. These are pulsating
variables of the galactic spherical component (old disk) population with
periods of approximately 0.8 to 35 days and amplitudes from 0.3 to 1.2
mag in V. They obey a period-luminosity relation different from that for
Delta Cep variables (see DCEP). For an equal period value, the W Vir
variables are fainter than the Delta Cep stars by 0.7 - 2 mag. The light
curves of W Vir variables for some period intervals differ from those of
Delta Cep variables for corresponding periods either by amplitudes or by
the presence of humps on their descending branches, sometimes turning
into broad flat maxima. W Vir variables are present in globular clusters
and at high galactic latitudes. They may be separated into the following
subtypes:\tabularnewline
\midrule
\emph{CWA} & W Vir variables with periods longer than 8 days (W
Vir);\tabularnewline
\midrule
\emph{CWB} & W Vir variables with periods shorter than 8 days (BL
Her).\tabularnewline
\midrule
\textbf{DCEP} & These are the classical cepheids, or Delta Cep-type
variables. Comparatively young objects that have left the main sequence
and evolved into the instability strip of the Hertzsprung-Russell (H-R)
diagram, they obey the well-known Cepheid period-luminosity relation and
belong to the young disk population. DCEP stars are present in open
clusters. They display a certain relation between the shapes of their
light curves and their periods.\tabularnewline
\midrule
\textbf{DCEPS} & These are Delta Cep variables having light amplitudes
\textless{}0.5 mag in V (\textless{}0.7 mag in B) and almost symmetrical
light curves (M-m approx. 0.4 - 0.5 periods); as a rule, their periods
do not exceed 7 days. They are probably first-overtone pulsators and/or
are in the first transition across the instability strip after leaving
the main sequence (SU Cas).

Traditionally, both Delta Cep and W Vir stars are quite often called
Cepheids because it is often impossible to discriminate between them on
the basis of the light curves for periods in the range 3 - 10 days.
However, these are distinct groups of entirely different objects in
different evolutionary stages. One of the significant spectral
differences between W Vir stars and Cepheids is the presence, during a
certain phase interval, of hydrogen-line emission in the former and of
Ca II H and K emission in the latter.\tabularnewline
\midrule
\textbf{DSCT} & Variables of the Delta Scuti type. These are pulsating
variables of spectral types A0-F5 III-V displaying light amplitudes from
0.003 to 0.9 mag in V (usually several hundredths of a magnitude) and
periods from 0.01 to 0.2 days. The shapes of the light curves, periods,
and amplitudes usually vary greatly. Radial as well as nonradial
pulsations are observed. The variability of some members of this type
appears sporadically and sometimes completely ceases, this being a
consequence of strong amplitude modulation with the lower value of the
amplitude not exceeding 0.001 mag in some cases. The maximum of the
surface layer expansion does not lag behind the maximum light for more
than 0.1 periods. DSCT stars are representatives of the galactic disk
(flat component) and are phenomenologically close to the SX Phe
variables.\tabularnewline
\midrule
\textbf{DSCTC} & Low amplitude group of Delta Sct variables (light
amplitude \textless{}0.1 mag in V). The majority of this type's
representatives are stars of luminosity class V; objects of this subtype
generally are representative of the Delta Sct variables in open
clusters.\tabularnewline
\midrule
\textbf{L} & Slow irregular variables. The light variations of these
stars show no evidence of periodicity, or any periodicity present is
very poorly defined and appears only occasionally. Like for the type I,
stars are often attributed to this type because of being insufficiently
studied. Many type L variables are really semiregulars or belong to
other types.\tabularnewline
\midrule
\textbf{LB} & Slow irregular variables of late spectral types (K, M, C,
S); as a rule, they are giants (CO Cyg). This type is also ascribed, in
the GCVS, to slow red irregular variables in the case of unknown
spectral types and luminosities.\tabularnewline
\midrule
\textbf{LC} & Irregular variable supergiants of late spectral types
having amplitudes of about 1 mag in V (TZ Cas).\tabularnewline
\midrule
\textbf{M} & Mira (Omicron) Ceti-type variables. These are long-period
variable giants with characteristic late-type emission spectra (Me, Ce,
Se) and light amplitudes from 2.5 to 11 mag in V. Their periodicity is
well pronounced, and the periods lie in the range between 80 and 1000
days. Infrared amplitudes are usually less than in the visible and may
be \textless{}2.5 mag. For example, in the K band they usually do not
exceed 0.9 mag. If the amplitudes exceed 1 - 1.5 mag , but it is not
certain that the true light amplitude exceeds 2.5 mag, the symbol ``M''
is followed by a colon, or the star is attributed to the semiregular
class with a colon following the symbol for that type
(SR).\tabularnewline
\midrule
\textbf{PVTEL} & Variables of the PV Telescopii type. These are helium
supergiant Bp stars with weak hydrogen lines and enhanced lines of He
and C. They pulsate with periods of approximately 0.1 to 1 days, or vary
in brightness with an amplitude of 0.1 mag in V during a time interval
of about a year.\tabularnewline
\midrule
\textbf{RR} & Variables of the RR Lyrae type, which are
radially-pulsating giant A-F stars having amplitudes from 0.2 to 2 mag
in V. Cases of variable light-curve shapes as well as variable periods
are known. If these changes are periodic, they are called the ``Blazhko
effect.''

Traditionally, RR Lyrae stars are sometimes called short-period Cepheids
or cluster-type variables. The majority of these stars belong to the
spherical component of the Galaxy; they are present, sometimes in large
numbers, in some globular clusters, where they are known as pulsating
horizontal-branch stars. Like Cepheids, maximum expansion velocities of
surface layers for these stars practically coincide with maximum
light.\tabularnewline
\midrule
\emph{RR(B)} & RR Lyrae variables showing two simultaneously operating
pulsation modes, the fundamental tone with the period P0 and the first
overtone, P1 (AQ Leo). The ratio P1/P0 is approximately
0.745;\tabularnewline
\midrule
\emph{RRAB} & RR Lyrae variables with asymmetric light curves (steep
ascending branches), periods from 0.3 to 1.2 days, and amplitudes from
0.5 to 2 mag in V;\tabularnewline
\midrule
\emph{RRC} & RR Lyrae variables with nearly symmetric, sometimes
sinusoidal, light curves, periods from 0.2 to 0.5 days, and amplitudes
not greater than 0.8 mag in V (SX UMa).\tabularnewline
\midrule
\textbf{RV} & Variables of the RV Tauri type. These are radially
pulsating supergiants having spectral types F-G at maximum light and K-M
at minimum. The light curves are characterized by the presence of double
waves with alternating primary and secondary minima that can vary in
depth so that primary minima may become secondary and vice versa. The
complete light amplitude may reach 3-4 mag in V. Periods between two
adjacent primary minima (usually called formal periods) lie in the range
30-150 days (these are the periods appearing in the Catalogue). Two
subtypes, RVA and RVB, are recognized:\tabularnewline
\midrule
\emph{RVA} & RV Tauri variables that do not vary in mean magnitude (AC
Her);\tabularnewline
\midrule
\emph{RVB} & RV Tauri variables that periodically (with periods from 600
to 1500 days and amplitudes up to 2 mag in V) vary in mean magnitude (DF
Cyg, RV Tau).\tabularnewline
\midrule
\textbf{SR} & Semiregular variables, which are giants or supergiants of
intermediate and late spectral types showing noticeable periodicity in
their light changes, accompanied or sometimes interrupted by various
irregularities. Periods lie in the range from 20 to \textgreater{}2000
days, while the shapes of the light curves are rather different and
variable, and the amplitudes may be from several hundredths to several
magnitudes (usually 1-2 mag in V).\tabularnewline
\midrule
\emph{SRA} & Semiregular late-type (M, C, S or Me, Ce, Se) giants
displaying persistent periodicity and usually small (\textless{}2.5 mag
in V) light amplitudes (Z Aqr). Amplitudes and light-curve shapes
generally vary and periods are in the range of 35-1200 days. Many of
these stars differ from Miras only by showing smaller light
amplitudes;\tabularnewline
\midrule
\emph{SRB} & Semiregular late-type (M, C, S or Me, Ce, Se) giants with
poorly defined periodicity (mean cycles in the range of 20 to 2300 days)
or with alternating intervals of periodic and slow irregular changes,
and even with light constancy intervals (RR CrB, AF Cyg). Every star of
this type may usually be assigned a certain mean period (cycle), which
is the value given in the Catalogue. In a number of cases, the
simultaneous presence of two or more periods of light variation is
observed;\tabularnewline
\midrule
\emph{SRC} & Semiregular late-type (M, C, S or Me, Ce, Se) supergiants
(Mu Cep) with amplitudes of about 1 mag and periods of light variation
from 30 days to several thousand days;\tabularnewline
\midrule
\emph{SRD} & Semiregular variable giants and supergiants of F, G, or K
spectral types, sometimes with emission lines in their spectra.
Amplitudes of light variation are in the range from 0.1 to 4 mag, and
the range of periods is from 30 to 1100 days (SX Her, SV
UMa).\tabularnewline
\midrule
\textbf{SXPHE} & Phenomenologically, these resemble DSCT (Delta Sct)
variables and are pulsating subdwarfs of the spherical component, or old
disk galactic population, with spectral types in the range A2-F5. They
may show several simultaneous periods of oscillation, generally in the
range 0.04-0.08 days, with variable-amplitude light changes that may
reach 0.7 mag in V. These stars are present in globular
clusters.\tabularnewline
\midrule
\textbf{ZZ} & ZZ Ceti variables. These are nonradially pulsating white
dwarfs that change their brightnesses with periods from 30 s to 25 min
and amplitudes from 0.001 to 0.2 mag in V. They usually show several
close period values. Flares of 1 mag are sometimes observed; however,
these may be explained by the presence of close UV Ceti companions.

These variables are divided into the following subtypes:\tabularnewline
\midrule
\emph{ZZA} & ZZ Cet-type variables of DA spectral type (ZZ Cet) having
only hydrogen absorption lines in their spectra;\tabularnewline
\midrule
\emph{ZZB} & ZZ Cet-type variables of DB spectral type having only
helium absorption lines in their spectra.\tabularnewline
\bottomrule
\end{longtabu}

\subsubsection{Rotating Variable Stars}\label{rotating-variable-stars}

Variable stars with nonuniform surface brightness and/or ellipsoidal
shapes, whose variability is caused by axial rotation with respect to
the observer. The nonuniformity of surface brightness distributions may
be caused by the presence of spots or by some thermal or chemical
inhomogeneity of the atmosphere caused by a magnetic field whose axis is
not coincident with the rotation axis. These stars are subdivided into
the following types:

\begin{longtabu} to \textwidth {l|X}
\toprule
\emph{Type} & \emph{Description}\tabularnewline
\midrule
\textbf{ACV} & Alpha2 Canum Venaticorum variables. These are
main-sequence stars with spectral types B8p-A7p and displaying strong
magnetic fields. Spectra show abnormally strong lines of Si, Sr, Cr, and
rare earths whose intensities vary with rotation. They exhibit magnetic
field and brightness changes (periods of 0.5-160 days or more). The
amplitudes of the brightness changes are usually withine 0.01-0.1 mag in
V.\tabularnewline
\midrule
\textbf{ACVO} & Rapidly oscillating Alpha2 CVn variables. These are
nonradially pulsating, rotating magnetic variables of Ap spectral type
(DO Eri). Pulsation periods are in the range of 6-12 mmag (0.004-0.01
days), while amplitudes of light variation caused by the pulsation are
about 0.01 mag in V. The pulsational variations are superposed on those
caused by rotation.\tabularnewline
\midrule
\textbf{BY} & BY Draconis-type variables, which are emission-line dwarfs
of dKe-dMe spectral type showing quasiperiodic light changes with
periods from a fraction of a day to 120 days and amplitudes from several
hundredths to 0.5 mag in V. The light variability is caused by axial
rotation of a star with a variable degree of nonuniformity of the
surface brightness (spots) and chromospheric activity. Some of these
stars also show flares similar to those of UV Cet stars, and in those
cases they also belong to the latter type and are simultaneously
considered eruptive variables.\tabularnewline
\midrule
\textbf{ELL} & Rotating ellipsoidal variables (b Per, Alpha Vir). These
are close binary systems with ellipsoidal components, which change
combined brightnesses with periods equal to those of orbital motion
because of changes in emitting areas toward an observer, but showing no
eclipses. Light amplitudes do not exceed 0.1 mag in V.\tabularnewline
\midrule
\textbf{FKCOM} & FK Comae Berenices-type variables. These are rapidly
rotating giants with nonuniform surface brightnesses, which have G-K
spectral types with broad H and K Ca II emission and sometimes Halpha.
They may also be spectroscopic binary systems. Periods of light
variation (up to several days) are equal to rotational periods, and
amplitudes are several tenths of a magnitude. It is not excluded that
these objects are the product of further evolution of EW (W UMa) close
binary systems (see below).\tabularnewline
\midrule
\textbf{PSR} & Optically variable pulsars (CM Tau), which are rapidly
rotating neutron stars with strong magnetic fields, radiating in the
radio, optical, and X-ray regions. Pulsars emit narrow beams of
radiation, and periods of their light changes coincide with rotational
periods (from 0.004 to 4 s), while amplitudes of the light pulses reach
0.8 mag.\tabularnewline
\midrule
\textbf{SXARI} & SX Arietis-type variables. These are main-sequence
B0p-B9p stars with variable-intensity He I and Si III lines and magnetic
fields. They are sometimes called helium variables. Periods of light and
magnetic field changes (about 1 day) coincide with rotational periods,
while amplitudes are approximately 0.1 mag in V. These stars are
high-temperature analogs of the ACV variables.\tabularnewline
\bottomrule
\end{longtabu}

\subsubsection{Cataclysmic (Explosive and Novalike)
Variables}\label{cataclysmic-explosive-and-novalike-variables}

These are variable stars showing outbursts caused by thermonuclear burst
processes in their surface layers (novae) or deep in their interiors
(supernovae). We use the term ``novalike'' for variables that show
novalike outbursts caused by rapid energy release in the surrounding
space (UG-type stars - see below) and also for objects not displaying
outbursts but resembling explosive variables at minimum light by their
spectral (or other) characteristics. The majority of explosive and
novalike variables are close binary systems, their components having
strong mutual influence on the evolution of each star. It is often
observed that the hot dwarf component of the system is surrounded by an
accretion disk formed by matter lost by the other, cooler, and more
extended component. This category is subdivided into the following
types:

\begin{longtabu} to \textwidth {l|X}
\toprule
\emph{Type} & \emph{Description}\tabularnewline
\midrule
\textbf{N} & Novae. Close binary systems with orbital periods from 0.05
to 230 days. One of the components of these systems is a hot dwarf star
that suddenly, during a time interval from one to several dozen or
several hundred days, increases its brightness by 7-19 mag in V, then
returns gradually to its former brightness over several months, years,
or decades. Small changes at minimum light may be present. Cool
components may be giants, subgiants, or dwarfs of K-M type. The spectra
of novae near maximum light resemble A-F absorption spectra of luminous
stars at first. Then broad emission lines (bands) of hydrogen, helium,
and other elements with absorption components indicating the presence of
a rapidly expanding envelope appear in the spectrum. As the light
decreases, the composite spectrum begins to show forbidden lines
characteristic of the spectra of gas nebulae excited by hot stars. At
minimum light, the spectra of novae are generally continuous or resemble
the spectra of Wolf-Rayet stars. Only spectra of the most massive
systems show traces of cool components.

Some novae reveal pulsations of hot components with periods of
approximately 100 s and amplitudes of about 0.05 mag in V after an
outburst. Some novae eventually turn out to be eclipsing systems.
According to the features of their light variations, novae are
subdivided into fast (NA), slow (NB), very slow (NC), and recurrent (NR)
categories.\tabularnewline
\midrule
\emph{NA} & Fast novae displaying rapid light increases and then, having
achieved maximum light, fading by 3 mag in 100 or fewer days (GK
Per);\tabularnewline
\midrule
\emph{NB} & Slow novae that fade after maximum light by 3 mag in
\textgreater{}= 150 days (RR Pic). Here the presence of the well-known
``dip'' in the light curves of novae similar to T Aur and DQ Her is not
taken into account: The rate of fading is estimated on the basis of a
smooth curve, its parts before and after the ``dip'' being a direct
continuation of one another;\tabularnewline
\midrule
\emph{NC} & Novae with a very slow development and remaining at maximum
light for more than a decade, then fading very slowly. Before an
outburst these objects may show long-period light changes with
amplitudes of 1-2 mag in V (RR Tel); cool components of these systems
are probably giants or supergiants, sometimes semiregular variables, and
even Mira variables. Outburst amplitudes may reach 10 mag. High
excitation emission spectra resemble those of planetary nebulae,
Wolf-Rayet stars, and symbiotic variables. The possibility that these
objects are planetary nebulae in the process of formation is not
excluded;\tabularnewline
\midrule
\emph{NL} & Novalike variables, which are insufficiently studied objects
resembling novae by the characteristics of their light changes or by
spectral features. This type includes, in addition to variables showing
novalike outbursts, objects with no bursts ever observed; the spectra of
novalike variables resemble those of old novae, and small light changes
resemble those typical for old novae at minimum light. However, quite
often a detailed investigation makes it possible to reclassify some
representatives of this highly inhomogeneous group of objects into other
types;\tabularnewline
\midrule
\emph{NR} & Recurrent novae, which differ from typical novae by the fact
that two or more outbursts (instead of a single one) separated by 10-80
years have been observed (T CrB).\tabularnewline
\midrule
\textbf{SN} & Supernovae (B Cas, CM Tau). Stars that increase, as a
result of an outburst, their brightnesses by 20 mag and more, then fade
slowly. The spectrum during an outburst is characterized by the presence
of very broad emission bands, their widths being several times greater
than those of the bright bands observed in the spectra of novae. The
expansion velocities of SN envelopes are in the thousands of km/s. The
structure of a star after outburst alters completely. An expanding
emission nebula results and a (not always observable) pulsar remains at
the position of the original star. According to the light curve shape
and the spectral features, supernovae are subdivided into types I and
II.\tabularnewline
\midrule
\emph{SNI} & Type I supernovae. Absorption lines of Ca II, Si, etc., but
no hydrogen lines are present in the spectra. The expanding envelope
almost lacks hydrogen. During 20-30 days following maximum light, the
brightness decreases by approximately 0.1 mag per day, then the rate of
fading slows and reaches a constant value of 0.014/day;\tabularnewline
\midrule
\emph{SNII} & Type II supernovae. Lines of hydrogen and other elements
are apparent in their spectra. The expanding envelope consists mainly of
H and He. Light curves show greater diversity than those of type I
supernovae. Usually after 40-100 days since maximum light, the rate of
fading is 0.1 mag per day.\tabularnewline
\midrule
\textbf{UG} & U Geminorum-type variables, quite often called dwarf
novae. They are close binary systems consisting of a dwarf or subgiant
K-M star that fills the volume of its inner Roche lobe and a white dwarf
surrounded by an accretion disk. Orbital periods are in the range
0.05-0.5 days. Usually only small, in some cases rapid, light
fluctuations are observed, but from time to time the brightness of a
system increases rapidly by several magnitudes and, after an interval of
from several days to a month or more, returns to the original state.
Intervals between two consecutive outbursts for a given star may vary
greatly, but every star is characterized by a certain mean value of
these intervals, i.e., a mean cycle that corresponds to the mean light
amplitude. The longer the cycle, the greater the amplitude. These
systems are frequently sources of X-ray emission. The spectrum of a
system at minimum is continuous, with broad H and He emission lines. At
maximum these lines almost disappear or become shallow absorption lines.
Some of these systems are eclipsing, possibly indicating that the
primary minimum is caused by the eclipse of a hot spot that originates
in the accretion disk from the infall of a gaseous stream from the K-M
star. According to the characteristics of the light changes, U Gem
variables may be subdivided into three types: SS Cyg, SU UMa, and Z
Cam.\tabularnewline
\midrule
\emph{UGSS} & SS Cygni-type variables (SS Cyg, U Gem). They increase in
brightness by 2-6 mag in V in 1-2 days and in several subsequent days
return to their original brightnesses. The values of the cycle are in
the range 10 days to several thousand;\tabularnewline
\midrule
\emph{UGSU} & SU Ursae Majoris-type variables. These are characterized
by the presence of two types of outbursts called ``normal'' and
``supermaxima''. Normal, short outbursts are similar to those of UGSS
stars, while supermaxima are brighter by 2 mag, are more than five times
longer (wider), and occur several times less frequently. During
supermaxima the light curves show superposed periodic oscillations
(superhumps), their periods being close to the orbital ones and
amplitudes being about 0.2-0.3 mag in V. Orbital periods are shorter
than 0.1 days; companions are of dM spectral type;\tabularnewline
\midrule
\emph{UGZ} & Z Camelopardalis-type stars. These also show cyclic
outbursts, differing from UGSS variables by the fact that sometimes
after an outburst they do not return to the original brightness, but
during several cycles retain a magnitude between maximum and minimum.
The values of cycles are from 10 to 40 days, while light amplitudes are
from 2 to 5 mag in V.\tabularnewline
\midrule
\textbf{ZAND} & Symbiotic variables of the Z Andromedae type. They are
close binaries consisting of a hot star, a star of late type, and an
extended envelope excited by the hot star's radiation. The combined
brightness displays irregular variations with amplitudes up to 4 mag in
V. A very inhomogeneous group of objects.\tabularnewline
\bottomrule
\end{longtabu}

\subsubsection{Close Binary Eclipsing
Systems}\label{close-binary-eclipsing-systems}

We adopt a triple system of classifying eclipsing binary systems:
according to the shape of the combined light curve, as well as to
physical and evolutionary characteristics of their components. The
classification based on light curves is simple, traditional, and suits
the observers; the second and third classification methods take into
account positions of the binary-system components in the (MV ,B-V)
diagram and the degree of inner Roche lobe filling. Estimates are made
by applying the simple criteria proposed by Svechnikov and Istomin
(1979). The symbols for the types of eclipsing binary systems that we
use are given below.

\subsubsection{Classification based on the shape of the light
curve}\label{classification-based-on-the-shape-of-the-light-curve}

\begin{longtabu} to \textwidth {l|X}
\toprule
\emph{Type} & \emph{Description}\tabularnewline
\midrule
\textbf{E} & Eclipsing binary systems. These are binary systems with
orbital planes so close to the observer's line of sight (the inclination
i of the orbital plane to the plane orthogonal to the line of sight is
close to 90 deg) that the components periodically eclipse each other.
Consequently, the observer finds changes of the apparent combined
brightness of the system with the period coincident with that of the
components' orbital motion.\tabularnewline
\midrule
\textbf{EA} & Algol (Beta Persei)-type eclipsing systems. Binaries with
spherical or slightly ellipsoidal components. It is possible to specify,
for their light curves, the moments of the beginning and end of the
eclipses. Between eclipses the light remains almost constant or varies
insignificantly because of reflection effects, slight ellipsoidality of
components, or physical variations. Secondary minima may be absent. An
extremely wide range of periods is observed, from 0.2 to \textgreater{}=
10000 days. Light amplitudes are also quite different and may reach
several magnitudes.\tabularnewline
\midrule
\textbf{EB} & Beta Lyrae-type eclipsing systems. These are eclipsing
systems having ellipsoidal components and light curves for which it is
impossible to specify the exact times of onset and end of eclipses
because of a continuous change of a system's apparent combined
brightness between eclipses; secondary minimum is observed in all cases,
its depth usually being considerably smaller than that of the primary
minimum; periods are mainly longer than 1 day. The components generally
belong to early spectral types (B-A). Light amplitudes are usually
\textless{}2 mag in V.\tabularnewline
\midrule
\textbf{EW} & W Ursae Majoris-type eclipsing variables. These are
eclipsers with periods shorter than 1 days, consisting of ellipsoidal
components almost in contact and having light curves for which it is
impossible to specify the exact times of onset and end of eclipses. The
depths of the primary and secondary minima are almost equal or differ
insignificantly. Light amplitudes are usually \textless{}0.8 mag in V.
The components generally belong to spectral types F-G and
later.\tabularnewline
\bottomrule
\end{longtabu}

\subsubsection{Classification according to the components' physical
characteristics}\label{classification-according-to-the-components-physical-characteristics}

\begin{longtabu} to \textwidth {l|X}
\toprule
\emph{Type} & \emph{Description}\tabularnewline
\midrule
\textbf{GS} & Systems with one or both giant and supergiant components;
one of the components may be a main sequence star.\tabularnewline
\midrule
\textbf{PN} & Systems having, among their components, nuclei of
planetary nebulae (UU Sge).\tabularnewline
\midrule
\textbf{RS} & RS Canum Venaticorum-type systems. A significant property
of these systems is the presence in their spectra of strong Ca II H and
K emission lines of variable intensity, indicating increased
chromospheric activity of the solar type. These systems are also
characterized by the presence of radio and X-ray emission. Some have
light curves that exhibit quasi sine waves outside eclipses, with
amplitudes and positions changing slowly with time. The presence of this
wave (often called a distortion wave) is explained by differential
rotation of the star, its surface being covered with groups of spots;
the period of the rotation of a spot group is usually close to the
period of orbital motion (period of eclipses) but still differs from it,
which is the reason for the slow change (migration) of the phases of the
distortion wave minimum and maximum in the mean light curve. The
variability of the wave's amplitude (which may be up to 0.2 mag in V) is
explained by the existence of a long-period stellar activity cycle
similar to the 11-year solar activity cycle, during which the number and
total area of spots on the star's surface vary.\tabularnewline
\midrule
\textbf{WD} & Systems with white-dwarf components.\tabularnewline
\midrule
\textbf{WR} & Systems having Wolf-Rayet stars among their components (V
444 Cyg).\tabularnewline
\bottomrule
\end{longtabu}

\subsubsection{Classification based on the degree of filling of inner Roche
lobes}\label{classification-based-on-the-degree-of-filling-of-inner-roche-lobes}

\begin{longtabu} to \textwidth {l|X}
\toprule
\emph{Type} & \emph{Description}\tabularnewline
\midrule
\textbf{AR} & Detached systems of the AR Lacertae type. Both components
are subgiants not filling their inner equipotential
surfaces.\tabularnewline
\midrule
\textbf{D} & Detached systems, with components not filling their inner
Roche lobes.\tabularnewline
\midrule
\textbf{DM} & Detached main-sequence systems. Both components are
main-sequence stars and do not fill their inner Roche
lobes.\tabularnewline
\midrule
\textbf{DS} & Detached systems with a subgiant. The subgiant also does
not fill its inner critical surface.\tabularnewline
\midrule
\textbf{DW} & Systems similar to W UMa systems in physical properties
(KW, see below), but not in contact.\tabularnewline
\midrule
\textbf{K} & Contact systems, both components filling their inner
critical surfaces.\tabularnewline
\midrule
\textbf{KE} & Contact systems of early (O-A) spectral type, both
components being close in size to their inner critical
surfaces.\tabularnewline
\midrule
\textbf{KW} & Contact systems of the W UMa type, with ellipsoidal
components of F0-K spectral type. Primary components are main-sequence
stars and secondaries lie below and to the left of the main sequence in
the (MV,B-V) diagram.\tabularnewline
\midrule
\textbf{SD} & Semidetached systems in which the surface of the less
massive component is close to its inner Roche lobe.\tabularnewline
\bottomrule
\end{longtabu}

The combination of the above three classification systems for eclipsers
results in the assignment of multiple classifications for object types.
These are separated by a solidus ("/") in the data field. Examples are:
E/DM, EA/DS/RS, EB/WR, EW/KW, etc.

\subsubsection{Optically Variable Close Binary Sources of Strong, Variable
X-ray Radiation (X-ray
Sources)}\label{optically-variable-close-binary-sources-of-strong-variable-x-ray-radiation-x-ray-sources}

\begin{longtabu} to \textwidth {l|X}
\toprule
\emph{Type} & \emph{Description}\tabularnewline
\midrule
\textbf{X} & Close binary systems that are sources of strong, variable
X-ray emission and which do not belong to or are not yet attributed to
any of the above types of variable stars. One of the components of the
system is a hot compact object (white dwarf, neutron star, or possibly a
black hole). X-ray emission originates from the infall of matter onto
the compact object or onto an accretion disk surrounding the compact
object. In turn, the X-ray emission is incident upon the atmosphere of
the cooler companion of the compact object and is reradiated in the form
of optical high-temperature radiation (reflection effect), thus making
that area of the cooler companion's surface an earlier spectral type.
These effects lead to quite a peculiar complex character of optical
variability in such systems. These objects may be subdivided into the
following types:\tabularnewline
\midrule
\emph{XB} & X-ray bursters. Close binary systems showing X-ray and
optical bursts, their duration being from several seconds to ten
minutes, with amplitudes of about 0.1 mag in V (V 801 Ara, V 926
Sco);\tabularnewline
\midrule
\emph{XF} & Fluctuating X-ray systems showing rapid variations of X-ray
(Cygnus X-1 = V1357 Cyg) and optical (V821 Ara) radiation on time scales
of dozens of milliseconds;\tabularnewline
\midrule
\emph{XI} & X-ray irregulars. Close binary systems consisting of a hot
compact object surrounded by an accretion disk and a dA - dM-type dwarf.
These display irregular light changes on time scales of minutes and
hours, and amplitudes of about 1 mag in V. Superposition of a periodic
variation because of orbital motion is possible (V818
Sco);\tabularnewline
\midrule
\emph{XJ} & X-ray binaries characterized by the presence of relativistic
jets evident at X-ray and radio wavelengths, as well as in the optical
spectrum in the form of emission components showing periodic
displacements with relativistic velocities (V1343 Aql);\tabularnewline
\midrule
\emph{XND} & X-ray, novalike (transient) systems containing, along with
a hot compact object, a dwarf or subgiant of G-M spectral type. These
systems occasionally rapidly increase in brightness by 4-9 mag in V, in
the visible simultaneously with the X-ray range, with no envelope
ejected. The duration of the outburst may be up to several months (V616
Mon);\tabularnewline
\midrule
\emph{XNG} & X-ray, novalike (transient) systems with an early-type
supergiant or giant primary component and a hot compact object as a
companion. Following the main component's outburst, the material ejected
by it falls onto the compact object and causes, with a significant
delay, the appearance of X rays. The amplitudes are about 1-2 mag in V
(V725 Tau);\tabularnewline
\midrule
\emph{XP} & X-ray pulsar systems. The primary component is usually an
ellipsoidal early-type supergiant. The reflection effect is very small
and light variability is mainly caused by the ellipsoidal primary
component's rotation. Periods of light changes are between 1 and 10
days; the period of the pulsar in the system is from 1 s to 100 min.
Light amplitudes usually do not exceed several tenths of a magnitude
(Vela X-1 = GP Vel);\tabularnewline
\midrule
\emph{XPR} & X-ray pulsar systems featuring the presence of the
reflection effect. They consist of a dB-dF-type primary and an X-ray
pulsar, which may also be an optical pulsar. The mean light of the
system is brightest when the primary component is irradiated by X rays;
it is faintest during a low state of the X-ray source. The total light
amplitude may reach 2-3 mag in V (HZ Her);\tabularnewline
\midrule
\emph{XPRM,XM} & X-ray systems consisting of a late-type dwarf (dK-dM)
and a pulsar with a strong magnetic field. Matter accretion on the
compact object's magnetic poles is accompanied by the appearance of
variable linear and circular polarization; hence, these systems are
sometimes known as ``polars''. The amplitudes of the light changes are
usually about 1 mag in V but, provided that the primary component is
irradiated by X rays, the mean brightness of a system may increase by 3
mag in V. The total light amplitude may reach 4-5 mag in V (AM Her, AN
UMa).\tabularnewline
\bottomrule
\end{longtabu}

If the beam of X-ray emission originating at the magnetic poles of the
rotating hot compact object does not pass through the observer's
position and the system is not observed as a pulsar, the letter ``P'' in
the above symbols for X-ray- system types is not used. If an X-ray
system is also an eclipsing or an ellipsoidal variable, the X-ray symbol
is preceded by ``E'' or ``ELL'' joined with the X-ray symbol by a "+"
sign (e.g., E+X, ELL+X).

\subsubsection{Other Symbols}\label{other-symbols}

In addition to the variable-star types described above, certain other
symbols that need to be explained will be found in the Type data field:

\begin{longtabu} to \textwidth {l|X}
\toprule
\emph{Type} & \emph{Description}\tabularnewline
\midrule
\textbf{BLLAC} & Extragalactic BL Lacertae-type objects. These are
compact quasistellar objects showing almost continuous spectra with weak
emission and absorption lines and relatively rapid irregular light
changes with amplitudes up to 3 mag in V or more. Sources of strong
X-ray radiation and radio waves, their emission displays strong and
variable linear polarization in the visible and infrared spectral
regions. Some objects of this type, considered erroneously to be
variable stars and designated in the GCVS system, will probably
sometimes be included in the main table of the Catalogue in the future,
too.\tabularnewline
\midrule
\textbf{CST} & Nonvariable stars, formerly suspected to be variable and
hastily designated. Further observations have not confirmed their
variability.\tabularnewline
\midrule
\textbf{GAL} & Optically variable quasistellar extragalactic objects
(active galactic nuclei {[}AGNs{]}) considered to be variable stars by
mistake.\tabularnewline
\midrule
\textbf{L:} & Unstudied variable stars with slow light
changes.\tabularnewline
\midrule
\textbf{QSO} & Optically variable quasistellar extragalactic sources
(quasars) that earlier were erroneously considered to be variable
stars.\tabularnewline
\midrule
\textbf{S} & Unstudied variable stars with rapid light
changes.\tabularnewline
\midrule
\textbf{*} & Unique variable stars outside the range of the
classifications described above. These probably represent either short
stages of transition from one variability type to another or the
earliest and latest evolutionary stages of these types, or they are
insufficiently studied members of future new types of
variables.\tabularnewline
\midrule
\textbf{+} & If a variable star belongs to several types of light
variability simultaneously, the types are joined in the Type field by a
"+" sign (e.g., E+UG, UV+BY).\tabularnewline
\midrule
\textbf{:} & Uncertainty flag on Type of Variability\tabularnewline
\bottomrule
\end{longtabu}

\subsubsection{The new variability types}\label{the-new-variability-types}

The new variability types have been added in the Name-Lists 67- 77 and
in the GCVS vol.V.

\begin{longtabu} to \textwidth {l|X}
\toprule
\emph{Type} & \emph{Description}\tabularnewline
\midrule
\textbf{ZZO} & ZZ Cet type variables of the DO spectral type showing
HeII and and CIV absorpion lines in their spectra.\tabularnewline
\textbf{AM} & AM Her type variables; close binary systems consisting of
a dK-dM type dwarf and of a compact object with strong magnetic field,
characterized by variable linear and circular polarization of light. The
total range of light variations may reach 4-5 mag V.\tabularnewline
\textbf{R} & Close binary systems characterized by the presence of
strong reflection (re-radiation) of the light of the hot star
illuminating the surface of the cooler companion. Light curves are
sinusoidal with the period equal to Porb, maximum brightness coinciding
with the passage of the hot star in front of the companion. The eclipse
may be absent. The range of light variation is about 0.5-1.0mag V (KV
Vel).\tabularnewline
\midrule
\textbf{BE} & It becomes more and more clear that, although the majority
of Be stars are photometrically variable, not all of them could be
properly called GCAS variables. Quite a number of them show small-scale
variations not necessarily related to shell events; in some cases the
variations are quasi-periodic. By now we are not able to present an
elaborated system of classification for Be variables, but we adopt a
decision that in the cases when a Be variable cannot be readily
described as a GCAS star we give simply BE for the type of
variability.\tabularnewline
\midrule
\textbf{EP} & Stars showing eclipses by their planets. Prototype: V0376
Peg.\tabularnewline
\midrule
\textbf{SRS} & Semiregular pulsating red giants with short period
(several days to a month), probably high-overtone pulsators. Prototype:
AU Ari.\tabularnewline
\midrule
\textbf{GDOR} & Gamma Doradus stars. Early type F dwarfs showing
(multiple) periods from several tenths of a day to slightly in excess of
one day. Amplitudes usually do not exceed 0.1 mag. Presumably low degree
g-mode non-radial pulsators. Prototype: gamma Dor.\tabularnewline
\textbf{RPHS} & Very rapidly pulsating hot (subdwarf B) stars. Typical
periods are hundreds of seconds, amplitudes are within several
hundredths of a magnitude. Prototype: V361 Hya = EC
14026-2647.\tabularnewline
\midrule
\textbf{LPB} & The comparatively long-period pulsating B stars (periods
exceeding (LBV) one day).\tabularnewline
\midrule
\textbf{BLBOO} & The so-called ``anomalous Cepheids'', i.e. stars with
periods characteristic of comparatively long-period RRAB variables, but
considerably brighter by luminosity (BL Boo = NGC 5466
V19).\tabularnewline
\bottomrule
\end{longtabu}

