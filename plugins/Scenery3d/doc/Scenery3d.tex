\documentclass[a4paper]{article}
\usepackage[latin1]{inputenc}
\usepackage[T1]{fontenc}
\usepackage{a4wide}
\usepackage{hyperref}
\newcommand{\filename}[1]{\texttt{#1}}
\newcommand{\cmd}[1]{\texttt{#1}}

\title{Scenery3d - Walkable 3D Models in Stellarium}
\author{Georg Zotti\thanks{\url{Georg.Zotti@univie.ac.at}, \url{http://astrosim.univie.ac.at}}}
\begin{document}
\maketitle

\section{Introduction}
\label{sec:Introduction}


Have you ever wished to be able to walk through Stonehenge or other
ancient building structures described as being constructed with astronomical
orientation in mind, and experience such orientation in 3D?

The Stellarium Scenery3d plugin
allows you to see architectural 3D models
embedded in a landscape combined with the excellent representation of
a sky simulation provided by Stellarium. You can walk around, check
for (or demonstrate) possible astronomical alignments of ancient
architecture, see sundials and other shadow casters in action, etc.

\section{Usage}
\label{sec:Usage}


You activate the plugin with the \emph{circular enclosure} button at screen
bottom or by pressing [Ctrl+3]. The other button with circular
enclosure and tool icon (or [Ctrl+Shift+3]) opens the settings
dialog. Once loaded and displaying, you can walk around pressing
[Ctrl] plus cursor keys. Change eye height with [Ctrl]+[PgUp]/[PgDn]
keys. Adding [Shift] key increases speed by 10, [Alt] by 5 (pressing
both keys multiplies by 50!). If you release [Ctrl] before the cursor
key, animation will continue. (Press [Ctrl]+any cursor key to stop
moving.)\footnote{Bug or feature? I (GZ) had to change keyboard
  handling in the main program, somewhat breaking the plugin
  concept. TBD: Discuss with the core team.}

[Ctrl-K] toggles coordinate display. If you have georeferenced models
in a true geographical coordinate grid, e.g. UTM or Gauss-Krueger, you
will especially like this, and this makes the plugin usable for
scientific purposes. Display shows grid name, Easting, Northing,
Altitude of ground, and eye height above ground.

Press [Ctrl+Space] to toggle shadow display. Note that currently
textures are not displayed when shadow map is active. \footnote{ShadowMap is currently being improved by Andrei Borza/TUW.}

\section{Hardware Requirements}
\label{sec:HardwareRequirements}


In order to work with the advanced projection models in Stellarium,
this plugin uses a trick to create the foreground renderings: it
renders the scene into the six planes of a cube map which are then
drawn into the foreground. Your graphics card must be able to do this,
i.e., it must support the OpenGL extension called
\texttt{EXT\_framebuffer\_object}. Typical modern 3D cards (by NVidia
or ATI/AMD) support this extension. In case your graphics hardware
does not suppport it, it still works, but you are limited to
perspective projection, and the program will switch to it as soon as
you switch on, and switch back once you switch off the the Scenery3d
plugin.

You can influence rendering quality, but also speed, with 2 config
variables in the global \filename{config.ini}, which default to these values:
\begin{verbatim}
[Scenery3d]
cubemapSize                         = 1024
shadowmapSize                       = 1024
\end{verbatim}

Larger numbers will improve image quality. On fast computers, try
sizes of upto 4096, on slower ones maybe only 512, but always use a
power of 2.  Use \verb|cubemapSize=0| to switch off using the framebuffer
hardware extension, but you will be limited to perspective projection
in this case.


\section{Model Configuration}
\label{sec:ModelConfiguration}


The model format supported in Scenery3d is Wavefront .OBJ, which is
pretty common for 3D models.  You can use several modeling programs to
build your models. Software such as Blender, Maya, 3D Studio
Max etc.\ can export OBJ. 

A simple and cost-free modeling program is Google Sketchup, commonly
used to create the 3D buildings seen in Google Earth. It can be used
to create georeferenced models.  OBJ is not a native export format for
the standard version of Google Sketchup. If you are not willing to
afford Sketchup Pro, you have to find another way to export a textured
OBJ model.

One good exporter is available in the Kerkythea renderer project
available at \url{http://www.kerkythea.net/joomla/}.  You need \filename{SU2KT~3.17}
or better, and \filename{KT2OBJ~1.1.0} or better.  Deselect any selection, then
export your model to the Kerkythea XML format with settings shown in figure~\ref{fig:KerkytheaExportSettings}.
\begin{figure}[hb]
  \centering
\begin{tabular}{rl}
Geometry&Yes\\Lights&Yes\\Clay&No\\Photomatched&Yes\\DefaultUVs&No\\Instanced&No
\end{tabular}
\caption{Kerkythea Export Settings}
  \label{fig:KerkytheaExportSettings}
\end{figure}


%(Or, with selection enabled, make sure settings are No-Yes-Yes-No-Yes-No-No).  
You do not have to launch Kerkythea. Then,
use the \filename{KT2OBJ} converter to create an OBJ.  You can delete the XML
after the conversion.  Note that some texture coordinates may not be
exported correctly. The setting Photomatched:Yes seems now to have
corrected this issue, esp. with distorted/manu\-ally shifted textures.

(Almost) Working alternative: \filename{ObjExporter.rb} by author Honing.  Here,
export with settings 0xxx00. This will not create a \filename{TX...} folder but
dump all textures in the same directory as the OBJ and MTL
files. Unfortunately, this time some material assignments seem to be
bad. Also, it swaps Y/Z coordinates, but you can add a key to the
config file to correct swapped axes, see below. Other exporters may
also provide coordinates in any order of X, Y, Z -- all those can be
properly configured.

Recently, another free OBJ exporter has been made available by user TIG:
\filename{OBJexporter.rb}. However, as of version 1.2 it still has bad texture coordinates.

Yet another exporter, \filename{su2objmtl}, does also not provide good texture
coordinates and cannot be recommended at this time.

\subsection{Notes on OBJ file format limitations}
\label{sec:OBJlimitations}

The OBJ format supported is only a subset of the full OBJ format: Only
(textured) triangle meshes are supported, i.e., only lines containing statements:  
\cmd{mtllib}, \cmd{usemtl}, \cmd{v}, \cmd{vn}, \cmd{vt}, \cmd{f} (with three elements only!), \cmd{g}.
Negative vertex numbers (i.e., a specification of relative positions) are not supported.

A further requirement for correct illumination is that all vertices must have
vertex normals. Sketchup models exported with the Kerkythea plugin
should have correct normals. If you model does not provide them, you
can add vertex normals using e.g. meshlab (\url{http://www.meshlab.org}).  

Every \cmd{usemtl} statement must come after a \cmd{g} (group) statement and sets
the material for the respective face group. If necessary, you must
edit your OBJ file.

On reasonably good hardware (tested on a notebook PC with NVidia M9800
GTS), models up to 100.000 triangles are fluent, up to 250.000 are
still "interactive".  If display is too slow, switch to perspective
projection. All other projections require sixfold effort! 


\subsection{Configuring OBJ for Scenery3d}
\label{sec:Configuring}

The walkaround in your scene can use a ground level (piece of terrain)
on which the observer can walk. The observer eye will always stay ``eye
height'' above ground. Currently, there is no collision detection with
walls implemented, so you can easily walk through walls, or jump on
high towers, if their platform or roof is exported in the ground
layer. If your model has no explicit ground layer, walk will be on the
highest surface of the scenery layer.  If you use the special name
NULL as ground layer, walk will be above a zero-height level.

Technically, if your model has cavities or doors, you should export
your model twice. Once, just the ground plane, i.e. where you will
walk. Of course, for a temple or other building, this includes its
socket above soil, and any steps.  This plane is required to compute
eye position above ground. Note that it is not possible to walk in
several floors of a building, or in a multi-plane staircase. You may
have to export several ``ground'' planes and configure several scenery
directories for those rare cases.

The second export includes all visible model parts, and will be used for
rendering. Of course, this requires the ground plane again, but also
all building elements, walls, roofs, etc. 

If you have not done so by yourself, it is recommended to separate
ground and buildings into Sketchup layers in order to easily switch
the model to the right state prior to exporting.

Filename recommendations: 
\begin{verbatim}
<Temple>.skp        Name of a Sketchup Model file. 
                    (The "<>" brackets signal "use your own name here!")
                    The SKP file is not used by Scenery3d.
<Temple>.obj        Model in OBJ format. 
<Temple>_ground.obj Ground layer, if different from Model file. 
\end{verbatim}

OBJ export may also create folders \verb|TX_<Temple>| and
\verb|TX_<Temple>_ground|. You can delete the \verb|TX_<Temple>_ground| folder, 
\verb|<Temple>_ground.obj| is just used to compute vertical height.

Stellarium uses a directory to store additional data per-user. On Windows, this defaults to 
\verb|C:\Documents and Settings\<username>\Application Data\Stellarium|, but you can use another directory by using the command-line argument 
\cmd{--user-dir <USERDATA>}. We will refer to this directory.
Put the OBJ, MTL and TX directories into a directory, \\
\verb|<USERDATA>/Stellarium/modules/scenery3d/<Temple>|, and add a text file
called \texttt{scenery3d.ini} (This name is fixed!) with content described as follows.

% A Sketchup plugin "Write scenery3d.ini for Stellarium" will write this
% file. Locate the directory where the .obj file(s) reside(s), and store
% scenery3d.ini there. If you have other modelers and models, or if your
% model is not georeferenced in Sketchup, write the file yourself and
% use the following format.
%
% TBD GZ: Write this Sketchup export plugin!

\begin{verbatim}

[model]
name=<Temple>                Unique ID within all models in scenery3d directory.
                             Recommendation: use directory name.
landscape=<landscapename>    Name of an available Stellarium landscape.
\end{verbatim}
This is required if the landscape file includes geographical
coordinates and your model does not: First, the location coordinates
of the Landscape file are used, then location coordinates given here.
The landscape also provides the background image of your scenery. - If
you want a zero-height (mathematical) horizon, use the provided
landscape called \filename{Zero}.
\begin{verbatim}
scenery=<Temple>.obj         The complete model, including visible ground.
ground=<Temple>_ground.obj   Optional: separate ground plane. (NULL for zero altitude.)
description=<Description>    What you can see in this scenery, what to 
                             look for, historical context, etc.
author=<Your Name yourname@yourplace.com>
copyright=<Copyright Info>

obj_order=XYZ      | Use this if you have used an exporter which swaps Y/Z coordinates.
                   | Defaults to XYZ, other options: XZY, YZX, YXZ, ZXY, ZYX


[location]
\end{verbatim}
Optional section to specify geographic longitude $\lambda$, latitude
$\varphi$, and altitude. Required if\\
\verb|coord/convergence_angle==from_grid|, else location is inherited
from landscape.
\begin{verbatim}
planet = Earth
latitude = +48d31'30.4"      ; Required if coord/convergence_angle==from_grid
longitude = +16d12'25.5"     ; "--"
altitude =from_model|<int>   ; 
\end{verbatim}
altitude (for astronomical computations) can be computed from the
model:  if \verb|from_model|, it is computed as $(z_{min}+z_{max})/2+\mathtt{orig\_H}$,
i.e. from the model bounding box centre height.

\begin{verbatim}
display_fog = 0
atmospheric_extinction_coefficient = 0.2
atmospheric_temperature = 10.0
atmospheric_pressure = 1013.0
light_pollution = 1

[coord]
\end{verbatim}

Entries in the \verb|[coord]| section are again optional, default to zero when not specified, but are
required if you want to display meaningful eye coordinates in your
survey (world) coordinate system, like UTM or Gauss-Krueger.  

\begin{verbatim}
grid_name=<string> 
\end{verbatim}
Name of grid coordinates, e.g. \texttt{``UTM 33 U (WGS 84)''}, \texttt{``Gauss-Kr�ger M34''} or \verb|``Relative to <Center>''| 
                       This name is only displayed, there is no evaluation of its contents.

\begin{verbatim}
orig_E=<double> | (Easting)  East-West-distance to zone central meridian
orig_N=<double> | (Northing) North distance from Equator
orig_H=<double> | (Height)   Altitude above Mean Sea Level of model origin
\end{verbatim}
These entries describe the offset, in metres, of the model coordinates relative to coordinates in a geographic grid, like Gauss-Kr�ger.
If you have your model vertices specified in grid coordinates, do not specify \verb|orig_...| data, 
but please add \verb|start_...| data, below. 


\begin{verbatim}
convergence_angle=from_grid|<double> 
grid_meridian=<double>|+<int>d<int>'<float>"      
\end{verbatim}
Typically, digital elevation models and building structures built on those are survey-grid aligned, so true
geographical north will not coincide with grid north, the difference
is known as meridian convergence. 
\begin{equation}
\gamma(\lambda, \varphi)=\arctan(\tan(\lambda-\lambda_0)\sin\varphi)
\end{equation}
This amount can be given in \verb|convergence_angle|
(degrees), so that your model will be aligned with True North\footnote{%
\url{http://en.wikipedia.org/wiki/Transverse_Mercator_projection}}.
Central meridian $\lambda_0$ of grid zone, e.g. for UTM or Gauss-Kr�ger. 
\verb|grid_meridian| is  only required to compute convergence angle if \verb|convergence_angle="from_grid"|

\begin{verbatim}
zero_ground_height=<double> 
\end{verbatim}
height of terrain outside \filename{ground.OBJ}, or if \verb|ground=NULL|. Allows smooth
approach from outside.  This value is relative to the model origin, or
typically close to zero, i.e.,  use a Z value in model coordinates, not
world coordinates! (If you want the terrain height surrounding
your model to be \verb|orig_H|, use 0, not the correct mean height above sea
level!)  Defaults to minimum of height of ground level (or model, resp.)
bounding box.

\begin{verbatim}
start_E=<double> 
start_N=<double> 
start_H=<double> /* only meaningful if ground==NULL, else H is derived from ground */
start_Eye=<double> /* default: 1.65m */
start_az_alt_fov=<az_deg>,<alt_deg>,<fov_deg> /* initial view direction and field of view.*/
\end{verbatim}
\verb|start...| coordinates to be set after loading the scenery. Default to center of model boundingbox. 

It is advisable to use the grid coordinates of the location of the panoramic photo ("landscape") as \verb|start_..| coordinates, 
or the correct coordinates and some carefully selected \texttt{start\_az\_alt\_fov} in case of certain view corridors (temple axes, \ldots).


\subsection{Working with non-georeferenced OBJ files}
\label{sec:NonGeoreferenced}


There exists modeling software which produces nice models, but without
concept of georeference. One spectacular example is AutoDesk PhotoFly,
a cloud application which delivers 3D models from a bunch of photos
uploaded via its program interface. This ``technological preview'' is
in version 2 and free of cost as of mid-2011.

The problem with these models is that you cannot assign surveyed
coordinates to points in the model, so either you can georeference the
models in other applications, or you must find the correct
transformation matrix.  Importing the OBJ in Sketchup may take a long
time for detailed photo-generated models, and the texturing may
suffer, so you can cut the model down to the minimum necessary e.g.\ in
Meshlab, and import just a stub required to georeference the model in
Sketchup. 

Now, how would you find the proper orientation? The easiest chance
would be with a structure visible in the photo layer of Google
Earth. So, start a new model and immediately "add location" from the
Google Earth interface. Then you can import the OBJ with TIG's importer
plugin.  If the imported model looks perfect, you may just place the
model into the Sketchup landscape and export a complete landscape just
like above. If not, or if you had to cut/simplify the OBJ to be able
to import it, you can rotate/scale the OBJ (it must be grouped!). If
you see a shadow in the photos, you may want to set the date/time of
the photos in the scene and verify that the shadows created by
Sketchup illuminating the model match those in the model's photo
texture. When you are satisfied with placement/orientation, you create
a \filename{scenery3d.ini} like above with the command
\cmd{Plugins->ASTROSIM/Stellarium scenery3d helpers->Create scenery3d.ini}.

Then, you select the OBJ group, open \cmd{Windows->Ruby Console} and call
\cmd{Plugins->ASTROSIM/Stellarium scenery3d helpers->Export transformation
of selected group (e.g., from PhotoFly import)}.

On the Ruby console, you will find a line of numbers (the $4\times4$ transformation matrix) which you
copy/paste into the \filename{[model]} section in \filename{scenery3d.ini}. 
\begin{verbatim}
obj2grid_trafo=<a11>,<a12>,<a13>,<a14>,<a21>,<a22>,<a23>,<a24>,<a31>,<a32>,<a33>,<a34>,<a41>,<a42>,<a43>,<a44>
\end{verbatim}
You edit the \filename{scenery3d.ini} to use your full (unmodified)
PhotoFly model and, if you don't have a panorama, take \filename{Zero}
landscape as (no-)background. It depends on the model if you want to
be able to step on it, or to declare \verb|ground=NULL| for a
constant-height ground. Run Stellarum once and adjust the
\verb|start_N|, \verb|start_E| and \verb|zero_ground_height|.

\subsubsection{Rotating OBJs with recognized survey points}
\label{sec:RotatingOBJ}


If you have survey points measured in a survey grid plus a photomodel
with those points visible, you can use Meshlab to find the model
vertex coordinates in the photo model, and some other program like
CoordTrans in the JavaGraticule3D suite to find either the matrix
values to enter in \filename{scenery3d.ini} or even rotate the OBJ
points. However, this involves more math than can be described here;
if you came that far, you likely know the required steps.  Here it
really helps if you know how to operate automatic text processors like
AWK.

\section*{Authors and Acknowledgements}
\label{Acknowledgments}


Scenery3d was conceived by Georg Zotti for the Astrosim project. It
was implemented (mostly) in 2010/2011 by Simon Parzer and Peter
Neubauer as student work supervised by Michael Wimmer (TU
Wien). Improvements in integration, user interaction, .ini option
handling, OBJ/MTL loader bugfixes and georeference testing by Georg
Zotti. Andrei Borza (again supervised by Michael Wimmer) in 2011
further improved rendering quality and speed.

This work has been created during the ASTROSIM project supported
2008-2011 by the Austrian Science Fund (FWF) under grant number
P~21208-G19. 




\end{document}
